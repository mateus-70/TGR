\documentclass[../main.tex]{subfiles}
\begin{document}
\chapter{INTRODUÇÃO} % LETRAS MAIÚSCULAS


% A introdução apresenta os objetivos do trabalho, bem como as razões de sua elaboração. Tem caráter didático de apresentação.
% Deve abordar:

%     a) o problema de pesquisa, proposto de forma clara e objetiva;
    
%     b) os objetivos, delimitando o que se pretende fazer;
    
%     c) a justificativa, destacando a importância do estudo;
    
%     d) apresentar as definições e conceitos necessários para a compreensão do estudo;
    
%     e) apresentar a forma como está estruturado o trabalho e o que contém cada uma de suas partes.
    
% O desenvolvimento é a demonstração lógica de todo o trabalho, detalha a pesquisa ou o estudo realizado. Explica, discute e demonstra a pertinência das teorias utilizadas na exposição e resolução do problema. 

% O desenvolvimento pode ser subdivido em seções e subseções com nomenclaturas definidas pelo autor conforme conteúdo apresentado. 

Os números fazem parte do dia-a-dia das pessoas no mundo contemporâneo. Não apenas pela sua utilização nas ciências naturais, onde são usados para explicar fenômenos naturais. Também são utilizados para o estudo de fenômenos das chamadas ciências humanas. Além disso, os números aparecem como uma ferramenta para o manuseio do dinheiro e do controle do tempo, coisas que na sociedade contemporânea são essenciais.

Na prática, normalmente não nos perguntamos o que é um número, e por que as operações com números funcionam da maneira como funcionam. Por outro lado, é essencial para um professor de matemática saber por que podemos usar números de determinadas maneiras e não de outras, as limitações que cada tipo de número tem, face a cada tipo de problema. Um caso essencial é que os números racionais não são suficientes ou adequados para medir a diagonal de um quadrado de lado $1$. Por outro lado, a natureza intrínseca do número não será considerada por nós, uma vez que a pergunta "o que é um número"\ é mais  filosófica do que matemática.

A partir das limitações dos conjuntos numéricos utilizados, surge a necessidade humana de estender o conceito de número. Entender como essas extensões funcionam, em especial as extensões do conceito de número natural para os números inteiros e para os números racionais é importante para um professor de matemática. Nesse contexto, o objetivo principal deste trabalho é apresentar as extensões do conceito de número, para poder justificar a maneira como operamos com eles cotidianamente. Um dos objetivos específicos é apresentar a justificativa para as inclusões $\mathbb{N} \subset \mathbb{Z} \subset \mathbb{Q} \subset \mathbb{R}$. Num sentido análogo, mostraremos que o conjunto $\mathbb{R}$ é único. 

O trabalho iniciará com os axiomas de Peano, e terminará com o estudo dos números reais. Escolhemos iniciar com os axiomas de Peano por dois motivos: o primeiro é por causa da simplicidade de trabalhar com eles, o segundo é devido à necessidade de escolher um ponto de partida para iniciar o trabalho. Poderíamos ter iniciado desenvolvendo uma teoria de conjuntos suficiente para poder provar os axiomas de Peano, mas seria necessário para isso, ainda assim, assumir a lógica e os axiomas da teoria de conjuntos. Caso fosse optado por iniciar com a própria lógica ou pela teoria de conjuntos, o trabalho perderia seu foco principal, que é e justificar propriedades relacionadas aos diferentes conjuntos numéricos. Uma abordagem para a dedução dos axiomas de Peano a partir da teoria de conjuntos pode ser encontrada em \textcite{suppes}.

A metodologia usada no desenvolvimento deste trabalho é a pesquisa bibliográfica, que incluirá em especial os temas de aritmética, álgebra e análise real. Para o tema de aritmética, as referências principais são \textcite{ferreira} e \textcite{domingues-2009}. A referência principal para álgebra é \textcite{domingues-iezzi-2018}. As referências principais para análise real são \textcite{guidorizzi} e \textcite{lima-analise-1}.

O trabalho está dividido em sete capítulos, começando com esta introdução. O segundo capítulo contém definições e teoremas básicos de álgebra, que têm o objetivo de centralizar definições que são usadas ao longo do restante do texto. O terceiro capítulo aborda o conjunto dos números naturais, iniciando pelos axiomas de Peano, depois definimos uma adição, uma multiplicação e uma relação de ordem nesse conjunto, e provamos algumas propriedades aritméticas básicas. Vale destacar que neste trabalho, o $0$ (zero) não será considerado número natural, pois pretende-se mostrar que as extensões para os conjuntos dos números inteiros, racionais e reais pode ser feita também com essa escolha. As bibliografias das áreas de aritmética e álgebra normalmente preferem considerar o $0$ como um número natural, pois isso simplifica a obtenção de alguns resultados. O quarto capítulo trabalha com os números inteiros, que são obtidos por meio de classes de equivalência de uma relação adequadamente estabelecida com números naturais. São definidas uma adição, uma multiplicação e uma relação de ordem, e provadas algumas propriedades básicas. No quinto capítulo estudamos os números racionais, que são construídos por meio de classes de equivalência de uma relação envolvendo números inteiros. Novamente, são definidas uma adição, uma multiplicação e uma relação de ordem, e provadas propriedades aritméticas básicas. O sexto capítulo constrói o conjunto dos números reais. A literatura clássica de análise aborda dois métodos principais para a construção dos números reais, que são via Cortes de Dedekind e via Sequências de Cauchy. Optamos por seguir a abordagem de cortes de Dedekind, pois temos uma familiaridade maior com o manuseio de conjuntos, do que com sequências. O sétimo capítulo apresenta dois resultados importantes sobre os números reais, o primeiro deles é a sua não enumerabilidade. O segundo estabelece a unicidade do conjunto dos números reais, num contexto de isomorfismos de corpos ordenados completos. 

\end{document}