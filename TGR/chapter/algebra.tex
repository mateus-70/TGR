\documentclass[../main.tex]{subfiles}
\begin{document}
\chapter{ÁLGEBRA BÁSICA}

Neste capítulo apresentaremos as definições usadas ao longo do texto e também apresentaremos alguns resultados importantes que, embora simples, podem ser generalizados para os conjuntos (numéricos) que trabalharemos. A referência principal para este capítulo é \textcite{domingues-iezzi-2018}. 

Assumimos familiaridade com a Teoria dos Conjuntos e com os símbolos e ideias básicas da lógica tais como implicação, conjunção e disjunção. Com essas noções assumidas, desenvolvemos esse capítulo com objetivo de centralizar definições e resultados.

% O objetivo aqui é colocar uma fundamentação inicial mais sólida para os capítulos subsequentes, para que o conteúdo principal (a construção dos conjuntos numéricos) fique melhor consolidada. 

\section{Conjuntos e relações}
Neste primeiro momento vamos abordar conjuntos, pares ordenados e relações, e veremos algumas definições básicas.
% É claro que no estudo mais profundo de conjuntos precisaríamos de mais axiomas e outras definições, mas foge do escopo do nosso trabalho.

\begin{defi}\label{agb-def-parOrdenado}
     Dados um conjunto não vazio $A$ e $a,b \in A$, definimos o par ordenado $(a,b)$ como o conjunto $\{\{a\}, \{a,b\}\}$.
\end{defi}
    O elemento $a$ dizemos que está na primeira entrada do par ordenado, e o $b$ está na segunda entrada do par ordenado.
    
    Essa definição de par ordenado visa "corrigir"\ o problema que surge quando associamos o par ordenado $(a,b)$ com o conjunto $\{a,b\}$, pois  $\{a,b\}$ e $\{b,a\}$, com $b$ diferente de $a$, são o mesmo conjunto.
    A \Cref{agb-def-parOrdenado} é satisfatória para esse objetivo, pois para nós o que irá importar é o lema a seguir:

\begin{lema}\label{agb-lema-parOrdenado}
    Dois pares ordenados $(a,b)$ e $(c,d)$ são iguais se, e somente se, $a=c$ e $b=d$.
\end{lema}
A demonstração do \Cref{agb-lema-parOrdenado} pode ser encontrada em \textcite[p. 42]{suppes}.

\begin{ex}
    Consideremos os pares ordenados $(1,2)$ e $(2,1)$. Eles não são iguais porque $1 \neq 2$ e $2 \neq 1$. Podemos observar também que os pares ordenados $(1,2)$ e $(1,1)$ não são iguais pois embora $1=1$, $2 \neq 1$, e basta que uma coordenada seja diferente para que o par como um todo, seja diferente.
\end{ex}

\begin{defi}\label{agb-def-produtoCartesiano}
     Dados dois conjuntos não vazios $A,B$, o produto cartesiano de A por B, denotado por $A \times B$, é o conjunto de todos os pares ordenados $(a,b)$ onde $a \in A$ e $b \in B$, isto é, $A \times B = \{ (a,b) : a \in A \land b \in B \}$.
\end{defi}
\begin{ex}
    Consideremos os conjuntos $A = \{1,5,6\}$ e o conjunto $B = \{1,3,4,9\}$. Temos que 
    \[ A \times B = \{(1,1), (1,3), (1,4), (1,9), (5,1), (5,3), (5,4), (5,9), (6,1), (6,3), (6,4), (6,9)\}.\] 
    Já se \[ A = \{-1,3,4\} \text{ e } B = \{0,5\},\] então 
    \[ A \times B = \{(-1,0), (-1,5), (3,0), (3,5), (4,0), (4,5)\}. \]
\end{ex}
    

\begin{defi}\label{agb-def-relacaoBinaria}
    Uma relação binária $R$ num conjunto $A$ é qualquer subconjunto do produto cartesiano $A \times A$, isto é, $R \subset A \times A$.
\end{defi}

Frequentemente utiliza-se a notação $aRb$ para indicar que o par ordenado $(a,b) \in R$.

\begin{ex}
    No conjunto $\mathbb{R}$ a relação binária definida por uma função quadrática, em que a parábola $P \subset \mathbb{R} \times \mathbb{R}$ é $P = \{(a,a^2) : a \in \mathbb{R} \}$. Embora nesses casos de funções não seja comum utilizar a notação $(a,a^2) \in P$ ou $a P a^2$.    
\end{ex}
\begin{ex}
    A desigualdade entre elementos de um conjunto numérico é uma relação binária. Em geral, opta-se pela utilização do símbolo da relação entre os elementos $a<b$, ao invés de usar a notação de pares ordenados $(a,b)$.
\end{ex}

\begin{defi}
    Seja $A$ um conjunto não vazio e seja $R$ uma relação sobre $A$. Dizemos que a relação $R$ tem a propriedade:
    \begin{enumerate}[label=(\roman*)]
        \item Reflexiva, quando para qualquer $a \in A$ valer $aRa$;
        \item Simétrica, quando para quaisquer $a,b \in A$, valer $aRb \implies bRa$.
        \item Antissimétrica, quando para quaisquer $a,b \in A$, valer ($a R b \land bRa) \implies a = b$;
        \item Transitiva, quando para quaisquer $a,b,c \in A$, valer $(aRb \land bRc) \implies aRc$;
        \item Totalidade, quando para quaisquer $a,b \in A$, valer $aRb \lor bRa$.
    \end{enumerate}
\end{defi}

Trabalharemos com dois tipos especiais de relações: as relações de equivalência e as relações de ordem. As relações de equivalência são necessárias para a criação dos conjuntos $\mathbb{Z}$ e $\mathbb{Q}$. Definiremos uma relação de ordem apropriada para cada um dos conjuntos numéricos que construiremos nos próximos capítulos.


\begin{defi}\label{agb-def-relacaoEquivalencia}
    Uma relação $R$ é chamada de relação de equivalência, quando possuir as seguintes propriedades:
    \begin{enumerate}[label=(\roman*)]
        \item reflexiva;
        \item simétrica;
        \item transitiva.
    \end{enumerate}
\end{defi}

\begin{ex}\label{agb-ex-restosDiv3}
    Seja $A = \{0,1,2,3,4,5,6,7,8\}$. A relação 
    \begin{align*}
        R = \{ & (0,0), (0,3), (0,6), (3,0), (3,3), (3,6), (6,0),(6,3), (6,6), \\
    & (1,1), (1,4), (1,7), (4,1), (4,4), (4,7), (7,1), (7,4), (7,7), \\
    & (2,2), (2,5), (2,8), (5,2), (5,5), (5,8), (8,2), (8,5), (8,8) \}
    \end{align*}
    é uma relação de equivalência, pois cumpre os critérios da \Cref{agb-def-relacaoEquivalencia}.    
\end{ex}


\begin{defi}\label{agb-def-classeEquivalencia}
    Seja $R$ uma relação de equivalência num conjunto $A$ não vazio e seja $a \in A$ um elemento fixado arbitrariamente. O conjunto \\
    \[ \Bar{a} = \{x \in A : xRa\} \]
    chama-se classe de equivalência de $a$ pela relação R.
\end{defi}

\begin{ex}\label{agb-ex-classeEquivalenciaDivisaoEuclidiana}
    Podemos citar o resto da divisão em $\mathbb{N} \cup \{0\}$. A divisão euclidiana de $a$ por $b$ deixa resto $r$ quando $n \cdot b + r = a$ para algum natural $n \in \mathbb{N} \cup \{0\}$. Se fixarmos $b = 4$, e atribuindo valores para $a$ obtemos:
    \begin{align*}
        a = 0 &= 0 \cdot 4 + 0, \\
        a = 1 &= 0 \cdot 4 + 1, \\
        a = 2 &= 0 \cdot 4 + 2, \\
        a = 3 &= 0 \cdot 4 + 3, \\
        a = 4 &= 1 \cdot 4 + 0, \\
        a = 5 &= 1 \cdot 4 + 1, \\
        a = 6 &= 1 \cdot 4 + 2, \\
        a = 7 &= 1 \cdot 4 + 3, \\
        a = 8 &= 2 \cdot 4 + 0, \\
        a = 9 &= 2 \cdot 4 + 1, \\
        a = 10 &= 2 \cdot 4 + 2, \\
        a = 11 &= 2 \cdot 4 + 3. \\
    \end{align*}
    % Neste caso temos que os restos são iguais a $0$ para os valores de $a \in \{\,0,4,8\,\}$. Os restos são iguais a $1$ para os valores de $a \in \{\,1,5,9\,\}$.
    Observando essas divisões podemos perceber que são $4$ números possíveis para $r$, com $ r \in \{\,0,1,2,3\,\}$.
    Assim temos $4$ classes de equivalência, que são os valores possíveis para o resto $r$, que são as classes $\Bar{0}, \Bar{1}, \Bar{2}$ e $\Bar{3}$. A classe de resto do $4$, que é $\Bar{4}$, é a mesma classe do $\Bar{0}$. Vamos considerar $r=3$, e vamos expressar essa classe como um conjunto, temos $\Bar{3} = \big\{\,a \in \mathbb{N} \cup \{0\} : aR3\,\big\}$, em que $R$ é a relação de equivalência sobre $\mathbb{N} \cup \{0\}$, definida por $mRn$ se, e somente se, $m$ e $n$ tem o mesmo resto na divisão por $4$.
\end{ex}

% \begin{ex} \todo{melhorar, primeiro exibir as classes de cada elemento}
%     Vejamos agora a classe de restos da divisão em $\mathbb{N}$. Na verdade é o que já foi mostrado no \Cref{agb-ex-restosDiv3}, mas note as entradas dos pares ordenados são os números que deixam o mesmo resto na divisão por $3$. Por exemplo, $1 \divisionsymbol 3$ tem resto $1$, assim como $4 \divisionsymbol 3$, assim $1R4$, e a classe de equivalências podemos denotar $\Bar{1}$ ou $\Bar{4}$, que é possível pelo resultado do teorema a seguir.
% \end{ex}


\begin{teo}\label{agb-teo-relacaoEquivalenciaPropriedades}
    Seja $R$ uma relação de equivalência em um conjunto não vazio $A$ e sejam $a,b$ elementos quaisquer de $A$, então:
    \begin{enumerate}[label=(\roman*)]
        \item $a \in \Bar{a}$;
        \item $\Bar{a} = \Bar{b} \iff aRb$;
        \item $\Bar{a} \neq \Bar{b} \implies \Bar{a} \cap \Bar{b} = \emptyset $.
    \end{enumerate}
\end{teo}

\begin{dem}
    Seja $R$ uma relação de equivalência. Tem-se que
    \begin{enumerate}[label=(\roman*)]
        \item $a \in A$ e como $R$ é reflexiva, vale $aRa$, assim $a \in \Bar{a}$.
        % \item Pelo item anterior, $a \in \Bar{a} = \Bar{b}$. 
        % Agora tentemos montar uma contradição, se é falso que $aRb$ então $a \not\in \Bar{b}$. Mas sabemos que $a \in \Bar{b}$ assim é verdade que $aRb$. 
        \item Para a ida, pelo item anterior temos $a \in \overline{a}$ logo $a \in \overline{b} = \overline{a}$, assim $aRb$. Para provar a volta, temos que $aRb$, e por contradição suponhamos sem perda de generalidade, que exista $c \in \Bar{a} \setminus \Bar{b}$. Daí temos $cRa$ e $aRb$. Como $R$ é transitiva temos $cRb$, desse modo $c \in \Bar{b}$ o que é uma contradição.
        \item Por contradição suponhamos que $\Bar{a} \cap \Bar{b} \neq \emptyset$. Assim existe $c \in \Bar{a} \cap \Bar{b}$, então $cRa$ e $cRb$. Como  $R$ é simétrica, vale $aRc$ e $cRb$, e como também é transitiva tem-se que $aRb$, pelo item anterior $\Bar{a} = \Bar{b}$, o que conclui a demonstração.
    \end{enumerate}
\end{dem}

\begin{defi}\label{agb-def-conjuntoQuociente}
    Seja $R$ uma relação de equivalência num conjunto $A$. O conjunto constituído das classes de equivalência em $A$ pela relação $R$ é denotado por $A / R$ e denominado conjunto quociente de $A$ por $R$. Assim
    \[ A / R = \{\Bar{a} : a \in A\}. \]
    
\end{defi}
O conjunto quociente serve para particionar o conjunto em subconjuntos distintos. No \Cref{agb-ex-classeEquivalenciaDivisaoEuclidiana} consideramos o resto na divisão por $4$ no conjunto $\mathbb{N} \cup \{0\}$. Notamos que formamos $4$ classes de equivalência distintas, e qualquer número natural pertence a uma, e apenas uma classe de equivalência.

\begin{ex}
    No \Cref{agb-ex-classeEquivalenciaDivisaoEuclidiana}, utilizando a notação de conjunto quociente e classes de equivalência, obtemos:
    \[ \mathbb{N} \cup \{0\} / R = \{\, \Bar{a} : a \in \mathbb{N} \cup \{0\} \,\} = \{\, \Bar{0}, \Bar{1}, \Bar{2}, \Bar{3} \,\}.\]
\end{ex}

\begin{defi}\label{agb-def-relacaoOrdemParcial}
    Seja $R$ uma relação sobre $A$. Ela é chamada de relação de ordem parcial se valer as propriedades:
    \begin{enumerate}[label=(\roman*)]
        \item Reflexiva;
        \item Antissimétrica;
        \item Transitiva.
    \end{enumerate}
\end{defi}

Além da relação de ordem habitual, dada por $aRb \iff a \leq b$, consideramos em $\mathbb{N}$ a relação de ordem parcial definida por $a R b \iff a | b$ (a divide b). Note que $a$ divide $a$, se $a | b$ e $b | c$, então $a | c$, e que $a | b \land b | a \implies a=b$. A prova dessas afirmações será omitida neste texto, mas podem ser encontradas em \textcite[p. 52]{domingues-2009}.

\begin{notacao}
    Utilizaremos a notação $a < b$ quando $a \leq b$ mas $a \neq b$. Também utilizaremos a notação $b \geq a$ e $b > a$ quando $a \leq b$ e $a < b$, respectivamente.
\end{notacao}

\begin{defi}
    Sejam $a,b \in A$, os elementos $a$ e $b$ são ditos comparáveis através da relação de ordem parcial $\leq$, caso $a \leq b$ ou caso $b \leq a$. 
\end{defi}

\begin{defi}\label{agb-def-relacaoOrdemTotal}
    Uma relação de ordem parcial $\leq$ sobre um conjunto $A$ é chamada de relação de ordem total caso quaisquer elementos de $A$ sejam comparáveis através da relação $\leq$.
\end{defi}

% Considerando a estrutura do nosso trabalho, com a \Cref{agb-def-relacaoOrdemTotal} esclarecemos que utilizaremos apenas relações de ordem que são totais, em especial as relações de ordem usuais em $\mathbb{N}$, $\mathbb{Z}$, $\mathbb{Q}$ e $\mathbb{R}$ são todas relações de ordem totais. 

Trabalharemos apenas com as relações de ordem usuais nos conjuntos $\mathbb{N}$, $\mathbb{Z}$, $\mathbb{Q}$ e $\mathbb{R}$. Provaremos nos próximos capítulos que essas relações de ordem são totais.

Usaremos os significados usuais para $=, <, \leq$. Também trabalharemos com a notação usual $a \leq b$ em vez de colocar $(a,b) \in \leq$, que embora correto, fica estranho. 

\begin{defi}\label{agb-def-relacaoTricotomica}
    Seja $A$ um conjunto. Uma relação de ordem parcial $\leq$ sobre $A$ é dita tricotômica,
    quando para quaisquer $a,b \in A$, valer um e apenas um, dos casos a seguir: ou $a < b$ ou $b < a$ ou $a = b$.
\end{defi}

\begin{defi}\label{agb-def-cotaSuperior}
    Sejam $A$ e $E$ conjuntos, com $\emptyset \neq A \subset E$, e $\leq$ uma relação de ordem parcial sobre $E$. Um elemento $L$ de $E$ é chamado de cota superior de $A$ se para qualquer elemento $a \in A$ vale $a \leq L$.
\end{defi}

\begin{defi}\label{agb-def-cotaInferior}
    Sejam $A$ e $E$ conjuntos, com $\emptyset \neq A \subset E$, e $\leq$ uma relação de ordem parcial sobre $E$. Um elemento $l$ de $E$ é chamado de cota inferior de $A$ se para qualquer elemento $a \in A$ vale $l \leq a$.
\end{defi}
\begin{ex}\label{agb-ex-cotasSuperiorInferior}
    Considere o conjunto $A = \{\,1,2,3,4,5\,\} \subset \mathbb{N}$. Esse conjunto é limitado superiormente pelo $5$, e inferiormente pelo $1$. Notemos que os números naturais $6,7,8$ também são cotas superiores desse conjunto. Por outro lado, o número $-1 \in \mathbb{Z}$ é uma cota inferior que não é um número natural. 
\end{ex}

\begin{defi}\label{agb-def-maximo}
     Sejam $A$ e $E$ conjuntos, com $\emptyset \neq A \subset E$, e $\leq$ uma relação de ordem parcial sobre $E$. Um elemento $M$ de $A$ é chamado de máximo de $A$ se para qualquer elemento $a \in A$ vale $a \leq M$. Em outras palavras, um máximo é uma cota superior que pertence ao conjunto $A$.   
\end{defi}
\begin{defi}\label{agb-def-minimo}
     Sejam $A$ e $E$ conjuntos, com $\emptyset \neq A \subset E$, e $\leq$ uma relação de ordem parcial sobre $E$. Um elemento $m$ de $A$ é chamado de mínimo de $A$ se para qualquer elemento $a \in A$ vale $m \leq a$, assim, um mínimo é uma cota inferior que pertence ao conjunto $A$.   
\end{defi}
\begin{ex}
    No \Cref{agb-ex-cotasSuperiorInferior}, $1$ é mínimo de $A$, $5$ é máximo de $A$ e $6$ é uma cota superior mas não é máximo de $A$, pois $6 \not\in A$.
\end{ex}

\begin{teo}\label{agb-teo-maximoUnico}
    Seja $A \neq \emptyset$ um subconjunto de um conjunto parcialmente ordenado $E$. Se existir um máximo para $A$ então esse elemento é único.
\end{teo} 
\begin{dem}
    Sejam $\leq$ uma relação de ordem sobre $E$, e $M_1$ e $M_2$ dois máximos de $A$. Temos $M_1, M_2 \in A$, e também $M_1 \leq M_2$  e $M_2 \leq M_1$. Como uma relação de ordem é antissimétrica, concluí-se que $M_1 = M_2$.
\end{dem}

%%%%%%%%%%%%%%%%%%%%%%%%%%%%%%%%%%%%%%%%%%%%%%%

\section{Operações}

Nesta seção apresentaremos algumas definições que tem como objetivo fundamentar o conceito de operação. Intuitivamente o conceito de operação é tomar algum elemento qualquer e fazer algum processo sobre ele. Como exemplo, temos a operação de negação da lógica, ou a operação de troca de sinal para um número inteiro. Nesses dois casos trata-se de uma operação que utiliza um elemento para chegar em outro elemento (como resultado, resposta).

Consideraremos as operações binárias, isto é, que utilizam dois elementos como entradas. De acordo com a próxima definição, o resultado será fixo, exato e único.
\begin{defi}\label{agb-def-operacao}
    Seja $A$ um conjunto. Uma operação $*$ sobre $A$ é uma função que a cada $a,b \in A$ associa um único elemento $a * b \in A$, ou seja, associa a cada dois elementos em $A$ a sua imagem $a * b$, que também é um elemento de $A$.
\end{defi}

\begin{obs}\label{agb-obs-operacao}
    Como $a*b = c$, o elemento $c$ será fixo se $a$ e $b$ também forem fixos. O elemento $c$ é também único, pois a operação é uma função, e com exato queremos dizer que ele existe, é fixo e é único. 
\end{obs}

Pela \Cref{agb-def-operacao}, a adição em $\mathbb{N}$, $\mathbb{Z}$, $\mathbb{Q}$ e $\mathbb{R}$ são operações, sobre cada um desses conjuntos, cada um com sua adição. Por outro lado, a subtração em $\mathbb{N}$ não é uma operação em $\mathbb{N}$ pois $1-2 \not \in \mathbb{N}$. A divisão não é uma operação em $\mathbb{N}$ e em $\mathbb{Z}$, pelo mesmo motivo.

\begin{defi}
    Seja $A$ um conjunto e seja $*$ uma operação em $A$. Se existir um $\mathfrak{e} \in A$ tal que, para qualquer $a \in A$, valer $\mathfrak{e} * a = a$ dizemos que $\mathfrak{e}$ é elemento neutro à esquerda da operação $*$. 
\end{defi}
Por analogia definimos elemento neutro à direita da operação $*$ como $\mathfrak{e'} \in A$ tal que $a * \mathfrak{e'} = a$ para qualquer $a$ em A.

\begin{defi}\label{agb-def-propriedades}
    Seja $A$ um conjunto e seja $*$ uma operação sobre $A$. Dizemos que a operação $*$ tem a propriedade:
    \begin{itemize}
        \item associativa, quando para quaisquer $a,b,c \in A$ é válido que $ a * ( b * c ) = ( a * b ) * c$.
        \item comutativa, quando para quaisquer $a,b \in A$ é válido que $a * b = b * a$.
        \item da existência do elemento neutro, quando existe $\mathfrak{e} \in A$, em que $\mathfrak{e}$ é neutro à esquerda e à direita. Ou seja, $\mathfrak{e} * a = a * \mathfrak{e} = a$, para qualquer $a \in A$. 
        \item da existência do elemento simétrico, quando para qualquer $a \in A$ existe algum $d \in A$, em que é válido que $a * d = d * a = \mathfrak{e}$, em que $\mathfrak{e}$ é o elemento neutro de $*$.
        \item do fechamento, quando para quaisquer $a, b \in A$ ocorre que $a * b \in A$.   
    \end{itemize}
\end{defi}
Usualmente o elemento neutro da soma é denotado por $0$ e o elemento neutro do produto é denotado por $1$ (nós adotaremos a notação usual). Além disso, denotaremos o elemento neutro de uma operação $*$ por $\mathfrak{e}$, onde o contexto deixará claro que estaremos nos referindo ao elemento neutro por apenas um lado, caso ainda não esteja provado que se trata de um elemento neutro de ambos os lados. Quando omitido o lado, considera-se que é elemento neutro de ambos os lados.

A propriedade do fechamento na verdade é imediata da definição de operação, apenas a destacamos na \Cref{agb-def-propriedades} para realçar o fato de que a operação tem imagem em $A$ e não qualquer elemento que não esteja em $A$.
\begin{obs}
    Definiremos duas operações principais para cada um dos conjuntos $\mathbb{N}, \mathbb{Z}, \mathbb{Q}$ e $\mathbb{R}$. Para cada um desses conjuntos teremos uma adição (soma) e uma multiplicação (produto). O elemento simétrico da \Cref{agb-def-propriedades} continuará sendo chamado de simétrico caso seja uma adição, e passará a ser chamado de inverso, quando se tratar de multiplicação.
\end{obs}
Poderemos omitir os parênteses numa operação associativa, uma vez as operações podem ser efetuadas em qualquer ordem.

A \Cref{agb-def-propriedades} é uma definição inicial, que utiliza a quantidade mínima de elementos para indicar a propriedade da operação.
\begin{ex}
    A "comutatividade entre três elementos"\ pode ser feita assim: 
    % mas por exemplo, a 
    %\[ a * b * c = a * (b*c) = a*(c*b) = (c*b)*a = (b*c)*a = b*(c*a) = b*(a*c) =b*(c*a) =(c*a)*b = c * a * b. \] \\ 
    \begin{center}
        $a * b * c = a * (b*c) = a*(c*b) = (c*b)*a = (b*c)*a =$ \\
         $b*(c*a) = b*(a*c) = b*(c*a) =(c*a)*b = c * a * b.$
    \end{center}
\end{ex}



% \begin{align*}
%     a * b * c &= a * (b*c) \\ 
%     &= a*(c*b) \\
%     &= (c*b)*a \\
%     &= (b*c)*a \\
%     &= b*(c*a) \\ 
%     &= b*(a*c) \\ 
%     &= b*(c*a) \\
%     &= (c*a)*b \\
%     &= c * a * b.  
% \end{align*} 
A associatividade pode ser feita de maneira análoga.

\begin{teo}\label{agb-teo-neutroUnico}
    Seja $A$ um conjunto e $*$ uma operação em $A$. Se existir um elemento neutro para $*$, então ele é único.
\end{teo}
\begin{dem}
    Sejam $\mathfrak{e_1}$ e $\mathfrak{e_2}$ dois elementos neutros para $*$. Temos que 
    \[ \mathfrak{e_1} = \mathfrak{e_1} * \mathfrak{e_2} = \mathfrak{e_2}, \] 
    portanto ${\mathfrak{e_1} = \mathfrak{e_2}}$. A primeira igualdade é porque $\mathfrak{e_2}$ é neutro à direita, a segunda igualdade é porque $\mathfrak{e_1}$ é neutro à esquerda.
\end{dem}

Em particular, deve ser observada a comutatividade do elemento neutro com qualquer elemento, independentemente da operação ser ou não comutativa.

\begin{teo}\label{agb-teo-simetricoUnico}
    Sejam $A$ um conjunto e $*$ uma operação em $A$. Se $*$ é associativa e tem a propriedade da existência do elemento simétrico, então o simétrico de cada elemento é único.
\end{teo}
\begin{dem}
    Seja $a \in A$, e sejam $b, c$ dois elementos simétricos de $a$, assim temos que $b*a = \mathfrak{e} = a*c$. 
    Com isso, 
    \[ b = b * \mathfrak{e} = b * (a * c) = (b * a) * c = \mathfrak{e} * c = c. \]
\end{dem}

\begin{prop}\label{agb-prop-simetricoSimetrico}
    Seja $A$ um conjunto e $*$ uma operação sobre $A$ que seja associativa e admita simétrico. Seja $a$ um elemento de $A$. O simétrico do simétrico de $a$ é o próprio $a$. 
\end{prop}
\begin{dem}
    Seja $a \in A$. Considere $a'$ o simétrico de $a$ e $a''$ o simétrico de $a'$. Assim temos que $a * a' = \mathfrak{e}$, além disso $a' * a'' = \mathfrak{e}$.
    Com isso 
    \[ a = a * \mathfrak{e} = a * (a' * a'') = (a * a') * a'' = \mathfrak{e} * a'' = a''. \]
\end{dem}
\begin{defi}
    Sejam $A$ um conjunto, $*$ uma operação em $A$ e $b,c$ elementos de $A$. Dizemos que um elemento $a \in A$ cumpre a lei do cancelamento à esquerda se 
    \[ a*b=a*c \implies b=c. \]
\end{defi}
    Analogamente define-se que um elemento cumpre a lei do cancelamento à direita. Caso o elemento cumpra ambos os tipos de cancelamento à direita e à esquerda para uma mesma operação, dizemos apenas que ele cumpre a lei do cancelamento para aquela operação.
\begin{teo}\label{agb-teo-leiCancelamento}
    Seja $A$ um conjunto e $*$ uma operação em $A$. Se $*$ for associativa e admitir simétrico, então vale a lei do cancelamento para a operação $*$.
\end{teo}
\begin{dem}
    Sejam $a,b,c,a'$ elementos de $A$, em que $a'$ é simétrico de $a$ para a operação $*$. Temos que 
    \[a*b = a*c \implies a' * (a * b) = a' *(a*c) \implies (a' * a ) * b = (a'*a)*c \implies \mathfrak{e} * b = \mathfrak{e} * c \implies b = c. \] 
    Assim, $a$ é cancelável à esquerda, analogamente prova-se que $a$ é cancelável à direita.
\end{dem}

Até agora utilizamos uma única operação $*$ com o intuito de generalizar resultados que são válidos tanto para as adições quanto para as multiplicações com as quais trabalharemos. Agora passaremos a distinguir duas operações, que chamaremos de adição (ou soma) e multiplicação (ou produto), denotadas respectivamente por $+$ e $\cdot$.

Deixemos claro que o nome de soma e de multiplicação não determinam como serão definidas essas operações. É claro que as somas e produtos com os quais trabalharemos serão os usuais (vistos apenas de uma maneira mais formal).

\begin{obs}
    No \Cref{agb-teo-leiCancelamento}, quando estamos nos referindo à lei do cancelamento para uma soma, não há exceções, todas as somas (neste trabalho) tem a propriedade da lei do cancelamento válidas para qualquer elemento de $A$. Por outro lado, quando a operação é um produto, diremos que esse produto tem a propriedade da lei do cancelamento quando todo elemento diferente do elemento neutro da soma é cancelável no produto.
\end{obs}

\begin{obs}
	Às vezes, quando conveniente, omitiremos o símbolo da multiplicação por simplicidade. E para omitir parênteses estabeleceremos a seguinte convenção: 
	\begin{center}
    	$a + b \cdot c \defeq a + (b \cdot c)$;  \\
    	$a \cdot b + c \defeq (a \cdot b) + c$; \\
    	$a \cdot b \defeq ab$.    
	\end{center}
\end{obs}

\begin{defi}
    Sejam $A$ um conjunto e sejam $+$ e $\cdot$ as operações de adição e multiplicação sobre $A$. Dizemos que a multiplicação é distributiva em relação à soma quando, para quaisquer $a, b, c \in A$ é válido que \[ a \cdot (b + c) = a \cdot b + a\cdot c. \]
\end{defi}

\begin{defi}
    Uma relação de ordem parcial $\leq$ é dita compatível com a adição, quando para quaisquer $a,b,c \in A$, 
    valer \[ a \leq b \implies a + c \leq b + c. \]
\end{defi}

\begin{defi}\label{agb-def-numPositivo}
    Dizemos que um número $a$ é positivo quando ocorre ao menos uma situação dos itens abaixo:
    \begin{itemize}
        \item $a \in \mathbb{N}$;
        \item $a$ é um elemento de algum conjunto de $\mathbb{Z}$, $\mathbb{Q}$, $\mathbb{R}$ e é maior do que o elemento neutro aditivo desse conjunto.
    \end{itemize}
\end{defi}
\begin{defi}\label{agb-def-numNegativo}
    Dizemos que um número $a$ é negativo, quando $a$ é um elemento de algum conjunto de $\mathbb{Z}$, $\mathbb{Q}$, $\mathbb{R}$ e é menor do que o elemento neutro aditivo desse conjunto.
\end{defi}
A \Cref{agb-def-numPositivo} acima nos permite a definição a seguir.
\begin{defi}\label{agb-def-ordemCompativelProduto}
    Seja $A$ um conjunto parcialmente ordenado pela relação $\leq$. Sejam \\ $a,b,c \in A$, em que $c$ é um número positivo. Seja também $\cdot$ uma multiplicação sobre $A$.
    Dizemos que $\leq$ é compatível com a multiplicação quando 
    \[ a \leq b \implies ac \leq bc. \]
\end{defi}

% \begin{defi}\label{agb-def-limiteSuperior}
%     Sejam $A$ um conjunto parcialmente ordenado pela relação $\leq$, e $\emptyset \neq X \subset A$. Um elemento $L \in A$ é cota superior de $X$ quando para qualquer $x \in  X$ temos $x \leq L$.
% \end{defi}
% \begin{defi}
%     Sejam $A$ um conjunto parcialmente ordenado pela relação $\leq$, e $\emptyset \neq X \subset A$. Um elemento $l \in A$ é cota inferior de $X$ quando para qualquer $x \in  X$ temos $l \leq x$.
% \end{defi}

% Existem conjuntos que não são limitados nem superiormente, nem inferiormente. Vamos explorar isso \todo{Provar que N , Z e Q nao sao limitados}.

% \begin{defi}\label{agb-def-maximo}
%     Seja $A$ um conjunto parcialmente ordenado pela relação $\leq$. Dizemos que um elemento $M$ é máximo de $A$ caso ele pertença a $A$ e também seja uma cota superior de $A$.
% \end{defi}

% \begin{defi}\label{agb-def-minimo}
%     Seja $A$ um conjunto parcialmente ordenado pela relação $\leq$. Dizemos que um elemento $m$ é máximo de $A$ caso ele pertença a $A$ e também seja uma cota inferior de $A$.
% \end{defi}



\begin{defi}\label{agb-def-anel}
    Seja $A$ um conjunto, e sejam $+, \cdot$ duas operações sobre $A$, chamadas de adição e multiplicação. Dizemos que $(A, +, \cdot)$ é um anel se são válidas as propriedades:
    \begin{enumerate}[label=(\roman*)]
        \item Associativa da adição;
        \item Comutativa da adição;
        \item Da existência do elemento neutro para adição;
        \item Da existência do elemento simétrico para a adição;
        \item Associativa da multiplicação;
        \item Comutativa da multiplicação;
        \item Da existência do elemento neutro da multiplicação;
        \item Da distributividade da multiplicação em relação à adição;
    \end{enumerate}
    Quando não houver ambiguidade chama-se $(A,+,\cdot)$ apenas de anel $A$.
\end{defi}



\begin{teo}\label{agb-teo-anulamentoProduto1}
    Seja $A$ um anel. Então para qualquer $a \in A$ vale que $a \cdot 0 = 0 = 0 \cdot a$.
\end{teo}
\begin{dem}
 Temos que as seguintes igualdades são válidas: 
    \[a \cdot 0+0 = a \cdot 0 = a \cdot (0+0) = a \cdot 0+a \cdot 0.\] 
    Assim $a \cdot 0+0 = a \cdot 0+a \cdot 0$ e pela lei do cancelamento, $0=a \cdot 0$. Ainda, como o produto é comutativo, vale $a \cdot 0=0 \cdot a$.
\end{dem}

\begin{teo}\label{agb-teo-anulamentoProduto2}
    Seja $A$ um anel. Então se $a,b \in A$, vale que $ab = 0$ se, e somente se, $a = 0$ ou $b = 0$.
\end{teo}
\begin{dem}
    Se $a = 0$ ou se $b=0$ então $ab = 0$ pelo \Cref{agb-teo-anulamentoProduto1}.
    Se $ab=0$, e suponhamos sem perda de generalidade, que $b \neq 0$, então temos $ab = 0 = 0b$, como $b \neq 0$ podemos aplicar o cancelamento para o produto e ficamos com $a=0$.
\end{dem}

\begin{defi}\label{agb-def-corpo}
    Um corpo é um anel $A$ em que cada elemento diferente do neutro aditivo admite um simétrico multiplicativo.
\end{defi}

% As manipulações numéricas que fazemos no ensino básico não levam em conta o fato de $\mathbb{Q}$ ser diferente de $\mathbb{R}$. Essa semelhança vem do fato de que ambos $\mathbb{Q}$ e $\mathbb{R}$ são corpos ordenados, como mostraremos no \Cref{cap-reais}, dos números reais. 

\end{document}
