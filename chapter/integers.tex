\documentclass[../main.tex]{subfiles}
\begin{document}
\chapter{O CONJUNTO DOS NÚMEROS INTEIROS}
Neste capítulos faremos a construção dos números inteiros. A bibliografia principal continua sendo \textcite{ferreira} e \textcite{domingues-2009}. Será apresentado uma relação de equivalência que servirá para criar $\mathbb{Z}$. Depois definiremos uma adição e um produto que tenha algumas propriedades que já tinham em $\mathbb{N}$, isto é, queremos "estender"  $\mathbb{N}$.

Devemos observar que em $\mathbb{N}$ temos que a subtração, tal como conhecemos no ensino básico, não é uma operação (no sentido da álgebra), pois a subtração não está definida para quaisquer dois elementos de $\mathbb{N}$ tal que o resultado esteja em $\mathbb{N}$.

Com o objetivo de contornar esse problema (e portanto poder recriar a subtração como no ensino básico, em que ela seja uma operação sobre um conjunto), iremos criar um novo conjunto a partir de $\mathbb{N}$. Veremos que isso é possível, tal conjunto será denotado por $\mathbb{Z}$ e chamaremos ele de conjunto dos \emph{números inteiros}.

Nos números naturais, poderíamos ter definido uma função que tivesse o intuito de fazer o inverso da soma, que seria a subtração denotada por "$-$". Seguindo esse raciocínio poderíamos mostrar que $9-3 = 8-2 = 7-1$. Concluiríamos com base nessas igualdades, que $9 + 2 = 8 + 3$ e $8 + 1 = 7 + 2$, o que em ambos os casos, os resultados são números naturais. 

Além disso, queremos expressar um número inteiro apenas com números naturais, sem ter que assumir mais que $\mathbb{N}$, a lógica e a teoria de conjuntos que já foram assumidas. Além disso, tentar dar significado à expressões do tipo $3-5, 4-8, 2-3$ que nesses exemplos não são números naturais.

A maneira como expressaremos será através de uma relação binária $\sim$ sobre $\mathbb{N} \times \mathbb{N}$ definida desse modo: $(a,b) \sim (c,d) \iff a+d = b+c$, sendo $a,b,c,d$ números naturais quaisquer.
\begin{teo}
    A relação $\sim$ é de equivalência.
\end{teo}
\begin{dem}
    \begin{enumerate}[label=(\roman*)]
        \item Reflexiva: $(a,b) \sim (a,b) \iff a+b=b+a$.
        \item Simetria: $(a,b) \sim (c,d) \iff (c,d) \sim (a,b)$ \\
        Temos $ a+d = b+c = c+b = d+a \iff (c,d) \sim (a,b)$.
        \item Transitiva: $(a,b) \sim (c,d) \land (c,d) \sim (e,f) \implies (a,b) \sim (e,f)$. \\
        Temos $(a,b) \sim (c,d) \iff a+d=b+c$ \\
        e também que $(c,d) \sim (e,f) \iff c+f=d+e$. \\
        Somando os dois primeiros termos entre si, e os dois segundos termos entre si, e igualando os resultados, temos: \\ 
        $(a+d)+(c+f)= (b+c)+(d+e) \iff a+f = b+e \iff (a,b) \sim (e,f)$.
    \end{enumerate}
\end{dem}
\begin{defi}\label{int-def-conjNum}
    O conjunto quociente $\mathbb{N} \times \mathbb{N} / \sim = \{ \overline{(a,b)}: (a,b) \in \mathbb{N} \times \mathbb{N}\}$ será chamado de $\emph{conjunto dos números inteiros}$ e será denotado, conforme já anunciado, por $\mathbb{Z}$.
\end{defi}
A \Cref{agb-def-conjQuoc} define conjunto quociente. Um exemplo de elemento de $\mathbb{Z}$ é $\overline{(1,5)} = \overline{(2,6)} = \overline{(3,7)}$, em que cada representante da classe, a saber, $(1,5), (2,6) e (3,7)$ representa o mesmo número inteiro (a mesma classe de equivalência). Embora essa não seja uma notação comum de ser utilizada fora da criação dos conjuntos numéricos (porque o objetivo de usar pares ordenados, classes de equivalência, etc. não é facilitar o manuseio de $\mathbb{Z}$, mas sim formalizar esse conjunto, com suas operações e relações como conhecemos). Enquanto escrevemos normalmente o número inteiro $(5,1)$ por $4$, escrevemos o número $(1,5)$ por $-4$.

\begin{obs}
    Utilizaremos a mesma notação para adição e multiplicação de números inteiros, que utilizamos nos números naturais. Inicialmente para o desenvolvimento do texto, sempre que $a,b \in \mathbb{N}$ entenderemos a soma $a+b$ como sendo executada em $\mathbb{N}$. Caso $a,b \in \mathbb{Z}$ entenderemos a soma $a+b$ sendo executada em $\mathbb{Z}$. Posteriormente justificaremos a escolha de utilizar os mesmos símbolos sendo que os conjuntos são diferentes. Isso se repetirá com a multiplicação e coma a relação de ordem. Quando $a \in N$ e $b \in Z$, entendemos a soma $a + b$ como $f(a) + b$ em que $f(a) \in \mathbb{Z}$, sendo $f$ a função do \Cref{int-teo-imersao}.
\end{obs}

Com essa definição de número inteiro podemos interpretar, a nível de ensino básico, que o número $\overline{(a,b)}$ é $a - b$. Embora caso fôssemos definir subtração em $\mathbb{N}$, ela não seria uma operação, pois estaria definida apenas quando $a > b$.

\begin{prop}\label{int-prop-cancelaCoordenadas}
    Se $a,b,c \in \mathbb{N}$ vale que: $\overline{(a+c,b+c)} = \overline{(a,b)}$.
\end{prop}
\begin{dem}
    A demonstração é imediata, vemos que os elementos (os pares ordenados) representam a mesmo conjunto pois $a+c+b = b+c+a$.
\end{dem}

\section{A adição em $\mathbb{Z}$}
Tínhamos a adição em $\mathbb{N}$. Vamos definir uma adição em $\mathbb{Z}$, mantendo algumas propriedades da adição em $\mathbb{N}$.
\begin{defi}\label{int-def-adicao}
    Dados $\overline{(a,b)}$ e $\overline{(c,d)}$ em $\mathbb{Z}$, definimos a adição $\overline{(a,b)} + \overline{(c,d)}$ pelo número $\overline{(a+c, b+d)}$.
\end{defi}
\begin{prop}
    A operação de adição está bem definida em $\mathbb{Z}$. Isto é, a adição em $\mathbb{Z}$ não depende do representante das classes envolvidas na adição. Podemos expressar isso assim também: $a,a',b,b',c,c',d,d' \in \mathbb{N}$, $\overline{(a,b)} = \overline{(a',b')} \land \overline{(c,d)} = \overline{(c',d')}
    \implies \overline{(a,b)} + \overline{(c,d)} = \overline{(a',b')} + \overline{(c',d')} $.
\end{prop}
\begin{dem}
    Vamos desenvolver as duas somas e mostrar que são iguais. Consideremos 
    $\overline{(a,b)} + \overline{(c,d)} = \overline{(a+c,b+d)}$.
    Considerando agora a segunda soma temos: 
    $\overline{(a',b')} + \overline{(c',d')} = \overline{(a'+c',b'+d')}$.
    Precisamos agora apenas mostrar que $(a+c,b+d) \sim (a'+c',b'+d')$, isso mostrará que 
    $\overline{(a+c,b+d)} = \overline{(a'+c',b'+d')}$. 
    Como $\overline{(a,b)} = \overline{(a',b')} \iff a+b' = b+a'$ e ainda
     $\overline{(c,d)} = \overline{(c',d')} \iff c+d' = d+c'$ teremos então que 
     \begin{center}
         $(a+b')+(c+d') = (b+a')+(d+c') \iff $ \\
         $(a+c) + (b'+d') = (b+d) + (a'+c') \iff $ \\
         $(a+c,b+d) \sim (a'+c',b'+d')$.
     \end{center}
\end{dem}

\begin{prop}{Sejam $\overline{(a,b)}, \overline{(c,d)}, \overline{(e,f)}$ números inteiros quaisquer, para a adição valem as seguintes propriedades:}
    \begin{enumerate}[label=(\roman*)]
        \item fechamento;
        \item associativa;
        \item comutativa;
        \item do elemento neutro; 
        \item do elemento simétrico;
        \item lei do cancelamento.
    \end{enumerate}
\end{prop}

\begin{dem}
    Explicitamos que entendemos $a,b,c,d,e,f$ como números naturais, o que é devido à \Cref{int-def-conjNum}.
    \begin{enumerate}[label=(\roman*)]
        \item Fechamento: $\overline{(a,b)} + \overline{(c,d)} \in \mathbb{Z}$:
        Essa é imediata pois $a+c \in \mathbb{N} \land b+d \in \mathbb{N} \implies \overline{(a+c,b+d)} \in \mathbb{Z}$.

        \item Associativa: 
        $\left(\overline{(a,b)} + \overline{(c,d)}\right) +  \overline{(e,f)} =
        \overline{(a+c,b+d)}+\overline{(e,f)} = \overline{(a+c+e, b+d+f)} = \overline{ \left( a+(c+e), b+(d+f) \right) } = \overline{(a,b)} + 
        \overline{ (c+e, d+f) } = \overline{(a,b)} + \left( \overline{(c,d)} + \overline{(e,f)} \right)$.
       
        \item Comutativa: $\overline{(a,b)} + \overline{(c,d)} = \overline{(a+c,b+d)} = \overline{(c+a,d+b)} = \overline{(c,d)} + \overline{(a,b)}$;
       
        \item Elemento neutro: $\exists! x \in \mathbb{Z} : x + \overline{(a,b)} = \overline{(a,b)} + x = \overline{(a,b)}$; \\
        Consideremos $\overline{(1,1)}$, vamos mostrar que ele o $x$ acima. 
        Provaremos que é neutro pela esquerda. Temos que $\overline{(1,1)} + \overline{(a,b)} = \overline{(1+a,1+b)}$. Provemos que $\overline{(1+a,1+b)} \sim \overline{(a,b)}$.
        Temos então que $(1+a)+b = (1+b)+a$, o que sempre ocorre, portanto $\overline{(1,1)}$ é neutro pela esquerda. Analogamente prova-se que é pela direita, e além disso, todo elemento neutro para uma operação é único, conforme o \Cref{agb-neutro-unico}, o que conclui nossa prova. Denotaremos o elemento neutro da adição de $0$ e chamaremos de zero.
       
        \item Simétrico: $(\forall x \in \mathbb{Z}) (\exists!y \in \mathbb{Z}) : x + y = y + x  = 0$; \\
        A comutatividade nós já temos, basta provar que $x + y = 0$.
        Consideremos então $x = \overline{(a,b)}$, mostraremos que $y = \overline{(b,a)}$ é tal que $\overline{(a,b)} + \overline{(b,a)} = \overline{(1,1)}$. Temos que $\overline{(a+b,b+a)} = \overline{(1,1)}$ porque $a+b+1 = b+a+1$, o que sempre ocorre. Pelo \Cref{agb-teo-simetricoUnico} concluímos também que esse simétrico é único.
       
        \item Lei do cancelamento: $\overline{(a,b)} + \overline{(c,d)} = \overline{(a,b)} + \overline{(e,f)} \implies \overline{(c,d)} = \overline{(e,f)}$.
        Temos $\overline{(a,b)} + \overline{(c,d)} = \overline{(a,b)} + \overline{(e,f)} \iff \overline{(a+c,b+d)} = \overline{(a+e,b+f)} \iff (a+c)+(b+f)=(b+d)+(a+e) \iff c+f = d+e \iff \overline{(c,d)} = \overline{(e,f)}$. A comutatividade garante a lei do cancelamento à esquerda e à direita.


    \end{enumerate}
\end{dem}



Com essas propriedades, podemos ver que a comutatividade, associatividade e fechamento são mantidos, com a diferença que agora trabalhamos com números inteiros e não naturais. Indo além, é possível ver que o elemento neutro da soma e o simétrico surgem naturalmente, embora não possamos dizer que o neutro em $\mathbb{Z}$ é $(0,0)$, pois nenhuma das coordenadas desse par ordenado são números naturais, não obstante qualquer $a \in \mathbb{N}$ temos que $(a,a) = 0 \in \mathbb{Z}$. Também temos agora um simétrico para a soma, o que em $\mathbb{N}$ não tem análogo. Por último, observando a \Cref{agb-def-operacao} concluímos que a adição é uma função.

Com relação ao elemento simétrico da soma, como ele sempre existe e é único, denotamos o simétrico de $a \in \mathbb{Z}$ por $-a$. Agora podemos definir a subtração em $\mathbb{Z}$ de uma maneira relativamente simples.

\begin{defi}\label{int-def-subtracao}
    Sejam $a,b \in \mathbb{Z}$. A subtração de $a$ por $b$, denotada $a - b$ é definida como $a + (-b)$, onde $-b$ é o simétrico de $b$.
\end{defi}
\begin{prop} Sejam $a,b \in \mathbb{Z}$, são válidas as seguintes propriedades:
    \begin{enumerate}[label=(\roman*)]
        \item $-a + b = b - a$;
        \item $a - ( - b) = a+b$;
        \item $-a-b = -(a+b)$.
    \end{enumerate}
\end{prop}
\begin{dem}
        \begin{enumerate}[label=(\roman*)]
        \item $-a + b = b - a$, pois $(-a) + b = b + (-a) = b-a$;
        \item $a - ( - b) = a+b$, pois $-(-b) = b$, pela \Cref{agb-prop-simetricoSimetrico};
        \item $-a-b = -(a+b)$, pois como a adição é função, podemos somar $(a+b)$ em ambos os lados da igualdade, temos
        $(a+b)+(-a)+(-b) = -(a+b) + (a+b) = 0 \iff a+(-a) + b+(-b) = 0$, o que sempre ocorre.
    \end{enumerate}
\end{dem}

\section{A multiplicação em $\mathbb{Z}$}
Agora vamos definir uma multiplicação em $\mathbb{Z}$, podemos entender a multiplicação em $\mathbb{Z}$ como tentativa de estensão de $\mathbb{N}$, assim várias propriedades queremos que sejam mantidas. Veremos que isso ocorrerá.
\begin{defi}\label{int-def-multiplicacao}
    Dados $\overline{(a,b)}$ e $\overline{(c,d)}$ em $\mathbb{Z}$, definimos a multiplicação $\overline{(a,b)} \cdot \overline{(c,d)}$ pelo número $\overline{(a \cdot c + b \cdot d, a \cdot d + b \cdot c)}$.
\end{defi}
Podemos observar a título de intuição que, se as primeiras coordenadas representam a parte "positiva" do número e as segundas coordenadas a parte negativa, a multiplicação de dois positivos é positiva, e a multiplicação de dois negativos é positiva, então esses resultados ficam na primeira coordenada. Semelhantemente, quando um é positivo e outro negativo, o resultado fica negativo, e assim fica na segunda coordenada.
\begin{teo}
    A operação de multiplicação está bem definida em $\mathbb{Z}$. Isto é, a multiplicação em $\mathbb{Z}$ não depende do representante das classes envolvidas na multiplicação. Equivale a dizer que, se \\
    $a,a',b,b',c,c',d,d' \in \mathbb{N} \land \overline{(a,b)} = \overline{(a',b')} \land \overline{(c,d)} = \overline{(c',d')}
    \implies \overline{(a,b)} \cdot \overline{(c,d)} = \overline{(a',b')} \cdot \overline{(c',d')} $.
\end{teo}
\begin{dem}
    Comecemos pela definição de multiplicação, $\overline{(a,b)} = \overline{(a',b')}$ e $\overline{(c,d)} = \overline{(c',d')}$ teremos os produtos:
    $\overline{(a,b)} \cdot \overline{(c,d)} = \overline{(ac+bd,ad+bc)}$ e $\overline{(a',b')} \cdot \overline{(c',d')} = \overline{(a'c'+b'd',a'd'+b'c')}$.
    
    Vamos considerar agora as igualdades das classes, temos: \\
        $\overline{(a,b)} = \overline{(a',b')} \iff a + b' = b + a' $ disso concluímos que 
        $ (a + b')c' = (b + a')c'$ e que $(a + b')d' = (b + a')d'$. \\
    Considerando $\overline{(c,d)} = \overline{(c',d')} \iff c + d' = d + c' $, temos que
    $a(c + d') = a(d + c')$ e também que $b(c + d') = b(d + c')$.
    
    Das igualdades que obtemos no parágrafo anterior, colocaremos na ordem de aparecimento as duas últimas de cada caso, embora apliquemos as distributivas e talvez troquemos o primeiro com o segundo termo da igualdade.
    \begin{align*}
        ac'+b'c' &= bc'+a'c' \\
        bd'+a'd' &= ad'+b'd' \\
        ac+ad'   &= ad+ac' \\
        bd+bc'   &= bc+bd'.
    \end{align*}
    Vamos agora somar a colona à esquerda, depois a coluna à direita, e colocar na mesma equação, assim:
    $$ac'+b'c' + bd'+a'd' + ac+ad' + bd+bc' = bc'+a'c' + ad'+b'd' + ad+ac' + bc+bd'$$
    Cancelando os termos $ac', bd', ad', bc'$ ficamos com:
    \begin{align*}
        b'c' +a'd' + ac + bd  &= a'c' + b'd' + ad+ bc \\
        ac + bd + a'd' + b'c' &= ad + bc + a'c' + b'd'.
    \end{align*}
    Com isso provamos que $ (ac+bd,ad+bc) \sim (a'c'+b'd',a'd'+b'c')$, o que nos mostra que a classe é a mesma, então o produto não depende do representante da classe, que pode, portanto, ser arbitrário.
\end{dem}
\begin{prop}{Sejam $\overline{(a,b)}, \overline{(c,d)}, \overline{(e,f)}$ números inteiros quaisquer, para a multiplicação valem as seguintes propriedades:}
    \begin{enumerate}[label=(\roman*)]
        \item fechamento;
        \item associativa;
        \item comutativa;
        \item do elemento neutro; 
        \item do elemento simétrico;
        \item lei do cancelamento \footnote{A lei do cancelamento do produto é para todo elemento diferente do neutro da soma, ou seja, diferente de $0$}.
    \end{enumerate}
\end{prop}
\begin{dem}
    Entendemos $a,b,c,d,e,f$ como números naturais, como é colocado na definição.
    \begin{enumerate}[label=(\roman*)]
        \item Fechamento:
        Temos $\overline{(a,b)} \cdot \overline{(c,d)} = \overline{(ac+bd, ad+bc)}$. Como $ac+bd \in \mathbb{N}$ e também $ad+bc \in \mathbb{N}$ temos $\overline{(ac+bd, ad+bc)} \in \mathbb{Z}$.

        \item Associativa: 
        \begin{align*}
            \big(\overline{(a,b)} \cdot \overline{(c,d)}\big) \cdot  \overline{(e,f)} &= \overline{(ac+bd, ad+bc)} \cdot \overline{(e,f)} \\
            &= \overline{((ac+bd)e + (ad+bc)f , (ac+bd)f + (ad+bc)e} \\
            &= \overline{(ace+bde+adf+bcf, acf+bdf+ade+bce)} \\
            &= \overline{(ace+adf+bcf+bde, acf+ade+bce+bdf)} \\
            &= \overline{(a(ce+df) + b(cf+de) , a(cf+de) + b(ce+df))} \\
            &= \overline{(a,b)} \cdot \overline{(ce+df, cf+de)} \\
            &= \overline{(a,b)} \cdot \big( \overline{(c,d)} \cdot \overline{(e,f)} \big)
        \end{align*}
        
        
        \item Comutativa: 
        $\overline{(a,b)} \cdot \overline{(c,d)} = \overline{(ac+bd, ad+bc)} =
        \overline{(ca+db, bc+da)} = \overline{(c,d)} \cdot \overline{(a,b)}$
            
        
        \item Elemento neutro: Consideremos o número $\overline{(2,1)}$. Vamos mostrar que ele é neutro pela direita. Temos $\overline{(a,b)} \cdot \overline{(2,1)} = \overline{(2a+b, a+2b)} = \overline{(a,b)}$, essa última igualdade é porque $a + a + 2b = b + 2a + b$. Analogamente se mostra que é neutro pela esquerda. A unicidade é garantido pelo teorema \Cref{agb-neutro-unico}.
        
        \item Distributiva: primeiro vamos mostrar a distributiva à esquerda.
        \begin{align*}
            \overline{(a,b)} \cdot \left( \overline{(c,d)} + \overline{(e,f)} \right) &= \overline{(a,b)} \cdot (\overline{(c+e,d+f)}) \\
            &= \overline{(a(c+e) + b(d+f), a(d+f) +b(c+e)} \\
            &= \overline{(ac+ae+bd+bf, ad+af+bc+be)} \\
            &= \overline{(ac+bd, ad+bc)} + \overline{(ae+bf, af+be)} \\
            &= \big( \overline{(a,b)} \cdot \overline{(c,d)} \big) + \big( \overline{(a,b)} \cdot \overline{(e,f)} \big)    
        \end{align*}
        Pela comutatividade do produto em $\mathbb{Z}$ e pela distributiva à esquerda, a distributiva a direita está também provada.

        \item Lei do cancelamento: Por hipótese temos 
        \begin{center}
            $\overline{(a,b)} \cdot \overline{(c,d)} = \overline{(a,b)} \cdot \overline{(e,f)}$ \\
            $\overline{(ac+bd, ad+bc)} = \overline{(ae+bf, af+be)}$
        \end{center}
        O que equivale a: 
        \begin{center}
            $ac+bd+af+be = ad+bc+ae+bf$ \\
             $a(c+f) + b(d+e) = a(d+e) + b(c+f)$
        \end{center}
        Como $\overline{(a,b)} \neq 0$, $a \neq b$. Suponhamos sem perda de generalidade que $a > b$, temos que $a = b + m$ para algum $m$ natural. Substituindo na equação ficamos com 
        \begin{center}
            $(b+m)(c+f)+ b(d+e) = (b+m)(d+e)+b(c+f)$ \\
            $bc + bf + mc + mf + bd + be = bd + be + md + me + bc + bf$
        \end{center}
        Cancelando os termos $bc, bf, bd, be$ ficamos com
        \[\begin{split}
             mc + mf = md + me \iff \\
            m(c+f) = m(d+e) \iff \\
            c+f = d+e \iff \\
            (c,d) \sim (e,f)           
        \end{split} \]
       $$\therefore \overline{(c,d)} = \overline{(e,f)}$$.

        % \item Lei do anulamento do produto:
        %     $\overline{(a,b)} \cdot \overline{(c,d)} = 0 \implies \overline{(a,b)} = 0 \lor \overline{(c,d)}=0$; \\
        %     \yellow{TODO}
        %     \gray{asd}
    \end{enumerate}
\end{dem}

\begin{prop}\label{int-prop-sinalRegra1}
    Sejam $a, b$ números inteiros quaisquer. É válido que $-(ab) = a(-b) = (-a)b$.
\end{prop}
\begin{dem}
    A demonstração consiste em explorar a unicidade do simétrico, conforme a \Cref{agb-teo-simetricoUnico}.
    Vamos mostrar que $ab$ é o simétrico de todos no enunciado. Basta observar que $ab + (-ab) = 0$.
    Também vale que $ab + a(-b) = a(b+(-b)) = a0 = 0$. Por último, $ab + (-a)b = (a + (-a))b = 0b = 0$.
\end{dem}
\begin{corol}\label{int-corol-sinalRegra2}
    Se $a,b \in \mathbb{Z}$, então vale $(-a)(-b) = ab$
\end{corol}
\begin{dem}
    Observando a \Cref{int-prop-sinalRegra1} concluímos que vale $(-a)(-b) = -(a(-b)$, como o produto é comutativo, temos $-((-b)a) = -(-(ba))$,
    o que, pelo \Cref{agb-prop-simetricoSimetrico} temos que $-(-(ba)) = ba = ab$.
\end{dem}

Com isso, podemos concluir que $\mathbb{Z}$ com essa soma e esse produto, é um anel, conforme definido na \Cref{agb-def-anel}.

\section{Relação de ordem}
\begin{defi}
    Dados $\overline{(a,b)}$ e $\overline{(c,d)}$ em $\mathbb{Z}$, definimos a relação de ordem $\leq$ e dizemos que $\overline{(a,b)}$ é menor do que ou igual a $\overline{(c,d)}$ quando $a+d \leq b+c$ e denotamos por $\overline{(a,b)} \leq \overline{(c,d)}$.
\end{defi}

\begin{prop}
    A relação de ordem está bem definida, isto é, sejam $\overline{(a,b)} = \overline{(a',b')}$ e $\overline{(c,d)} = \overline{(c',d')}$. Se $\overline{(a,b)} \leq \overline{(c,d)}$, então $\overline{(a',b')} \leq \overline{(c',d')}$.
\end{prop}
\begin{dem}
    Temos $\overline{(a,b)} = \overline{(a',b')} \iff a+b' = b+a'$. 
    Também $\overline{(c,d)} = \overline{(c',d')} \iff c+d' = d+c'$. 
    Como $\overline{(a,b)} \leq \overline{(c,d)} \iff a+d \leq b+c$, temos:
    \begin{align*}
        b+a'+c+d' &= a+b' + d+c' \\
        a'+d' + a+d + m &= b' + c' + a + d \\
        a'+d' + m &= b' + c'  \\
        a'+d' & \leq b'+c'.
    \end{align*}
     
\end{dem}
\begin{prop}{Sejam $\overline{(a,b)}, \overline{(c,d)}, \overline{(e,f)}$ números inteiros quaisquer, para a relação de ordem valem as seguintes propriedades:}\label{int-prop-relacaoOrdem}
    \begin{enumerate}[label=(\roman*)]
        \item Reflexiva;
        \item Antissimétrica;
        \item Transitiva;
        \item Totalidade;
        \item Compatível com a adição;
        \item Compatível com a multiplicação.
    \end{enumerate}
\end{prop}
\begin{dem}
    \begin{enumerate}[label=(\roman*)]
        \item Reflexiva: \\
            De fato ocorre que $a+b \leq b+a \therefore \overline{(a,b)} \leq \overline{(a,b)}$.
        
        \item Antissimétrica: 
            \begin{center}
                $\overline{(a,b)} \leq \overline{(c,d)} \myspace\land \overline{(c,d)} \leq \overline{(a,b)}$ \\
                $a+d \leq b+c \myspace\land c+b \leq d+a$
            \end{center}
            O que, pela antissimetria em $\mathbb{N}$ temos que $a+d = c+b$, então $(a,b) \sim (c,d)$ e $\overline{(a,b)} = \overline{(c,d)} $.
        
        \item Transitiva: 
        \begin{align*}
            \overline{(a,b)} \leq \overline{(c,d)} & \myspace\land \overline{(c,d)} \leq \overline{(e,f)}  \\
            a+d \leq b+c & \myspace\land c+f \leq d+e  \\
            a+d+f \leq b + c + f & \myspace\land c + f + b \leq d + e + b  \\
            a + d + f & \myspace\leq d + e + b \\
            a + f & \myspace\leq b + e 
            \therefore \overline{(a,b)}  \myspace\leq \overline{(e,f)}.
        \end{align*}
            
        \item Totalidade: 
        \begin{align*}
            \overline{(a,b)} \leq \overline{(c,d)} & \myspace\lor \overline{(c,d)} \leq \overline{(a,b)} \\ 
            a+d \leq b+c & \myspace\lor c+b \leq d+a   
        \end{align*}
        Pela totalidade em $\mathbb{N}$ temos que sempre pelo menos uma das proposições disjuntas pelo $\lor$ ocorre.
        
        \item Compatível com a adição: \\
        \begin{center}
            $\overline{(a,b)} \myspace\leq \overline{(c,d)}$ \\
            $a+d \myspace\leq b+c$ 
        \end{center}
            Assim, temos $b+c = a+d + m$, para algum $m$ natural. Se $e, f$ são também naturais, temos: \\
        \begin{center}    
           $a+d+e+f \myspace\leq a+d+m+e+f$ \\

            $a + d +e+f \myspace\leq b + c + e+f$ \\
            $a + e + d +f \myspace\leq b + f + c + e$ \\
            $\overline{(a+e,b+f)} \myspace\leq \overline{(c+e, d+f)}$ \\
            $\overline{(a,b)} + \overline{(e,f)} \myspace\leq \overline{(c,d)} + \overline{(e,f)}$

        \end{center}

        
        \item Compatível com a multiplicação: \\
        Suponhamos $\overline{(a,b)} \leq \overline{(c,d)}$ e também $0 = \overline{(1,1)} < \overline{(e,f)}$.
        Desse modo, temos que $e > f$ pois $1+f < 1+e$. Seja então $e = f+m$ para algum $m$ natural. As linhas abaixo são equivalentes:
        \begin{center}
            $a+d \myspace\leq b+c  $\\
            $(a+d)m \myspace\leq (b+c)m $ \\
            $(a+d)m + (a+d)f + (b+c)f \myspace\leq (b+c)m + (a+d)f + (b+c)f$ \\
            $(a+d)(f+m) + (b+c)f \myspace\leq (a+d)f + (b+c)(f+m)$ \\
            $(a+d)e + (b+c)f \myspace\leq (a+d)f + (b+c)e$ \\
            $ae+bf + cf+de \myspace\leq af+be + ce+df  $\\
            $\overline{(ae+bf, af+be)} \myspace\leq \overline{(ce+df, cf+de)}  $ \\           
            $\overline{(a,b)} \cdot \overline{(e,f)} \myspace\leq \overline{(c,d)} \cdot \overline{(e,f)} $ 
        \end{center}        
        
    \end{enumerate}
\end{dem}
\begin{obs}
    Na compatibilidade com a multiplicação que acabamos de provar, ela também prova que se $\overline{(a,b)} \cdot \overline{(e,f)} 
    \leq \overline{(c,d)} \cdot \overline{(e,f)}$, então $\overline{(a,b)} \leq \overline{(c,d)}$. Comentário análogo vale para essa compatibilidade com a adição.
\end{obs}

\begin{prop}
    É válido que $\overline{(a,b)} < \overline{(c,d)}$ se, e somente se, existe um $\overline{(e,f)} > 0$ onde $\overline{(a,b)} + \overline{(e,f)} = \overline{(c,d)}$.
\end{prop}
\begin{dem}
    Como $\overline{(a,b)} < \overline{(c,d)}$, temos $a+d + m = b+c$, para algum $m$ natural. Consideremos o número $\overline{(m+1, 1)}$, vamos mostrar que ele é o número procurado. Sabemos que ele é positivo pois $\overline{(1,1)} < \overline{(m+1, 1)} \iff 1+1 < 1+m+1 $. \\
    Agora, $\overline{(a,b)} + \overline{(m+1,1)} = \overline{(a+m+1,b+1)} = \overline{(a+m, b)}$.
    Podemos observar que $\overline{(a+m, b)} = \overline{(c,d)}$, pois $a+m+d = b+c$.

    Provemos a volta. Suponhamos que $\overline{(a,b)} + \overline{(e,f)} = \overline{(c,d)}$, para algum $\overline{(e,f)} > 0$. Temos $e=f+m$ para algum $m$ natural. Assim, ficamos com $\overline{(a+f+m, b+f)} = \overline{(a+m, b)} = \overline{(c,d)}$. Assim, $a+m+d = b+c$, logo, $a+d < b+c 
    \therefore \overline{(a,b)} < \overline{(c,d)}$.
\end{dem}

\begin{prop}
    Complementando a proposição anterior, vale que $\overline{(a,b)} \leq \overline{(c,d)}$ se, e somente se, existe um $\overline{(e,f)} \geq 0$ onde $\overline{(a,b)} + \overline{(e,f)} = \overline{(c,d)}$.
\end{prop}
\begin{dem}
    Vamos separar em 4 casos: \\
    \begin{enumerate}
        \item Se $\overline{(a,b)} = \overline{(c,d)}$ então $\overline{(a,b)} + 0 = \overline{(c,d)}$.
        \item Se $\overline{(a,b)} < \overline{(c,d)}$, caímos na proposição anterior.
        \item Se $\overline{(e,f)} = 0$, temos $\overline{(a,b)} + 0 = \overline{(c,d)}$.
        \item Se $\overline{(e,f)} > 0$, caímos na proposição anterior.
    \end{enumerate}
   
\end{dem}

\begin{prop}\label{int-prop-coordenadaMaior}
    Um número inteiro é não negativo se, e somente se a primeira coordenada é maior do que ou igual a segunda coordenada. 
    Isto é, se $\overline{(a,b)}$ é um número inteiro, então vale $0 \leq \overline{(a,b)}  \iff b \leq a$
\end{prop}
\begin{dem}
    Seja $0 = \overline{(1,1)}$.
    As linhas a seguir são equivalentes:
    \[ 0 = \overline{(1,1)} \leq \overline{(a,b)} \]
    \[ 1+b \leq 1+a \]
    \[ b \leq a. \]
\end{dem}

\begin{corol}\label{int-corol-numeroOuSimetricoPositivo}
    Dado um número inteiro, ou ele é maior do que ou igual a zero, ou seu simétrico aditivo é, ou seja
    se $\overline{(a,b)} \in \mathbb{Z}$, então $\overline{(a,b)} \geq 0 = \overline{(1,1)}$ ou $-\overline{(a,b)} \geq 0$.
\end{corol}
\begin{dem}
    Como a relação de ordem $\leq$ é total conforme a \Cref{int-prop-relacaoOrdem}, vale que $\overline{(a,b)} \leq \overline{(1,1)}$ ou $\overline{(1,1)} \leq \overline{(a,b)}$. Se $\overline{(a,b)}$ é não negativo é imediato que o corolário é válido. Já no segundo caso, 
    de $\overline{(a,b)} \leq \overline{(1,1)}$ temos $a+1 \leq b+1 \implies a \leq b$. \\
    Temos que o número $-\overline{(a,b)} = \overline{(b,a)} \geq \overline{(1,1)}$ pela \Cref{int-prop-coordenadaMaior}.
\end{dem}

\begin{corol}\label{int-corol-simetricoSinalTrocado}
    Um número inteiro é não negativo se, e somente se, seu simétrico aditivo é não positivo.
\end{corol}
\begin{dem}
    Pelo \Cref{int-corol-numeroOuSimetricoPositivo}, considerando um número $x$ e seu simétrico $-x$, ao menos um deles é não negativo. Caso seja $0 \leq x$ temos $0 + (-x) \leq x + (-x) \iff -x \leq 0$, assim $-x$ é não positivo. No outro caso supomos que $0 \leq -x$, daí 
    $0 + x \leq -x + x = 0$, assim $x \leq 0$.
\end{dem}

\begin{prop}\label{int-prop-somaPositivosPositiva}
    A soma de dois inteiros não negativos é não negativa, isto é, se $x, y \in \mathbb{Z}$ com $x,y \geq 0$ então $0 \leq x+y$.
\end{prop}
\begin{dem}
    Conforme o \Cref{int-corol-simetricoSinalTrocado}, como $x,y \geq 0$ temos $-x \leq 0$ e $-y \leq 0$, como qualquer inteiro não negativo é maior do que ou igual a qualquer inteiro não positivo, temos $-y \leq +x$ e pela compatibilidade da soma com a relação de ordem, \Cref{int-prop-relacaoOrdem}, temos 
    $-y + y \leq x + y$, desse modo $x+y \geq 0$.
\end{dem}

\begin{prop}\label{int-prop-diferencaPositiva}
    Se $x,x' \in \mathbb{Z}$ com $x' \geq x$, então existe $y$ inteiro, onde $y \geq 0$ e $x' = x+y$.
\end{prop}
\begin{dem}
    Se $x' = x$ então $y=0$.
    Caso contrário, se $x'>x$ então considere $x = \overline{(a,b)}, x'=\overline{(a',b')}$, temos 
    \[ x < x' \iff a+b' < b + a' \iff b+a' = a+b'+c \iff \] 
    \[  a+c+b' = b+a' \iff (a+c,b) \sim (a',b') \text{ , com $c$ natural.} \] 

    Temos também que $\overline{(a+c,b)} = \overline{(a+c+1,b+1)} = \overline{(a,b)} + \overline{(c+1, 1)}$, onde usamos a \Cref{int-def-adicao} e a \Cref{int-prop-cancelaCoordenadas}.

    O número $\overline{(c+1,1)}$ é inteiro, pois suas entradas são números naturais, e é positivo pois $\overline{(1,1)} \leq \overline{(c+1,1)} 
    \iff 1+1 \leq 1+(c+1)$, como $c$ é natural ele não pode ser zero, conforme nossa construção. Dessa forma $\overline{(c+1,1)}$ é o $y$ da proposição.
    % O número inteiro $y = \overline{(m+1,1)}$ é tal que $y>0$ e $x' = x+ y$. Ele é positivo porque $0 = \overline{(1,1)} \leq \overline{(m+1,1)}$ uma vez que $1+1 \leq 1+(c+1)$. Ainda, $0 \neq y$ pois $c \in \mathbb{N}$ e nosso $\mathbb{N}$ não tem o zero. Agora mostremos que $x+y = x'$, temos:
    % $x+y = \overline{(a,b)} + \overline{(c+1,1)} = \overline{(a+c+1,b+1)} = \overline{(a+c,b)}$, isso usando a \Cref{int-def-adicao} e \Cref{int-prop-cancelaCoordenadas}. 
    % Temos também que $ a+c+b' = b+a' \iff (a+c,b) \sim (a',b')$ 
\end{dem}

\begin{prop}\label{int-prop-quadradoPositivo}
    O quadrado de um número inteiro é um inteiro não negativo, ou seja, se $\overline{(a,b)} \in \mathbb{Z}$, então vale $0 = \overline{(1,1)} \leq \overline{(a,b)} \cdot \overline{(a,b)}$. 
\end{prop}
\begin{dem}
    Temos $\overline{(a,b)} \cdot \overline{(a,b)} \iff \overline{(aa + bb, ab+ba)}$. Observando a \Cref{int-prop-coordenadaMaior}, queremos mostrar que $aa+bb \geq ab+ba$. Considerando a tricotomia de $\mathbb{N}$ (aplicada em $a$ e $b$) dada na \Cref{nat-soma-props} separaremos em 3 casos:
    \begin{enumerate}[label=(\roman*)]
        \item Caso $a=b$ \\
        Temos $aa+aa \geq aa+aa$.

        \item Caso $a < b$: \\
        Temos $b = a+c$, com $c \in \mathbb{N}$ \ \footnote{Lembrando que sem o zero.} \\
        \[ aa+bb = aa+(a+c)(a+c) = aa+aa+ac+ac+cc \] 
        \[ ab+ba = a(a+c)+(a+c)a = aa+ac+aa+ca \]
        Notemos que o membro da direita na penúltima linha tem o número $cc$ e na última linha não tem, assim $aa+bb \geq ab+ba$.

        \item Caso $b < a$: análogo ao caso anterior\\
        Temos $a = (b+c)$, com $c \in \mathbb{N}$
        \[ aa+bb = (b+c)(b+c) + bb = bb+bc+cb+cc+bb \]
        \[ ab+ba = (b+c)b+b(b+c) = bb+cb+bb+bc \]
        Como $aa+bb = ab+ba+cc$, provamos que $aa+bb \geq ab+ba$.
    \end{enumerate}
\end{dem}

\begin{prop}\label{int-prop-produtoPositivo}
    O produto de dois inteiros não negativos é um número não negativo.
\end{prop}
\begin{dem}
    Seja $r$ um número inteiro, tal que $0 \leq r$. Se $s \in \mathbb{Z}_{+}$, pela compatibilidade do produto com a relação de ordem \Cref{int-prop-relacaoOrdem}, temos $0 \cdot s \leq r \cdot s$, assim $0 \leq rs$.
\end{dem}
\begin{prop}
    Sejam $a, b, a', b'$ números inteiros com $a \leq a'$ e $b \leq b'$. Vale que $ab \leq a'b'$.
\end{prop}
\begin{dem}
    Temos $a' \geq a \iff a' = a + r$ com $r > 0$ (conforme \Cref{int-prop-diferencaPositiva}), e também $b' \geq b \iff b' = b+s$ com $s > 0$.
    Daí temos $a'b' = (a+r)(a+s) = aa+as+ar+rs$, e cada uma dessas parcelas é não negativa, conforme a \Cref{int-prop-produtoPositivo}, assim, essa expressão da soma é não negativa (\Cref{int-prop-somaPositivosPositiva}).
    
\end{dem}

\begin{prop}\label{int-prop-produtoMaiores}
    Sejam $\overline{(a,b)},\ \overline{(a',b')},\ \overline{(c,d)},\ \overline{(c',d')}$ números inteiros, tais que $\overline{(a,b)} \leq \overline{(a',b')}$ e $\overline{(c,d)} \leq \overline{(c',d')}$. Vale que $\overline{(a,b)} \cdot \overline{(c,d)} \leq \overline{(a',b')}
    \overline{(c',d')}$.
\end{prop}
\begin{dem}
    \[ (ac+bd) + (a'd'+b'c') \leq (ad+bc) + (a'c' + b'd') \]
    \[ \overline{(ac+bd, ad+bc)} \leq \overline{(a'c'+b'd', a'd'+b'c')} \]
    \[ \overline{(a,b)} \cdot \overline{(c,d)} \leq \overline{(a',b')} \cdot \overline{(c',d')}  \]
\end{dem}

\section{Imersão de $\mathbb{N}$ em $\mathbb{Z}$}
A imersão que trataremos a seguir justificará que utilizemos apenas um símbolo para adição, quer trabalhemos com $\mathbb{N}$, quer trabalhemos com $\mathbb{Z}$. Isso valerá também para $\mathbb{Q}$ e $\mathbb{R}$ também, que serão apresentados nos seus respectivos capítulos. A ideia de imersão trata de nos permitir corresponder um conjunto num outro, no caso, $\mathbb{N}$ em $\mathbb{Z}$, e assim trabalhar como se $\mathbb{N}$ fosse um subconjunto de $\mathbb{Z}$.

\begin{teo}\label{int-teo-imersao}
    Seja a função $f: \mathbb{N} \rightarrow \mathbb{Z}, f(x) \mapsto \overline{(x+1, 1)}$. Essa função tem as propriedades a seguir:
    \begin{enumerate}[label=(\roman*)]
        \item $f(a + b) = f(a) + f(b)$;
        \item $f(a \cdot b) = f(a) \cdot f(b)$;
        \item $a \leq b \implies f(a) \leq f(b)$.
    \end{enumerate}
\end{teo}
\begin{dem}
    \begin{enumerate}[label=(\roman*)]
        \item $f(a + b) = f(a) + f(b)$\\
        
            $f(a + b) = \overline{(a+b+1, 1)}
                     = \overline{(a+1+b+1, 1+1)} 
                     = f(a) + f(b)$.
        
    
        \item $f(a \cdot b) = f(a) \cdot f(b)$;
        \begin{align*}
            f(a \cdot b) &= \overline{(ab+1, 1)}\\
                        &= \overline{(ab+a+b+1+1, a+1+b+1)}\\
                        &= \overline{((a+1)(b+1)+1, (a+1)1 + 1(b+1))}\\
                        &= \overline{(a+1,1)} \cdot \overline{(b+1,1)}\\
                        &= f(a) + f(b).
        \end{align*}        
        
        \item $a \leq b \implies f(a) \leq f(b)$ \\
        Se $a=b$ então $f(a)=f(b)$ é óbvio. Suponhamos, por outro lado, $a \neq b$.
        Então $a<b \iff b=a+m$ para algum $m \in \mathbb{N}$.
        Temos então $f(a) = \overline{(a+1,1)}$ e $f(b) = f(a+m) = \overline{(a+m+1,1)}$.
        Vemos que $\overline{(a+1,1)} \leq \overline{(a+m+1,1)}$ porque $a+1+1 \leq 1 + a + m + 1$.
    \end{enumerate}
\end{dem}
\begin{obs}
    A partir de agora, temos a liberdade de escrever os números inteiros com a notação usual de $\mathbb{N}$, isto é $1 = \overline{(2,1)}, 2 = \overline{(3,1)}, 3 = \overline{(4,1)}$ e assim por diante, além de representar o $\overline{(1,2)} = -1,\ \overline{(1,3)} = -2,\ \overline{(4,1)} = -3$ analogamente, ad infinitum.
\end{obs}

\begin{ex}\label{int-ex-imersaoProduto}
    Considerando a imersão provada acima, podemos entender o produto $2 \cdot \overline{(a,b)}$ da seguinte forma:
    $2 = \overline{(3,1)}$ aí $\overline{(3,1)} \cdot \overline{(a,b)} = \overline{(3a+b,3b+a)} = \overline{(2a,2b)}$.
    Essa é a notação usual quando se trabalha com vetores onde a multiplicação por um escalar funciona da maneira usual 
    $k \cdot \overline{(a,b)} = \overline{(ka,kb)}$ quando $k > 0$, já quando $k < 0$ precisamos inverter as coordenadas, o que não é usual.
\end{ex}
A imersão de $\mathbb{N}$ em $\mathbb{Z}$ permite-nos interpretar $\mathbb{N}$ como um subconjunto de $\mathbb{Z}$, embora pela nossa construção fique evidente que isso não ocorra, nos termos de elementos dos conjuntos. Mas, por outro lado, as operações de adição e multiplicação e a relação de ordem funcionam analogamente em $\mathbb{Z}$, ou seja, o objetivo inicial de construir mantendo certa semelhança funcionou.

Essa é uma parte importante da construção dos números inteiros. Em geral não importa o que é um número, mas apenas o que conseguimos fazer com ele, as regras do jogo. Por outro lado, a construção do conjunto dos números inteiros mostra-nos que, se existir um conjunto $\mathbb{N}$, a lógica e a teoria de conjuntos elementar que assumimos até agora, então existe um conjunto $\mathbb{Z}$ que podemos manipular com as regras mostradas.

Por certo ponto de vista, não interessa o que são e como são definidas as operações de soma e produto, mas, admitindo que exista um conjunto $A$ com uma operação de soma e uma operação de produto, ambas tendo certas propriedades, é possível provar teoremas sem levar em conta a \emph{natureza} dos entes matemáticos envolvidos.

A construção de $\mathbb{Z}$, as apresentações e provas das proposições apresentadas caminham nesse sentido de permitir uma abstração do conjunto, para que não seja mais necessário trabalhado com pares ordenados (ainda mais sem o $0$ em $\mathbb{N}$!).

Poderíamos desenvolver alguns tópicos importantes levando em conta a natureza dos nossos números inteiros, que são pares ordenados. Isso é satisfatório, se conseguirmos provar o que queremos. Por outro lado, tratar com conjuntos genéricos (independente da construção utilizada) permite uma abstração e uma generalização, mostrando que certas características não dependem de um conjunto específico (no nosso caso $\mathbb{N} \times \mathbb{N} / \sim$). As abstrações permitem formar conclusões dependendo de apenas algumas caracterísitcas/propriedades do conjunto. Dessa maneira, em geral a pergunta "qual conjunto de números inteiros?" não faz sentido, pois o que nos importa é o que podemos fazer com ele, e não como ele foi construído. Isso se tornará importante na construção de $\mathbb{R}$, podendo ser feita de dois métodos que formalizam o conjunto, mas nos interessa como $\mathbb{R}$ foi construído porque nosso tema é construção dos números, mas na utilização do conjunto, se for feito por séries ou por cortes de Dedekind não é relevante. 

Tá! Mas e aí, muitos números e cadê o $15$, o $35$, o $-351$? A resposta é que nesse momento não possuimos um sistema de numeração, que para ser desenvolvido precisa da divisão euclidiana. O leitor interessado em verificar o sistema de numeração para $\mathbb{Z}$ pode consultar \cite{hefez-algebra}.

\begin{teo}\label{int-teo-ilimitadoSuperiormente}
    O conjunto $\mathbb{N}$ não é limitado superiormente em $\mathbb{Z}$.
\end{teo}
\begin{dem}
    \todo{demonstrar}
    .
\end{dem}

\end{document}