\documentclass[../main.tex]{subfiles}
\begin{document}
\chapter{O CONJUNTO DOS NÚMEROS INTEIROS}\label{cap-inteiros}
Neste capítulos faremos a construção dos números inteiros. A bibliografia principal continua sendo \textcite{ferreira} e \textcite{domingues-2009}. Será apresentada uma relação de equivalência que servirá para criar $\mathbb{Z}$. Depois definiremos uma adição e um produto que tenham algumas propriedades que já eram válidas em $\mathbb{N}$, isto é, queremos "estender"\ $\mathbb{N}$.

Devemos observar que em $\mathbb{N}$ temos que a subtração, tal como conhecemos no ensino básico, não é uma operação (no sentido da álgebra), pois a subtração não está definida para quaisquer dois elementos de $\mathbb{N}$ tal que o resultado esteja em $\mathbb{N}$.

Com o objetivo de contornar esse problema (e portanto poder recriar a subtração como no ensino básico, em que ela seja uma operação sobre um conjunto), iremos criar um novo conjunto a partir de $\mathbb{N}$. Veremos que isso é possível, tal conjunto será denotado por $\mathbb{Z}$ e o chamaremos ele de conjunto dos \emph{números inteiros}.

Nos números naturais, poderíamos ter definido uma função que tivesse o intuito de fazer o papel de inverso da soma, que seria a subtração, denotada por "$-$". Seguindo esse raciocínio poderíamos mostrar que $9-3 = 8-2 = 7-1$. Concluiríamos com base nessas igualdades, que $9 + 2 = 8 + 3$ e $8 + 1 = 7 + 2$, o que em ambos os casos, os resultados são números naturais. 

Além disso, queremos expressar um número inteiro sem ter que assumir novos conceitos além de $\mathbb{N}$, da lógica e a teoria de conjuntos que já foram assumidas no capítulo anterior. Além disso, daremos significado à expressões do tipo $3-5, 4-8, 2-3$, que nesses exemplos não são números naturais.

A maneira como expressaremos será por meio de uma relação binária $\sim$ sobre $\mathbb{N} \times \mathbb{N}$ definida desse modo: $(a,b) \sim (c,d) \iff a+d = b+c$, sendo $a,b,c,d$ números naturais quaisquer.

\begin{ex}
    Os pares ordenados $(1,2)$ e $(5,6)$ se relacionam através da relação $\sim$, pois $1+6 = 2+5$.
\end{ex}
\begin{ex}
    Os pares ordenados $(1,2)$ e $(5,5)$ não se relacionam através da relação $\sim$, pois $1+5 \neq 2+5$.
\end{ex}
\begin{ex}\label{int-ex-elementosRelacionadosPorSim}
    Os pares ordenados $(6,6)$ e $(8,8)$ se relacionam por meio da relação $\sim$, pois $6+8 = 6+8$. Fica claro que para quaisquer naturais $m,n$ temos que $(m,m) \sim (n,n)$, uma vez que sempre ocorrerá $m+n = m+n$.
\end{ex}


\begin{teo}\label{int-teo-classeEquivalenciaNumero}
    A relação $\sim$ é de equivalência.
\end{teo}
\begin{dem}
    Sejam $a,b,c,d \in \mathbb{N}$. A relação $(a,b) \sim (c,d) \iff a+d = b+c$ possui as seguintes propriedades:
    \begin{enumerate}[label=(\roman*)]
        \item Reflexiva: $(a,b) \sim (a,b)$, pois $ a+b=b+a$.
        \item Simétrica: $(a,b) \sim (c,d) \implies (c,d) \sim (a,b)$ \\
        Temos $ a+d = b+c \implies c+b = d+a \iff (c,d) \sim (a,b)$.
        \item Transitiva: $(a,b) \sim (c,d) \land (c,d) \sim (e,f) \implies (a,b) \sim (e,f)$. \\
        Supondo $(a,b) \sim (c,d) $ e $(c,d) \sim (e,f)$ obtemos $a+d=b+c$ e $c+f=d+e$. \\
        Somando termo a termo temos: \\ 
        $(a+d)+(c+f)= (b+c)+(d+e) \iff a+f = b+e \iff (a,b) \sim (e,f)$.
    \end{enumerate}
\end{dem}

\begin{ex}\label{int-ex-classesIguaisRepresentantesDiferentes}
    Sabemos que $(1,1) \sim (2,2)$, e conforme a notação utilizada no capítulo anterior, temos que $\overline{(1,1)} = \overline{(2,2)}$.
\end{ex}
\begin{ex}
    As representações $\overline{(1,5)}, \overline{(2,6)}, \overline{(3,7)}$ representam a mesma classe de equivalência, pois $1+6 = 5+2$ e $2+7=6+3$.
\end{ex}
\begin{ex}
    Os elementos $\overline{(10,1)}$ e $\overline{(12,3)}$ representam a mesma classe de equivalência pois $(10,1) \sim (12,3)$ uma vez que
    $10+3 = 1 + 12$, e conforme o \Cref{agb-teo-relacaoEquivalenciaPropriedades} se os elementos estão relacionados via relação de equivalência, então eles estão na mesma classe de equivalência.
\end{ex}
\begin{ex}
    Os elementos $\overline{(1,5)} = \overline{(2,6)} = \overline{(3,7)}$, em que cada representante da classe, a saber, $(1,5), (2,6)$ e $(3,7)$ representam a mesma classe de equivalência. A ideia de criar essas classes é que um número inteiro é uma classe de equivalência. Embora essa não seja uma ideia normal de ser encontrada fora do contexto de criação dos conjuntos numéricos (porque o objetivo de usar pares ordenados e classes de equivalência não é facilitar o manuseio de $\mathbb{Z}$, mas sim formalizar esse conjunto, com suas operações e relações como conhecemos).
\end{ex}
% \begin{ex}
%     A ideia intuitiva com a utilização dos pares é que a primeira coordenada do par seja a "parte positiva"\ enquanto que a segunda coordenada seja a "parte negativa". Com essa intuição futuramente vamos justificar a compreensão do $5 \in \mathbb{N}$ como $\overline{(6,1)}$, pois $6-1 = 5$. 
%     De modo a extender essa notação, como conhecemos no ensino básico, teremos que o $-5 = \overline{(1,6)}$, sendo que nessa representação o $-5$ é um número inteiro, e o $1$ e o $6$ são números naturais.
% \end{ex}

\begin{defi}\label{int-def-conjuntoZ}
    O conjunto quociente $\mathbb{N} \times \mathbb{N} / \sim = \{ \overline{(a,b)}: (a,b) \in \mathbb{N} \times \mathbb{N}\}$ será chamado de $\emph{conjunto dos números inteiros}$ e será denotado por $\mathbb{Z}$.
\end{defi}

\begin{obs}
    Utilizaremos a mesma notação para adição e multiplicação de números inteiros, que utilizamos nos números naturais. Inicialmente para o desenvolvimento do texto, sempre que $a,b \in \mathbb{N}$ entenderemos a soma $a+b$ como sendo executada em $\mathbb{N}$. Caso $a,b \in \mathbb{Z}$ entenderemos a soma $a+b$ sendo executada em $\mathbb{Z}$. Posteriormente justificaremos a escolha de utilizar os mesmos símbolos sendo que os conjuntos são diferentes. Isso se repetirá com a multiplicação e com a relação de ordem.
\end{obs}

Com essa definição de número inteiro poderemos interpretar, a nível de ensino básico, que o número $\overline{(a,b)}$ é $a - b$. Caso fôssemos definir a subtração em $\mathbb{N}$, ela não seria uma operação, pois estaria definida apenas quando $a > b$.

\begin{prop}\label{int-prop-cancelaCoordenadas}
    Se $a,b,c \in \mathbb{N}$ vale que: $\overline{(a+c,b+c)} = \overline{(a,b)}$.
\end{prop}
\begin{dem}
    A demonstração é imediata, visto que $(a+c)+b = (b+c)+a$, ou seja, $(a+c,b+c) \sim (a,b)$ e pelo \Cref{agb-teo-relacaoEquivalenciaPropriedades} temos $\overline{(a+c,b+c)} = \overline{(a,b)}$.
\end{dem}

\section{A adição em $\mathbb{Z}$}
Vamos definir uma adição em $\mathbb{Z}$, mantendo algumas propriedades da adição em $\mathbb{N}$.
\begin{defi}\label{int-def-soma}
    Dados $\overline{(a,b)}$ e $\overline{(c,d)}$ em $\mathbb{Z}$, definimos a adição $\overline{(a,b)} + \overline{(c,d)}$ como $\overline{(a+c, b+d)}$.
\end{defi}
\begin{prop}
    A operação de adição está bem definida em $\mathbb{Z}$. Isto é, a adição em $\mathbb{Z}$ não depende do representante das classes de equivalência envolvidas na adição. 
\end{prop}
\begin{dem}
    Sejam $a,a',b,b',c,c',d,d' \in \mathbb{N}$. Vamos provar que 
    \[ \overline{(a,b)} = \overline{(a',b')} \land \overline{(c,d)} = \overline{(c',d')}
    \implies \overline{(a,b)} + \overline{(c,d)} = \overline{(a',b')} + \overline{(c',d')}. \]

    \noindent Vamos desenvolver as duas somas e mostrar que são iguais. Consideremos  
    \[\overline{(a,b)} + \overline{(c,d)} = \overline{(a+c,b+d)}.\]
    Considerando agora a segunda soma temos: 
    \[ \overline{(a',b')} + \overline{(c',d')} = \overline{(a'+c',b'+d')}. \]
    Precisamos agora apenas mostrar que 
    \[ (a+c,b+d) \sim (a'+c',b'+d'), \]
    isso mostrará que 
    \[ \overline{(a+c,b+d)} = \overline{(a'+c',b'+d')}. \] 
    Como $\overline{(a,b)} = \overline{(a',b')} \iff a+b' = b+a'$ e ainda
     $\overline{(c,d)} = \overline{(c',d')} \iff c+d' = d+c'$ teremos então que 
    
         \[ (a+b')+(c+d') = (b+a')+(d+c') \]
         ou seja
         \[ (a+c) + (b'+d') = (b+d) + (a'+c') \]
         isto é
         \[ (a+c,b+d) \sim (a'+c',b'+d'). \]
    
\end{dem}

\begin{teo}\label{int-teo-somaPropriedades}
    Sejam $\overline{(a,b)}, \overline{(c,d)}, \overline{(e,f)}$ números inteiros quaisquer. Para a adição em $\mathbb{Z}$, valem as seguintes propriedades:
    \begin{enumerate}[label=(\roman*)]
        \item Fechamento;
        \item Associativa;
        \item Comutativa;
        \item Da existência do elemento neutro; 
        \item Da existência do elemento simétrico;
        \item Lei do cancelamento.
    \end{enumerate}
\end{teo}

\begin{dem}
    Sejam $a,b,c,d,e,f \in  \mathbb{N}$.
    \begin{enumerate}[label=(\roman*)]
        \item Fechamento: $\overline{(a,b)} + \overline{(c,d)} \in \mathbb{Z}$:
        É imediata, pois \[ a+c \in \mathbb{N} \land b+d \in \mathbb{N} \implies \overline{(a+c,b+d)} \in \mathbb{Z}. \]

        \item Associativa: 
        \begin{align*}
            \left(\overline{(a,b)} + \overline{(c,d)}\right) +  \overline{(e,f)} 
            &= \overline{(a+c,b+d)}+\overline{(e,f)} \\ 
            &= \overline{(a+c+e, b+d+f)} \\
            &= \overline{ \left( a+(c+e), b+(d+f) \right) } \\
            &= \overline{(a,b)} + \overline{ (c+e, d+f) } \\
            &= \overline{(a,b)} + \left( \overline{(c,d)} + \overline{(e,f)} \right).
        \end{align*}
        % \left(\overline{(a,b)} + \overline{(c,d)}\right) +  \overline{(e,f)} =
        % \overline{(a+c,b+d)}+\overline{(e,f)} = \overline{(a+c+e, b+d+f)} = \overline{ \left( a+(c+e), b+(d+f) \right) } = \overline{(a,b)} + 
        % \overline{ (c+e, d+f) } = \overline{(a,b)} + \left( \overline{(c,d)} + \overline{(e,f)} \right)$.
       
        \item Comutativa: $\overline{(a,b)} + \overline{(c,d)} = \overline{(a+c,b+d)} = \overline{(c+a,d+b)} = \overline{(c,d)} + \overline{(a,b)}$.
       
        \item Da existência do elemento neutro: $\exists! x \in \mathbb{Z} : x + \overline{(a,b)} = \overline{(a,b)} + x = \overline{(a,b)}$; \\
        Consideremos $\overline{(1,1)}$, vamos mostrar que ele é o $x$ acima. 
        Provaremos que é neutro pela esquerda. Temos que $\overline{(1,1)} + \overline{(a,b)} = \overline{(1+a,1+b)}$. Provemos que  \\ $\overline{(1+a,1+b)} \sim \overline{(a,b)}$.
        Para isso basta notar que $(1+a)+b = (1+b)+a$. Portanto $\overline{(1,1)}$ é neutro pela esquerda. Analogamente prova-se que é neutro pela direita. Além disso, todo elemento neutro para uma operação é único, conforme o \Cref{agb-teo-neutroUnico}, o que conclui nossa prova. Denotaremos o elemento neutro da adição por $0$ e chamaremos de zero. Conforme os \Cref{int-ex-elementosRelacionadosPorSim,int-ex-classesIguaisRepresentantesDiferentes} podemos representar de maneiras diferentes esse elemento neutro, mas sempre entendendo, em última instância, como $\overline{(m,m)}$ para qualquer $m \in \mathbb{N}$.
       
        \item Da existência do simétrico: \\
        Vamos mostrar que para qualquer inteiro $\overline{(a,b)}$, o elemento $\overline{(b,a)}$ é seu simétrico. 
        Temos que $\overline{(a,b)}+ \overline{(b,a)} = \overline{(a+b,b+a)} = \overline{(1,1)}$, onde esse último elemento é o neutro da soma. Uma vez que a adição também é comutativa, temos $ \overline{(b,a)} + \overline{(a,b)} = \overline{(1,1)}$, o que mostra que $\overline{(b,a)}$ é simétrico de $\overline{(a,b)}$. Que ele é único é consequência do \Cref{agb-teo-simetricoUnico}.
        % A comutatividade nós já temos, basta provar que $x + y = 0$.
        % Consideremos então $x = \overline{(a,b)}$, mostraremos que $y = \overline{(b,a)}$ é tal que $\overline{(a,b)} + \overline{(b,a)} = \overline{(1,1)}$. Temos que $\overline{(a+b,b+a)} = \overline{(1,1)}$ porque $a+b+1 = b+a+1$, o que sempre ocorre. Pelo \Cref{agb-teo-simetricoUnico} concluímos também que esse simétrico é único.
       
        \item Lei do cancelamento: $\overline{(a,b)} + \overline{(c,d)} = \overline{(a,b)} + \overline{(e,f)} \implies \overline{(c,d)} = \overline{(e,f)}$. \\
        Temos 
        \begin{align*}
            \overline{(a,b)} + \overline{(c,d)} = \overline{(a,b)} + \overline{(e,f)} 
            &\implies \overline{(a+c,b+d)} = \overline{(a+e,b+f)} \\ 
            &\implies (a+c)+(b+f)=(b+d)+(a+e) \\
            &\implies c+f = d+e \\ 
            &\implies \overline{(c,d)} = \overline{(e,f)}.
        \end{align*}
        Desse modo a adição admite a lei do cancelamento à esquerda. Para ver que o cancelamento à direita também é válido, basta notar que a adição é comutativa.
    \end{enumerate}
\end{dem}



Com o \Cref{int-teo-somaPropriedades}, podemos ver que a comutatividade, a associatividade e o fechamento são mantidos, com a diferença que agora trabalhamos com números inteiros e não naturais. Indo além, é possível ver que o elemento neutro da soma e o simétrico em $\mathbb{Z}$, surgem naturalmente, embora não possamos dizer que o neutro, de acordo com o \Cref{cap-naturais} é $\overline{(0,0)}$, pois nenhuma das coordenadas desse par ordenado são números naturais. Não obstante, para qualquer $a \in \mathbb{N}$ temos que $(a,a) \sim (1,1)$ o que garante que $\overline{(a,a)} =\overline{(1,1)} = 0 \in \mathbb{Z}$. Também temos agora um simétrico para a soma, o que em $\mathbb{N}$ não ocorre. Por último, observando a \Cref{agb-def-operacao} concluímos que a adição em $\mathbb{Z}$ é uma operação.

\begin{obs}\label{int-obs-notacaoSimetrico}
    Com relação ao elemento simétrico da soma, como ele sempre existe e é único, denotaremos o simétrico de $x = \overline{(a,b)} \in \mathbb{Z}$ por $-x = -\overline{(a,b)} = \overline{(b,a)}$. Com isso, podemos definir a subtração em $\mathbb{Z}$ de uma maneira relativamente simples.  
\end{obs}

\begin{defi}\label{int-def-subtracao}
    Sejam $x,y \in \mathbb{Z}$. A subtração de $x$ por $y$, denotada $x - y$ é definida como $x + (-y)$, onde $-y$ é o simétrico de $y$.
\end{defi}
\begin{prop} 
    Sejam $x,y \in \mathbb{Z}$, são válidas as seguintes propriedades:
    \begin{enumerate}[label=(\roman*)]
        \item $-x + y = y - x$;
        \item $x - ( - y) = x+y$;
        \item $-x-y = -(x+y)$.
    \end{enumerate}
\end{prop}
\begin{dem}
    Sejam $x, y \in \mathbb{Z}$, temos:
        \begin{enumerate}[label=(\roman*)]
        \item Pela comutatividade da soma, $(-x) + y = y + (-x) = y-x$;
        \item Como $-(-y) = y$, pela \Cref{agb-prop-simetricoSimetrico}, segue que $x - ( - y) = x+y$;
        \item Vamos mostrar que $-(x+y)$ é simétrico de $x+y$, e também que $-x-y$ é simétrico de $x+y$. De $-(x+y) + (x+y)$
            temos $(x+y) -(x+y) = 0$, o que mostra que $-(x+y)$ é simétrico de $(x+y)$. Para mostrar que $(-x-y)$ é simétrico de $(x+y)$ temos que 
            $(-x-y)+(x+y) = (-x)+(-y)+(x)+(y) = x-x+y-y = 0$. Conforme o \Cref{agb-teo-simetricoUnico}, o simétrico é único, e portanto $-(x+y)=-x-y$.
    \end{enumerate}
\end{dem}

\section{A multiplicação em $\mathbb{Z}$}
Agora vamos definir uma multiplicação em $\mathbb{Z}$. Podemos entender a multiplicação em $\mathbb{Z}$ como uma tentativa de extensão da operação de multiplicação válida em $\mathbb{N}$. Com isso, várias propriedades que queremos serão mantidas em $\mathbb{Z}$.
\begin{defi}\label{int-def-produto}
    Dados $\overline{(a,b)}$ e $\overline{(c,d)}$ em $\mathbb{Z}$, definimos a multiplicação como 
    \[ \overline{(a,b)} \cdot \overline{(c,d)} = \overline{(a \cdot c + b \cdot d, a \cdot d + b \cdot c)}. \]
\end{defi}
% Podemos observar a título de intuição que, se as primeiras coordenadas representam a parte "positiva" do número e as segundas coordenadas a parte negativa, a multiplicação de dois positivos é positiva, e a multiplicação de dois negativos é positiva, então esses resultados ficam na primeira coordenada. Semelhantemente, quando um é positivo e outro negativo, o resultado fica negativo, e assim fica na segunda coordenada.
\begin{teo}\label{int-teo-produtoBemDefinido}
    A operação de multiplicação está bem definida em $\mathbb{Z}$. Isto é, a multiplicação em $\mathbb{Z}$ não depende do representante das classes de equivalência. 
\end{teo}
\begin{dem}
    Sejam $a,a',b,b',c,c',d,d' \in \mathbb{N} $ tais que 
    \[ \overline{(a,b)} = \overline{(a',b')} \land \overline{(c,d)} = \overline{(c',d')}\]
    Vamos mostrar que
    \[ \overline{(a,b)} \cdot \overline{(c,d)} = \overline{(a',b')} \cdot \overline{(c',d')}. \]

    % Pela \Cref{int-def-produto}, $\overline{(a,b)} = \overline{(a',b')}$ e $\overline{(c,d)} = \overline{(c',d')}$ 
    Pela definição de multiplicação, temos que 
    \[ \overline{(a,b)} \cdot \overline{(c,d)} = \overline{(ac+bd,ad+bc)} \text{ e } \overline{(a',b')} \cdot \overline{(c',d')} = \overline{(a'c'+b'd',a'd'+b'c')}.\]
    
    Vamos considerar agora as igualdades das classes. Como 
    %     \[ \overline{(a,b)} = \overline{(a',b')} \iff a + b' = b + a' \] 
    %     disso concluímos que 
    %     \[ (a + b')c' = (b + a')c' \text{ e que } (a + b')d' = (b + a')d' .\] 
    % Considerando 
    % \[ \overline{(c,d)} = \overline{(c',d')} \iff c + d' = d + c',\] 
    % temos que
    % \[ a(c + d') = a(d + c') \] 
    % e também que 
    % \[ b(c + d') = b(d + c'). \]
    \begin{align}
        \overline{(a,b)} = \overline{(a',b')} &\iff a + b' = b + a', \nonumber \\        \shortintertext{disso concluímos que}  
        (a + b')c' &= (b + a')c' \label{int-dummyProduto-eq1} \\         \shortintertext{e que} 
        (a + b')d' &= (b + a')d' . \label{int-dummyProduto-eq2} \\        \shortintertext{Considerando} 
        \overline{(c,d)} = \overline{(c',d')} &\iff c + d' = d + c', \nonumber \\        \shortintertext{temos que}  
        a(c + d') &= a(d + c') \label{int-dummyProduto-eq3} \\        \shortintertext{e também que}  
        b(c + d') &= b(d + c'). \label{int-dummyProduto-eq4}
    \end{align}
    
    Considerando as \Cref{int-dummyProduto-eq1,int-dummyProduto-eq2,int-dummyProduto-eq3,int-dummyProduto-eq4} que obtemos acima, aplicando a distributiva ficamos respectivamente, com:
    \begin{align*}
        ac'+b'c' &= bc'+a'c' \text{ (\Cref{int-dummyProduto-eq1}) }, \\
        bd'+a'd' &= ad'+b'd' \text{ (\Cref{int-dummyProduto-eq2}) },\\
        ac+ad'   &= ad+ac' \text{ (\Cref{int-dummyProduto-eq3}) },\\
        bd+bc'   &= bc+bd' \text{ (\Cref{int-dummyProduto-eq4}) }.
    \end{align*}
    Vamos agora somar termo a termo essas equações e obter:
    \[ ac'+b'c' + bd'+a'd' + ac+ad' + bd+bc' = bc'+a'c' + ad'+b'd' + ad+ac' + bc+bd'. \]
    Cancelando os termos $ac', bd', ad', bc'$ ficamos com:
    
    \[ b'c' +a'd' + ac + bd  = a'c' + b'd' + ad+ bc, \]
    ou seja,
    \[ ac + bd + a'd' + b'c' = ad + bc + a'c' + b'd'. \]
    
    Com isso, provamos que $ (ac+bd,ad+bc) \sim (a'c'+b'd',a'd'+b'c')$, o que nos mostra que $\overline{(a,b)} \cdot \overline{(c,d)} = 
    \overline{(a',b')} \cdot \overline{(c',d')}$ e o produto não depende do representante da classe.
\end{dem}
\begin{teo}\label{int-teo-produtoPropriedades}
    Sejam $\overline{(a,b)}, \overline{(c,d)}, \overline{(e,f)}$ números inteiros quaisquer. Para a multiplicação valem as seguintes propriedades:
    \begin{enumerate}[label=(\roman*)]
        \item Fechamento;
        \item Associativa;
        \item Comutativa;
        \item Da existência do elemento neutro; 
        \item Da existência do elemento simétrico;
        \item Lei do cancelamento\footnote{A lei do cancelamento do produto é para todo elemento diferente do neutro da soma, ou seja, diferente de $0$.}.
    \end{enumerate}
\end{teo}
\begin{dem}
    Sejam $a,b,c,d,e,f \in \mathbb{N}$.
    \begin{enumerate}[label=(\roman*)]
        \item Fechamento: \\
        Temos $\overline{(a,b)} \cdot \overline{(c,d)} = \overline{(ac+bd, ad+bc)}$. Como $ac+bd \in \mathbb{N}$ e também $ad+bc \in \mathbb{N}$ temos $\overline{(ac+bd, ad+bc)} \in \mathbb{Z}$.

        \item Associativa: 
        \begin{align*}
            \big(\overline{(a,b)} \cdot \overline{(c,d)}\big) \cdot  \overline{(e,f)} &= \overline{(ac+bd, ad+bc)} \cdot \overline{(e,f)} \\
            &= \overline{((ac+bd)e + (ad+bc)f , (ac+bd)f + (ad+bc)e} \\
            &= \overline{(ace+bde+adf+bcf, acf+bdf+ade+bce)} \\
            &= \overline{(ace+adf+bcf+bde, acf+ade+bce+bdf)} \\
            &= \overline{(a(ce+df) + b(cf+de) , a(cf+de) + b(ce+df))} \\
            &= \overline{(a,b)} \cdot \overline{(ce+df, cf+de)} \\
            &= \overline{(a,b)} \cdot \big( \overline{(c,d)} \cdot \overline{(e,f)} \big).
        \end{align*}
              
        \item Comutativa: \\
        $\overline{(a,b)} \cdot \overline{(c,d)} = \overline{(ac+bd, ad+bc)} =
        \overline{(ca+db, bc+da)} = \overline{(c,d)} \cdot \overline{(a,b)}$ .     
        
        \item Da existência do elemento neutro: \\
        Consideremos o número $\overline{(2,1)}$. Vamos mostrar que ele é neutro pela direita. Temos $\overline{(a,b)} \cdot \overline{(2,1)} = \overline{(2a+b, a+2b)} = \overline{(a,b)}$. A última igualdade ocorre porque $a + a + 2b = b + 2a + b$. Analogamente, se mostra que $\overline{(2,1)}$ é neutro pela esquerda. A unicidade é garantida pelo \Cref{agb-teo-neutroUnico}, a menos de escolha do representante da classe.
        
        \item Distributiva: \\
        Primeiro vamos mostrar a distributiva à esquerda.
        \begin{align*}
            \overline{(a,b)} \cdot \left( \overline{(c,d)} + \overline{(e,f)} \right) &= \overline{(a,b)} \cdot (\overline{(c+e,d+f)}) \\
            &= \overline{(a(c+e) + b(d+f), a(d+f) +b(c+e)} \\
            &= \overline{(ac+ae+bd+bf, ad+af+bc+be)} \\
            &= \overline{(ac+bd, ad+bc)} + \overline{(ae+bf, af+be)} \\
            &= \big( \overline{(a,b)} \cdot \overline{(c,d)} \big) + \big( \overline{(a,b)} \cdot \overline{(e,f)} \big)  .  
        \end{align*}
        Pela comutatividade do produto em $\mathbb{Z}$ e pela distributividade à esquerda, a distributividade a direita está também provada.

        \item Lei do cancelamento: \\
        Suponha que $\overline{(a,b)} \neq 0$ seja tal que $\overline{(a,b)} \cdot \overline{(c,d)} = \overline{(a,b)} \cdot \overline{(e,f)}$. Ou seja,  
        % \begin{center}
        %     $\overline{(a,b)} \cdot \overline{(c,d)} = \overline{(a,b)} \cdot \overline{(e,f)}$ \\
        \[ \overline{(ac+bd, ad+bc)} = \overline{(ae+bf, af+be)}, \]
        % \end{center}
        o que equivale a
        \[ ac+bd+af+be = ad+bc+ae+bf, \]
        isto é, 
        \[ a(c+f) + b(d+e) = a(d+e) + b(c+f). \]
        Como $\overline{(a,b)} \neq 0$, $a \neq b$. Suponhamos, sem perda de generalidade, que $a > b$. Temos que $a = b + m$ para algum $m$ natural. Substituindo na igualdade anterior, ficamos com 
        \[ (b+m)(c+f)+ b(d+e) = (b+m)(d+e)+b(c+f) \]
        e então,
        \[ bc + bf + mc + mf + bd + be = bd + be + md + me + bc + bf. \]
        
        Cancelando os termos $bc, bf, bd, be$, obtemos
        \begin{align*}
            mc + mf = md + me 
            &\iff m(c+f) = m(d+e) \\ 
            &\iff c+f = d+e  \\
            &\iff (c,d) \sim (e,f).
        \end{align*}
        E portanto
       \[ \overline{(c,d)} = \overline{(e,f)}. \]
    \end{enumerate}
\end{dem}

Podemos usar a notação como na \Cref{int-obs-notacaoSimetrico} e na \Cref{int-def-subtracao}. Assim, se $x = \overline{(a,b)}$, poderemos quando conveniente trabalhar somente com a notação da variável $x$, ao invés de usar a classe. Por outro lado, em vista de necessidade, não abandonaremos a notação das classes de equivalência.

\begin{teo}\label{int-teo-produtoRegraSinal1}
    Sejam $x, y$ números inteiros quaisquer. É válido que 
    \[ -(xy) = x(-y) = (-x)y. \]
\end{teo}
\begin{dem}
    A demonstração consiste em explorar a unicidade do simétrico, conforme o \Cref{agb-teo-simetricoUnico}.
    Vamos mostrar que $xy$ é o simétrico de todos no enunciado. Basta observar que $xy + (-xy) = 0$.
    Também vale que $xy + x(-y) = x(y+(-y)) = x0 = 0$. Por último, $xy + (-x)y = (x + (-x))y = 0y = 0$.
\end{dem}
\begin{corol}\label{int-corol-produtoRegraSinal2}
    Se $x,y \in \mathbb{Z}$, então vale $(-x)(-y) = xy$.
\end{corol}
\begin{dem}
    Observando o \Cref{int-teo-produtoRegraSinal1} concluímos que vale \\ $(-x)(-y) = -(x(-y)$, como o produto é comutativo, temos $-((-y)x) = -(-(yx))$,
    o que, pela \Cref{agb-prop-simetricoSimetrico} implica que $-(-(yx)) = yx = xy$.
\end{dem}

Com isso, podemos observar que $\mathbb{Z}$ com a soma dada na \Cref{int-def-soma} e com o produto dado na \Cref{int-def-produto}, tem as propriedades enunciadas na \Cref{agb-def-anel}, e portanto é um anel.

\section{A relação de ordem em $\mathbb{Z}$}
\begin{defi}\label{int-def-relacaoOrdem}
    Dados $\overline{(a,b)}$ e $\overline{(c,d)}$ em $\mathbb{Z}$, definimos a relação de ordem $\leq$ e dizemos que $\overline{(a,b)}$ é menor do que ou igual a $\overline{(c,d)}$ quando $a+d \leq b+c$ e denotamos por $\overline{(a,b)} \leq \overline{(c,d)}$.
\end{defi}

\begin{teo}\label{int-teo-relacaoOrdemBemDefinida}
    A relação de ordem está bem definida, isto é, para $\overline{(a,b)} = \overline{(a',b')}$ e $\overline{(c,d)} = \overline{(c',d')}$, se $\overline{(a,b)} \leq \overline{(c,d)}$, então $\overline{(a',b')} \leq \overline{(c',d')}$.
\end{teo}
\begin{dem}
    % Sejam $\overline{(a,b)} = \overline{(a',b')} \iff a+b' = b+a'$. 
    % Também $\overline{(c,d)} = \overline{(c',d')} \iff c+d' = d+c'$. 
    % Como $\overline{(a,b)} \leq \overline{(c,d)} \iff a+d \leq b+c$, temos:
    % \begin{align*}
    %     b+a'+c+d' &= a+b' + d+c' \\
    %     a'+d' + a+d + m &= b' + c' + a + d \\
    %     a'+d' + m &= b' + c'  \\
    %     a'+d' & \leq b'+c'.
    % \end{align*}

    Sejam $\overline{(a,b)} = \overline{(a',b')}$ e $\overline{(c,d)} = \overline{(c',d')}$. Logo $a+b' = b+a'$ e $c+d' = d+c'$.
   
    Supondo $\overline{(a,b)} \leq \overline{(c,d)} \iff a+d \leq b+c$, temos $a+d \leq b+c$ e existe $m \in \mathbb{N}$ tal que $(a+d)+m=b+c$. Assim
    
    \begin{align*}
        b+a'+c+d' &= a+b' + d+c' \\
        a'+d' + a+d + m &= b' + c' + a + d \\
        a'+d' + m &= b' + c'  \\
        a'+d' & \leq b'+c',
    \end{align*}
    o que mostra que $\overline{(a',b')}\leq\overline{(c',d')}$
     
\end{dem}
\begin{teo}{Sejam $\overline{(a,b)}, \overline{(c,d)}, \overline{(e,f)}$ números inteiros quaisquer. Para a relação de ordem valem as seguintes propriedades:}\label{int-teo-relacaoOrdemPropriedades}
    \begin{enumerate}[label=(\roman*)]
        \item Reflexiva;
        \item Antissimétrica;
        \item Transitiva;
        \item Totalidade;
        \item Compatibilidade com a adição;
        \item Compatibilidade com a multiplicação.
    \end{enumerate}
\end{teo}
\begin{dem}
    \begin{enumerate}[label=(\roman*)]
        \item Reflexiva: \\
            De fato, como $a+b \leq b+a$, segue que $\overline{(a,b)} \leq \overline{(a,b)}$.
        
        \item Antissimétrica: \\
            Supondo
            
            \[ \overline{(a,b)} \leq \overline{(c,d)} \myspace\land \overline{(c,d)} \leq \overline{(a,b)} \]
            temos que 
            \[ a+d \leq b+c \myspace\land c+b \leq d+a \]
            Pela antissimetria em $\mathbb{N}$, temos que $a+d = c+b$, então $(a,b) \sim (c,d)$ e $\overline{(a,b)} = \overline{(c,d)} $.
        
        \item Transitiva: \\
            Supondo
            \begin{align*}
                \overline{(a,b)} \leq \overline{(c,d)} & \myspace\land \overline{(c,d)} \leq \overline{(e,f)},  \\ \shortintertext{temos que}
                a+d \leq b+c & \myspace\land c+f \leq d+e.  \\ \shortintertext{Logo}
                a+d+f \leq b + c + f & \myspace\land c + f + b \leq d + e + b.  \\ \shortintertext{Assim}
                a + d + f & \myspace\leq d + e + b, \\ \shortintertext{ou seja}
                a + f & \myspace\leq b + e. \\ \shortintertext{Portanto}
                \overline{(a,b)}  & \myspace\leq \overline{(e,f)}.
            \end{align*}
            
        \item Totalidade: \\
        Pela totalidade em $\mathbb{N}$, dados $a,b,c,d \in \mathbb{N}$, temos que $a+d \leq b+c \lor c+b \leq d+a$. Logo 
        $\overline{(a,b)} \leq \overline{(c,d)} \lor  \overline{(c,d)} \leq  \overline{(a,b)}$.
        % \begin{align*}
        %     \overline{(a,b)} \leq \overline{(c,d)} & \myspace\lor \overline{(c,d)} \leq \overline{(a,b)} \\ 
        %     a+d \leq b+c & \myspace\lor c+b \leq d+a   
        % \end{align*}
        % Pela totalidade em $\mathbb{N}$ temos que sempre pelo menos uma das proposições disjuntas pelo $\lor$ ocorre.
        
        \item Compatibilidade com a adição: \\
        Supondo que $\overline{(a,b)} \myspace\leq \overline{(c,d)}$, temos que $a+d \myspace\leq b+c$. 

            Se $e, f$ são também naturais, pela compatibilidade da adição em $\mathbb{N}$ temos: \\
        \begin{align*}    
           % $a+d+e+f \myspace\leq a+d+m+e+f$ \\
            a+d +e+f \leq b+c+e+f, \\ \shortintertext{ou seja,}
            a + e + d +f \myspace\leq b + f + c + e. \\ \shortintertext{Logo}
            \overline{(a+e,b+f)} \myspace\leq \overline{(c+e, d+f)}. \\ \shortintertext{Portanto}
            \overline{(a,b)} + \overline{(e,f)} \myspace\leq \overline{(c,d)} + \overline{(e,f)}.
        \end{align*}

        
        \item Compatibilidade com a multiplicação: \\
        Suponhamos $\overline{(a,b)} \leq \overline{(c,d)}$ e também $0 = \overline{(1,1)} < \overline{(e,f)}$.
        Desse modo, temos que $e > f$ pois $1+f < 1+e$. Seja então $e = f+m$ para algum $m$ natural. As linhas abaixo são equivalentes:
        \begin{center}
            $\overline{(a,b)} \leq \overline{(c,d)}$\\
            $a+d \myspace\leq b+c  $\\
            $(a+d)m \myspace\leq (b+c)m $ \\
            $(a+d)m + (a+d)f + (b+c)f \myspace\leq (b+c)m + (a+d)f + (b+c)f$ \\
            $(a+d)(f+m) + (b+c)f \myspace\leq (a+d)f + (b+c)(f+m)$ \\
            $(a+d)e + (b+c)f \myspace\leq (a+d)f + (b+c)e$ \\
            $ae+bf + cf+de \myspace\leq af+be + ce+df  $\\
            $\overline{(ae+bf, af+be)} \myspace\leq \overline{(ce+df, cf+de)}  $ \\           
            $\overline{(a,b)} \cdot \overline{(e,f)} \myspace\leq \overline{(c,d)} \cdot \overline{(e,f)}$.
        \end{center}        
        
    \end{enumerate}
\end{dem}
\begin{obs}
    A compatibilidade com a multiplicação, que acabamos de provar, também garante que se $\overline{(a,b)} \cdot \overline{(e,f)} 
    \leq \overline{(c,d)} \cdot \overline{(e,f)}$, então $\overline{(a,b)} \leq \overline{(c,d)}$. Comentário análogo vale para a compatibilidade com a adição.
\end{obs}

\begin{prop}\label{int-prop-AMenorBAMaisCIgualB}
    É válido que $\overline{(a,b)} < \overline{(c,d)}$ se, e somente se, existe um $\overline{(e,f)} > 0$ em que  $\overline{(a,b)} + \overline{(e,f)} = \overline{(c,d)}$.
\end{prop}
\begin{dem}
    Supondo $\overline{(a,b)} < \overline{(c,d)}$, temos $a+d < b+c$, ou seja, \\ 
    $a+d + m = b+c$, para algum $m$ natural. Consideremos o número $\overline{(m+1, 1)}$, vamos mostrar que ele é o número procurado. Sabemos que ele é positivo pois 
    \[ \overline{(1,1)} < \overline{(m+1, 1)} \iff 1+1 < 1+m+1. \] \\
    Agora, $\overline{(a,b)} + \overline{(m+1,1)} = \overline{(a+m+1,b+1)} = \overline{(a+m, b)}$.
    Podemos observar que $\overline{(a+m, b)} = \overline{(c,d)}$, pois $a+m+d = b+c$.

    Provemos a volta. Suponhamos que $\overline{(a,b)} + \overline{(e,f)} = \overline{(c,d)}$, para algum $\overline{(e,f)} > 0$. Temos $e=f+m$ para algum $m$ natural. Assim, ficamos com 
    \[ \overline{(a+f+m, b+f)} = \overline{(a+m, b)} = \overline{(c,d)}. \] 
    Assim, $a+m+d = b+c$, logo, $a+d < b+c$. Portanto $ \overline{(a,b)} < \overline{(c,d)}$.
\end{dem}

A próxima proposição complementa a proposição anterior.
\begin{prop}
    É válido que $\overline{(a,b)} \leq \overline{(c,d)}$ se, e somente se, existe um $\overline{(e,f)} \geq 0$ em que $\overline{(a,b)} + \overline{(e,f)} = \overline{(c,d)}$. 
\end{prop}
\begin{dem}
    Para provar a ida, separemos em dois casos:
    \begin{enumerate}[label=(\roman*)]
        \item Se $\overline{(a,b)} = \overline{(c,d)}$ então $\overline{(a,b)} + 0 = \overline{(c,d)}$.
        \item Se $\overline{(a,b)} < \overline{(c,d)}$, caímos na \Cref{int-prop-AMenorBAMaisCIgualB}.
    \end{enumerate} 
    Para provar a volta, também separaremos em dois casos:    
    \begin{enumerate}[label=(\roman*)]
        \item Se $\overline{(e,f)} = 0$, temos $\overline{(a,b)} + 0 = \overline{(c,d)}$.
        \item Se $\overline{(e,f)} > 0$, caímos na \Cref{int-prop-AMenorBAMaisCIgualB}.
    \end{enumerate} 
\end{dem}

Vamos relembrar que um número negativo, conforme \Cref{agb-def-numNegativo}, em $\mathbb{Z}$, é qualquer número menor que $0$.
\begin{ex}
    O número $\overline{(1,2)} < \overline{(1,1)} = 0$. De fato, $1+1 < 2+1$. 
\end{ex}
\begin{ex}
    O número $\overline{(4,2)}$ é positivo. Pois de acordo com a \Cref{agb-def-numPositivo}, temos que $\overline{(1,1)} < \overline{(4,2)}$ pois 
    $1+2 < 1+4$.
\end{ex}
\begin{ex}
    O número $0 = \overline{(1,1)}$ não é nem positivo, nem negativo, conforme as \Cref{agb-def-numPositivo,agb-def-numNegativo}, pois ele não é maior do que o zero do conjunto, embora ele seja maior do que ou igual.
\end{ex}
\begin{ex}
    Quando quisermos mencionar os números positivos e incluir o zero, chamaremos simplesmente de números não negativos. Analogamente, os números não positivos são os números negativos junto com o zero.
\end{ex}

\begin{prop}\label{int-prop-coordenadaMaior}
    Um número inteiro é não negativo se, e somente se, a primeira coordenada é maior do que ou igual a segunda coordenada. 
    Isto é, se $\overline{(a,b)}$ é um número inteiro, então vale $0 \leq \overline{(a,b)}  \iff b \leq a$.
\end{prop}
\begin{dem}
    Seja $0 = \overline{(1,1)}$. Tem-se que:
    \[ 0 = \overline{(1,1)} \leq \overline{(a,b)} \iff 1+b \leq 1+a \iff b \leq a. \]
\end{dem}

\begin{corol}\label{int-corol-numeroOuSimetricoPositivo}
    Dado um número inteiro, ou ele é maior do que ou igual a zero, ou seu simétrico aditivo é, ou seja,
    se $\overline{(a,b)} \in \mathbb{Z}$, então $\overline{(a,b)} \geq 0 = \overline{(1,1)}$ ou $-\overline{(a,b)} \geq 0$.
\end{corol}
\begin{dem}
    Como a relação de ordem $\leq$ é total conforme o \Cref{int-teo-relacaoOrdemPropriedades}, vale que $\overline{(a,b)} \leq \overline{(1,1)}$ ou $\overline{(1,1)} \leq \overline{(a,b)}$. Se $\overline{(a,b)}$ é não negativo é imediato que o corolário é válido. Já no segundo caso, 
    de $\overline{(a,b)} \leq \overline{(1,1)}$ temos $a+1 \leq b+1 \implies a \leq b$. \\
    Com isso, $-\overline{(a,b)} = \overline{(b,a)} \geq \overline{(1,1)}$ pela \Cref{int-prop-coordenadaMaior}.
\end{dem}

\begin{corol}\label{int-corol-simetricoSinalTrocado}
    Um número inteiro é não negativo se, e somente se, seu simétrico aditivo é não positivo.
\end{corol}
\begin{dem}
    Pelo \Cref{int-corol-numeroOuSimetricoPositivo}, considerando um número $x$ e seu simétrico $-x$, ao menos um deles é não negativo. Caso seja $0 \leq x$ temos $0 + (-x) \leq x + (-x)$. Logo $-x \leq 0$, assim $-x$ é não positivo. No outro caso supomos que $0 \leq -x$, daí 
    $0 + x \leq -x + x = 0$, assim $x \leq 0$.
\end{dem}

\begin{prop}\label{int-prop-somaPositivosPositiva}
    A soma de dois inteiros não negativos é não negativa, isto é, se $x, y \in \mathbb{Z}$ com $x,y \geq 0$ então $0 \leq x+y$.
\end{prop}
\begin{dem}
    Conforme o \Cref{int-corol-simetricoSinalTrocado}, como $x,y \geq 0$ temos $-x \leq 0$ e $-y \leq 0$. Como qualquer inteiro não negativo é maior do que ou igual a qualquer inteiro não positivo, temos $-y \leq +x$, e pela compatibilidade da soma com a relação de ordem (\Cref{int-teo-relacaoOrdemPropriedades}), temos 
    $-y + y \leq x + y$. Desse modo, $x+y \geq 0$.
\end{dem}

\begin{prop}\label{int-prop-diferencaPositiva}
    Se $x,x' \in \mathbb{Z}$ com $x' \geq x$, então existe $y$ inteiro, tal que $y \geq 0$ e $x' = x+y$.
\end{prop}
\begin{dem}
    Se $x' = x$ então $y=0$.
    Caso contrário, se $x'>x$ então considere $x = \overline{(a,b)}, x'=\overline{(a',b')}$. Temos 
    \begin{align*}
        x < x' \iff a+b' < b + a' &\iff b+a' = a+b'+c \text{ para algum } c \in \mathbb{N}. \\ \shortintertext{Logo}
        a+c+b' = b+a' &\iff (a+c,b) \sim (a',b').
    \end{align*}
    % \[ x < x' \iff a+b' < b + a' \iff b+a' = a+b'+c  \] 
    % \[  a+c+b' = b+a' \iff (a+c,b) \sim (a',b') \text{ , com $c$ natural.} \] 

    Temos também que $\overline{(a+c,b)} = \overline{(a+c+1,b+1)} = \overline{(a,b)} + \overline{(c+1, 1)}$, onde usamos a \Cref{int-def-soma} e a \Cref{int-prop-cancelaCoordenadas}. Assim $x' = \overline{(a+c,b)} = \overline{(a,b)} + \overline{(c+1,1)}$.

    O número $\overline{(c+1,1)}$ é inteiro, pois suas entradas são números naturais, e é positivo pois $\overline{(1,1)} \leq \overline{(c+1,1)} 
    \iff 1+1 \leq 1+(c+1)$ e como $c$ é natural, ele não pode ser zero, conforme nossa construção. Dessa forma $\overline{(c+1,1)}$ é o $y$ desejado.
    % O número inteiro $y = \overline{(m+1,1)}$ é tal que $y>0$ e $x' = x+ y$. Ele é positivo porque $0 = \overline{(1,1)} \leq \overline{(m+1,1)}$ uma vez que $1+1 \leq 1+(c+1)$. Ainda, $0 \neq y$ pois $c \in \mathbb{N}$ e nosso $\mathbb{N}$ não tem o zero. Agora mostremos que $x+y = x'$, temos:
    % $x+y = \overline{(a,b)} + \overline{(c+1,1)} = \overline{(a+c+1,b+1)} = \overline{(a+c,b)}$, isso usando a \Cref{int-def-soma} e \Cref{int-prop-cancelaCoordenadas}. 
    % Temos também que $ a+c+b' = b+a' \iff (a+c,b) \sim (a',b')$ 
\end{dem}

\begin{prop}\label{int-prop-diferencaPositivaII}
    Sejam $x, x'$ números inteiros quaisquer. Se existe $y$ inteiro tal que $y \geq 0$ e $x' = x+y$, então $x \leq x'$.    
\end{prop}
\begin{dem}
    Seja $x = \overline{(a,b)}$.
    Suponhamos $0 \leq y$, assim $y = \overline{(e,f)}$ para $e,f$ naturais tais que $e \geq f$ (\Cref{int-prop-coordenadaMaior}). 
    De $x' = x + y$ temos que $x' = \overline{(a+e,b+f)}$. Temos que 
    \begin{align*}
        f &\leq e \\ \shortintertext{assim}
        a+b+f &\leq b+a+e \\ \shortintertext{e então}
        a+(b+f) &\leq b+ (a+e). \\ \shortintertext{Logo}
        \overline{(a,b)} &\leq \overline{(a+e,b+f)} \\ \shortintertext{e portanto}
        x &\leq x'.
    \end{align*}
\end{dem}

\begin{prop}\label{int-prop-quadradoPositivo}
    O quadrado de um número inteiro é um inteiro não negativo, ou seja, se $\overline{(a,b)} \in \mathbb{Z}$, então vale $0 = \overline{(1,1)} \leq \overline{(a,b)} \cdot \overline{(a,b)}$. 
\end{prop}
\begin{dem}
    Temos $\overline{(a,b)} \cdot \overline{(a,b)} = \overline{(aa + bb, ab+ba)}$. Observando a \Cref{int-prop-coordenadaMaior}, queremos mostrar que $aa+bb \geq ab+ba$. Considerando a tricotomia de $\mathbb{N}$ (aplicada em $a$ e $b$) dada na \Cref{nat-teo-somaPropriedades}, separaremos a demonstração em 3 casos:
    \begin{enumerate}[label=(\roman*)]
        \item Caso $a=b$: \\
        Temos $aa+aa \geq aa+aa$.

        \item Caso $a < b$: \\
        Temos $b = a+c$, com $c \in \mathbb{N}$. Logo \\
        \[ aa+bb = aa+(a+c)(a+c) = aa+aa+ac+ac+cc \] 
        e
        \[ ab+ba = a(a+c)+(a+c)a = aa+ac+aa+ca \]
        Notemos que $aa+bb = ab+ba+cc$, com $cc \in \mathbb{N}$, assim $aa+bb \geq ab+ba$.

        \item Caso $b < a$: \\
        Temos $a = b+c$, com $c \in \mathbb{N}$. Logo
        \[ aa+bb = (b+c)(b+c) + bb = bb+bc+cb+cc+bb \]
        e
        \[ ab+ba = (b+c)b+b(b+c) = bb+cb+bb+bc. \]
        Como $aa+bb = ab+ba+cc$, com $cc \in \mathbb{N}$, provamos que $aa+bb \geq ab+ba$.
    \end{enumerate}
\end{dem}

\begin{prop}\label{int-prop-produtoPositivo}
    O produto de dois inteiros não negativos é não negativo.
\end{prop}
\begin{dem}
    Seja $r$ um número inteiro, tal que $0 \leq r$. Se $s$ é um inteiro tal que $0 \leq s$, pela compatibilidade do produto com a relação de ordem (\Cref{int-teo-relacaoOrdemPropriedades}), então $0 \cdot s \leq r \cdot s$, assim $0 \leq rs$.
\end{dem}
\begin{prop}\label{int-prop-produtoMaiores}
    Sejam $x, y, x', y'$ números inteiros não negativos, tais que $x \leq x'$ e $y \leq y'$. Vale que $xy \leq x'y'$.
\end{prop}
\begin{dem}
    Suponhamos $x' \geq x $ e $y' \geq y$. Temos que  $x' = x + r$ para algum $r$ inteiro não negativo, e que $y' = y+s$ para algum $s$ inteiro não negativo.
    Segue que $ x'y' = (x+r)(y+s) = xy+xs+ry+rs \geq 0$, pois cada uma dessas parcelas é não negativa. De fato, sendo $x,y,r,s$ todos não negativos,  o produto de quaisquer dois deles é não negativo (\Cref{int-prop-produtoPositivo}). Pela \Cref{int-prop-diferencaPositivaII} obtemos $xy \leq x'y'$.
\end{dem}

% \begin{prop}\MODIFICADOlabel{int-prop-produtoMaiores} % Foi alterado a notação, ao invés de trabalhar com pares ordenados foi trabalhado somente com uma variável em Z.
%     Sejam $\overline{(a,b)},\ \overline{(a',b')},\ \overline{(c,d)},\ \overline{(c',d')}$ números inteiros não negativos, tais que \\
%     $\overline{(a,b)} \leq \overline{(a',b')}$ e $\overline{(c,d)} \leq \overline{(c',d')}$. Vale que 
%     \[ \overline{(a,b)} \cdot \overline{(c,d)} \leq \overline{(a',b')} \cdot \overline{(c',d')}. \]
% \end{prop}
% \begin{dem}
%     Supondo $\overline{(a,b)} \leq \overline{(a,b)}$ e $\overline{(c,d)} \leq \overline{(c',d')}$ temos $a+b' \leq b+a'$ e $c+d' \leq d+c'$.
%     \[ (ac+bd) + (a'd'+b'c') \leq (ad+bc) + (a'c' + b'd') \]
%     \[ \overline{(ac+bd, ad+bc)} \leq \overline{(a'c'+b'd', a'd'+b'c')} \]
%     \[ \overline{(a,b)} \cdot \overline{(c,d)} \leq \overline{(a',b')} \cdot \overline{(c',d')}  \]
% \end{dem}

\section{Imersão de $\mathbb{N}$ em $\mathbb{Z}$}
A imersão que trataremos a seguir justificará que utilizemos apenas um símbolo para a adição, quer trabalhemos com $\mathbb{N}$, quer trabalhemos com $\mathbb{Z}$. Isso valerá também para $\mathbb{Q}$ e $\mathbb{R}$, que serão apresentados nos próximos capítulos. A ideia de imersão trata de nos permitir corresponder um conjunto num outro, no caso, $\mathbb{N}$ em $\mathbb{Z}$, e assim trabalhar como se $\mathbb{N}$ fosse um subconjunto de $\mathbb{Z}$.

\begin{teo}\label{int-teo-imersao}
    Considere a função definida abaixo:
    % $f: \mathbb{N} \rightarrow \mathbb{Z}, f(x) \mapsto \overline{(x+1, 1)}$. 
    \begin{align*}
        f \colon &\mathbb{N} \to \mathbb{Z} \\
        &x \mapsto \overline{(x+1, 1)}.
    \end{align*}
    Essa função tem as propriedades a seguir:
    \begin{enumerate}[label=(\roman*)]
        \item $f(a + b) = f(a) + f(b)$;
        \item $f(a \cdot b) = f(a) \cdot f(b)$;
        \item $a \leq b \implies f(a) \leq f(b)$.
    \end{enumerate}
\end{teo}
\begin{dem}
    \begin{enumerate}[label=(\roman*)]
        \item 
            \begin{align*}
                f(a + b) 
                &= \overline{(a+b+1, 1)} \\
                &= \overline{(a+1+b+1, 1+1)}  \\
                &= \overline{(a+1,1)} + \overline{(b+1,1)} \\
                &= f(a) + f(b).
            \end{align*}
        
        \item 
        \begin{align*}
            f(a \cdot b) &= \overline{(ab+1, 1)}\\
                        &= \overline{(ab+a+b+1+1, a+1+b+1)}\\
                        &= \overline{((a+1)(b+1)+1, (a+1)1 + 1(b+1))}\\
                        &= \overline{(a+1,1)} \cdot \overline{(b+1,1)}\\
                        &= f(a) + f(b).
        \end{align*}        
        
        \item 
        Se $a=b$ então $f(a)=f(b)$ é óbvio. Suponhamos, por outro lado, $a \neq b$.
        Então $a<b \iff b=a+m$ para algum $m \in \mathbb{N}$.
        Temos então $f(a) = \overline{(a+1,1)}$ e $f(b) = f(a+m) = \overline{(a+m+1,1)}$.
        Vemos que $\overline{(a+1,1)} \leq \overline{(a+m+1,1)}$ porque $a+1+1 \leq 1 + a + m + 1$.
    \end{enumerate}
\end{dem}



\begin{obs}
    A partir de agora, temos a liberdade de escrever os números inteiros com a notação usual de $\mathbb{N}$, isto é $1 = \overline{(2,1)}, 2 = \overline{(3,1)}, 3 = \overline{(4,1)}$ e assim por diante, além de representar o $\overline{(1,2)} = -1,\ \overline{(1,3)} = -2,\ \overline{(4,1)} = -3$ e, analogamente, ad infinitum.
\end{obs}

\begin{teo}\label{int-teo-ilimitadoSuperiormente}
    O conjunto $\mathbb{N}$ não é limitado superiormente em $\mathbb{Z}$.
\end{teo}
\begin{dem}
    O teorema deve ser entendido no contexto da imersão de $\mathbb{N}$ em $\mathbb{Z}$.
    Por contradição, vamos supor que $\overline{(a,b)} \in \mathbb{Z}$ seja um número fixo e uma cota superior de $\mathbb{N}$. Sabemos que um número natural em $\mathbb{Z}$ é da forma $\overline{(n+1,1)} \in \mathbb{Z}$, com $n \in \mathbb{N}$. Logo, devemos ter $\overline{(n+1,1)} \leq \overline{(a,b)}$ para qualquer $n \in \mathbb{N}$. Temos $n+1+b \leq 1+a$ e então
    \begin{equation}
        n+b \leq a, \label{int-dummyIlimitadoSuperiormente}     
    \end{equation}
    em que $a$ e $b$ são naturais fixos e $n$ um natural qualquer.
    
    Substituindo $n$ por $n_0$ na Inequação \ref{int-dummyIlimitadoSuperiormente}, obtemos $n_0 + b \leq a$. Se for $n_0 + b = a$, tomando $n_1 = n_0 + 1$, obtemos que 
    \[ a = n_0 + b < n_0 + 1 + b =  n_1 + b, \] 
    em que $n_1 + b \in \mathbb{N}$, o que é uma contradição. Se for 
    $n_0 + b < a$, então $a = n_0 + b + m$, para algum $m \in \mathbb{N}$. Tomando $n_1 = n_0 + m + 1$, obtemos 
    \[ a = n_0 + b + m < n_0+m+1 + b = n_1 + b, \] 
    em que $n_1 + b \in \mathbb{N}$, o que é uma contradição. Portanto, em $\mathbb{Z}$ não existe uma cota superior para $\mathbb{N}$.

\end{dem}

\begin{ex}\label{int-ex-imersaoProduto}
    Considerando a imersão, podemos entender o produto $2 \cdot \overline{(a,b)}$ da seguinte forma:
    $2 = \overline{(3,1)}$ e $\overline{(3,1)} \cdot \overline{(a,b)} = \overline{(3a+b,3b+a)} = \overline{(2a,2b)}$.
    Essa é a notação usual quando se trabalha com elementos onde a multiplicação por um escalar funciona da maneira usual 
    $k \cdot \overline{(a,b)} = \overline{(ka,kb)}$ quando $k > 0$, já quando $k < 0$ precisamos inverter as coordenadas, o que não é usual.
\end{ex}
A imersão de $\mathbb{N}$ em $\mathbb{Z}$ permite-nos interpretar $\mathbb{N}$ como um subconjunto de $\mathbb{Z}$, embora pela nossa construção fique evidente que isso não ocorra, nos termos de elementos dos conjuntos. Mas, por outro lado, as operações de adição e multiplicação e a relação de ordem funcionam analogamente em $\mathbb{Z}$, ou seja, o nosso objetivo do capítulo de construir uma adição, uma multiplicação e uma relação de ordem mantendo certa semelhança com $\mathbb{N}$ foi alcançado.

Essa é uma parte importante da construção dos números inteiros. Em geral não importa o que é um número, mas apenas o que conseguimos fazer com ele, as regras do jogo. Por outro lado, a construção do conjunto dos números inteiros mostra-nos que, a existência de um conjunto $\mathbb{N}$, a lógica e a teoria de conjuntos elementar que assumimos até agora garantem que existe um conjunto $\mathbb{Z}$ que podemos manipular com as regras mostradas.

Por certo ponto de vista, não interessa o que são e como são definidas as operações de soma e produto, mas, admitindo que exista um conjunto $A$ com uma operação de soma e uma operação de produto, ambas tendo certas propriedades, é possível provar teoremas sem levar em conta a \emph{natureza} dos entes matemáticos envolvidos.

A construção de $\mathbb{Z}$, as apresentações e provas das proposições apresentadas caminham nesse sentido de permitir uma abstração do conjunto, para que não seja mais necessário trabalhar com pares ordenados (ainda mais sem o $0$ em $\mathbb{N}$!).

Poderíamos desenvolver alguns tópicos importantes levando em conta a natureza dos nossos números inteiros, que são pares ordenados. Isso é satisfatório, se conseguirmos provar o que queremos. Por outro lado, tratar com conjuntos genéricos (independente da construção utilizada) permite uma abstração e uma generalização, mostrando que certas características não dependem de um conjunto específico (no nosso caso $\mathbb{N} \times \mathbb{N} / \sim$). As abstrações permitem formar conclusões dependendo de apenas algumas caracterísitcas/propriedades do conjunto. Dessa maneira, em geral a pergunta "qual é o conjunto de números inteiros?"\ não faz sentido, pois o que nos importa é o que podemos fazer com ele, e não como ele foi construído. Isso se tornará importante na construção de $\mathbb{R}$, podendo ser feita de dois métodos que formalizam o conjunto, mas nos interessa como $\mathbb{R}$ foi construído porque o tema deste trabalho é a construção dos números, mas na utilização do conjunto, se for feito por sequências ou por cortes de Dedekind, não é relevante. 

Tá! Mas e aí, existem muitos números e cadê o $15$, o $35$, o $-351$? A resposta é que nesse momento não possuimos um sistema de numeração, que para ser desenvolvido precisa da divisão euclidiana. O leitor interessado em verificar o sistema de numeração para $\mathbb{Z}$ pode consultar \cite{hefez-algebra}.



\end{document}