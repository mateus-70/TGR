\documentclass[../main.tex]{subfiles}
\begin{document}
\chapter{INTRODUÇÃO} % LETRAS MAIÚSCULAS


A introdução apresenta os objetivos do trabalho, bem como as razões de sua elaboração. Tem caráter didático de apresentação.
Deve abordar:

    a) o problema de pesquisa, proposto de forma clara e objetiva;
    
    b) os objetivos, delimitando o que se pretende fazer;
    
    c) a justificativa, destacando a importância do estudo;
    
    d) apresentar as definições e conceitos necessários para a compreensão do estudo;
    
    e) apresentar a forma como está estruturado o trabalho e o que contém cada uma de suas partes.
    
O desenvolvimento é a demonstração lógica de todo o trabalho, detalha a pesquisa ou o estudo realizado. Explica, discute e demonstra a pertinência das teorias utilizadas na exposição e resolução do problema. 

O desenvolvimento pode ser subdivido em seções e subseções com nomenclaturas definidas pelo autor conforme conteúdo apresentado. 
\end{document}