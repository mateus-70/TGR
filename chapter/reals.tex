\documentclass[../main.tex]{subfiles}
\begin{document}
\chapter{O CONJUNTO DOS NÚMEROS REAIS}
\label{cap-reais}
\begin{defi}
    A regra de sinais para o produto é como a seguir:
    \begin{enumerate}[label=(\roman*)]
        \item $-\left(\abs{\alpha}\abs{\beta}\right)$, se $\alpha < 0^*, \beta \geq 0^*$
        \item $-\left(\abs{\alpha}\abs{\beta}\right)$, se $\alpha \geq 0^*, \beta < 0^*$
        \item $\left(\abs{\alpha}\abs{\beta}\right)$, se $\alpha < 0^*, \beta < 0^*$
    \end{enumerate}
\end{defi}
\begin{prop}
    Sejam $\alpha$ e $\beta$ cortes. Vale que $\left( - \alpha \right) \beta = \alpha \left( -\beta \right) = -\left(\alpha \beta \right) $ e também que $\left( -\alpha \right) \left( -\beta \right) = \alpha \beta$.
\end{prop}
\begin{dem}
Consideremos quatro casos distintos, de combinações de $\alpha, \beta$ sendo maior ou menor do que $0^*$.
\begin{enumerate}
    \item $\alpha \geq 0^*$ e $\beta \geq 0^*$ \\
        $(-\alpha)\beta = -(\abs{-\alpha}\abs{\beta}) = -(\abs{\alpha}\abs{\beta})$ \\
        $\alpha(-\beta) = -(\abs{\alpha}\abs{-\beta}) = -(\abs{\alpha}\abs{\beta})$ \\
        Mas $-(\abs{\alpha}\abs{\beta}) = -(\alpha\beta)$.
        
    \item $\alpha < 0^*$ e $\beta \geq 0^*$
    
    \item $\alpha \geq 0^*$ e $\beta < 0^*$
    
    \item $\alpha < 0^*$ e $\beta < 0^*$
\end{enumerate}
\end{dem}
\end{document}