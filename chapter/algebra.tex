\documentclass[../main.tex]{subfiles}
\begin{document}
\chapter{ALGEBRA Def E teo}

\begin{defi}{Produto cartesiano}
    O produto cartesiano de X por Y é o conjunto $\{(x,y) : x \in X \land y \in Y\}$. Denotamos por $X \times Y$.
\end{defi}
\begin{obs}
    O elemento $(x,y)$ é chamado de par ordenado, e ocorre que $(a,b) = (c,d) \iff a = c \land b = d$. Fica subentendido na definição anterior que é o conjunto de todos os pares ordenados com $x \in X$ e $y \in Y$.
\end{obs}

\begin{defi}{Relação}
    Uma relação é qualquer subconjunto do produto cartesiano.
\end{defi}

\begin{teo}\label{agb-neutro-unico}
    Seja $A$ um conjunto, e seja $*$ uma operação sobre \N. Se existir um elemento $e \in A$ que é neutro, então ele é único.
\end{teo}
\begin{dem}
    Vamos considerar dois elementos neutros quaisquer, $e, e'$. 
    Temos então que $e = e * e' = e'$. A primeira igualdade se justifica porque $e'$ é neutro pela direita, a segunda igualdade porque $e$ é neutro pela esquerda.
\end{dem}

\end{document}