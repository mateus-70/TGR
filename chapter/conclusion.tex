\documentclass[../main.tex]{subfiles}
\begin{document}
\chapter*[Conclusão]{CONSIDERAÇÕES FINAIS}
\addcontentsline{toc}{chapter}{CONSIDERAÇÕES FINAIS}
% É a parte final do texto. Deve retomar o problema inicial, revendo os objetivos
% e comentando se foram atingidos ou não, enunciando as principais contribuições.
% Sintetiza as principais idéias, bem como os resultados, avaliando pontos positivos e
% negativos. Geralmente inclui recomendações e/ou sugestões. 

Ao longo deste trabalho muitos pontos importantes foram colocados em questão. Um deles era verificar se poderíamos desenvolver este trabalho sem considerar o $0$ como um número natural. Isso foi possível e não impediu o desenvolvimento dos assuntos abordados, embora tenha sido necessário fazer pequenos ajustes, mas nada substancial. 

Conseguimos estender o conceito de número a partir dos números naturais até os números reais. Mostramos também que a interpretação, a nível de ensino básico, da inclusão dos conjuntos $\mathbb{N} \subset \mathbb{Z} \subset \mathbb{Q} \subset \mathbb{R}$ é correta, por causa das imersões, e é conveniente, porque simplifica a notação para os números. 

Foram provadas muitas proposições, algumas com caráter essencial para o desenvolvimento do trabalho, tais como a completude e a não enumerabilidade de $\mathbb{R}$, e outras com caráter mais ilustrativo, tais como a enumerabilidade de $\mathbb{Z}$ e de $\mathbb{Q}$ e que a equação $r^2 = 2$ não admite solução em $\mathbb{Q}$. Como principais resultados, destacamos a demonstração de que o conjunto $\mathbb{Q}$ é um corpo ordenado, e que o conjunto $\mathbb{R}$ é o único corpo ordenado completo existente, a menos de isomorfismos.

% Este trabalho permite que sejam estudados extensões do conceito de número para além dos números reais, tais como números complexos ou octónios, ou fundamentação para os axiomas de Peano, através da teoria de conjuntos e da lógica clássica, ou ainda, será que é possível obter os axiomas de Peano a por meio de uma lógica não clássica? 
A compreensão dos temas estudados neste trabalho nos permite almejar novas pesquisas, como a extensão para além do conjunto dos números reais, tais como números complexos ou octónios. Números complexos podem ser compreendidos como pares de números reais, mas que não possuem uma relação de ordem total. Octónios por sua vez, podem ser considerados octetos de números reais cuja álgebra, baseada na álgebra dos quaternários, é não comutativa e não associativa.
Como temas de interesse para estudos futuros, destacamos a fundamentação para os axiomas de Peano, através da teoria de conjuntos e da lógica clássica e a investigação à respeito da possibilidade de obtenção dos axiomas de Peano por meio de uma lógica não clássica.

\end{document}