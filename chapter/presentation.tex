\documentclass[../main.tex]{subfiles}

\begin{document}

% ----------------------------------------------------------
% ELEMENTOS PRÉ-TEXTUAIS
% ----------------------------------------------------------

\pretextual

% ---------------------------------------------
% capa
% ---------------------------------------------
\begin{center}
\thispagestyle{empty}%página não enumerada
 \textbf{ UNIVERSIDADE DO ESTADO DE SANTA CATARINA - UDESC\\
CENTRO DE CIÊNCIAS TECNOLÓGICAS - CCT\\
CURSO DE LICENCIATURA EM MATEMÁTICA\\}
\vspace{4 cm}\textbf{\authorName}

\vspace{4 cm}\titleComplete

\vspace{\stretch{1}}
\textbf{JOINVILLE - SC}

\textbf{2023}
 \pagebreak
\end{center}
% ---


% ------------------------------------------------
% folha de rosto
% ------------------------------------------------

\begin{folhaderosto}

  \begin{center}
    \textbf{\authorName}

    \vspace*{\fill}\vspace*{\fill}
    \titleComplete
    \vspace*{\fill}
  \end{center}
  
  \begin{flushright}
  \begin{minipage}[t]{8 cm}
  { Trabalho de conclusão de curso apresentado como requisito parcial para obtenção do título de licenciado em Matemática pelo curso de Licenciatura em Matemática do Centro de Ciências Tecnológicas - CCT, da Universidade do Estado de Santa Catarina - UDESC. 

 Orientador: \orientationBy}
  \end{minipage}
  \end{flushright}
  
  \vspace{\stretch{1}}
  
  \begin{center}
   \textbf{JOINVILLE - SC}
 
   \textbf{2023} 
  \end{center}
\end{folhaderosto}
% ---



% ----------------------------------------------------
% Inserir folha de aprovação
% ----------------------------------------------------

% Isto é um exemplo de Folha de aprovação, elemento obrigatório da NBR
% 14724/2011 (seção 4.2.1.3). Você pode utilizar este modelo até a aprovação
% do trabalho. Após isso, substitua todo o conteúdo deste arquivo por uma
% imagem da página assinada pela banca com o comando abaixo:
%
% \includepdf{folhadeaprovacao_final.pdf}
% ---


% ---------------------------------------------------
% folha de aprovação
% ---------------------------------------------------
\begin{folhadeaprovacao}

  \begin{center}
    \textbf{\authorName}
\vspace {1 cm}

   \titleComplete 
   \vspace {1 cm}
  \end{center}
    
\begin{flushright}
  \begin{minipage}[t]{8 cm}
  { Trabalho de conclusão de curso apresentado como requisito parcial para obtenção do título de licenciado em Matemática pelo curso de Licenciatura em Matemática do Centro de Ciências Tecnológicas - CCT, da Universidade do Estado de Santa Catarina - UDESC. 

 Orientador: \orientationBy}
 
  \end{minipage}
  \end{flushright}
     
     \begin{minipage}[c]{3cm} 
	Membros:
\end{minipage}
\begin{minipage}[c]{8 cm}
	\begin{center}
	\textbf{BANCA EXAMINADORA}
	\vspace {2 cm}
	
    \orientationBy
	
    Nome da Instituição
    
    \vspace {1.5 cm}
    
    Nome do Membro da banca e Titulação
	
    Nome da Instituição
    
    \vspace {1.5 cm}
    
    Nome do Membro da banca e Titulação
	
    Nome da Instituição
    \vspace {1.5 cm}
    
    
\end{center}

\end{minipage}

\vspace*{\fill}
     \begin{center}
	     Joinville, \today.
\end{center}

    
\end{folhadeaprovacao}
% ---



% ---------------------------------------
% Dedicatória
% ---------------------------------------
\vspace*{\fill}

\begin{dedicatoria}
   \vspace*{10 cm}
   \begin{flushright}
   \begin{minipage}[t]{7cm}
     À minha mãe
   \end{minipage}
   \end{flushright}
\end{dedicatoria}
% ---


% ---------------------------------------
% Agradecimentos
% ---------------------------------------
\begin{center}
	\textbf{AGRADECIMENTOS}
\end{center}

%Elemento opcional utilizado pelo autor para registrar agradecimento às pessoas que contribuíram para a elaboração do trabalho.
Agradeço à minha mãe, sem ela não teria sido possível concluir este curso. Agradeço também a todos que ajudaram direta ou indiretamente na elaboração do trabalho, aos desenvolvedores das ferramentas utilizadas na elaboração do trabalho, em especial aos contribuintes do \LaTeX, do Overleaf, e da comunidade Debian.

\newpage
% ---


% -----------------------------------------
% Epígrafe
% -----------------------------------------
\vspace*{\fill}

\begin{epigrafe}
    \vspace*{10cm}
	\begin{flushright}
	  \begin{minipage}[t]{7cm}
		\textit{I dunno about this}
	 \end{minipage}
	\end{flushright}
\end{epigrafe}
% ---


% -----------------------------------------
% RESUMOS
% -----------------------------------------
% resumo em português
\begin{center}
	\textbf{RESUMO}
\end{center}



  \vspace{\onelineskip}

\noindent Elemento obrigatório que contém a apresentação concisa dos pontos relevantes do trabalho, fornecendo uma visão rápida e clara do conteúdo e das conclusões do mesmo. A apresentação e a redação do resumo devem seguir os requisitos estipulados pela NBR 6028 (ABNT, 2003). Deve descrever de forma clara e sintética a natureza do trabalho, o objetivo, o método, os resultados e as conclusões, visando fornecer elementos para o leitor decidir sobre a consulta do trabalho no todo.

 \vspace{\onelineskip}
    
 \noindent\textbf{Palavras-chave:} Número. Conjunto. Construção. Teorema.

\newpage


% resumo em inglês
 \begin{otherlanguage*}{english}

\begin{center}
	\textbf{ABSTRACT}
\end{center}



 \vspace{\onelineskip}

\noindent Elemento obrigatório para todos os trabalhos de conclusão de curso. Opcional para os demais trabalhos acadêmicos, inclusive para artigo científico. Constitui a versão do resumo em português para um idioma de divulgação internacional. Deve aparecer em página distinta e seguindo a mesma formatação do resumo em português.
 \vspace{\onelineskip}

\noindent\textbf{Keywords:} Keyword 1. Keyword 2. Keyword 3. Keyword 4. Keyword 5.

 \end{otherlanguage*}

\newpage


% ---


% -----------------------------------------
% inserir lista de ilustrações
% -----------------------------------------
\renewcommand\listfigurename{{\fontsize{12pt}{\baselineskip}\normalfont \bfseries LISTA DE ILUSTRAÇÕES}}
\pdfbookmark[0]{\listfigurename}{lof}
\listoffigures*
\cleardoublepage
% ---


% -----------------------------------------
% inserir lista de tabelas
% -----------------------------------------
\renewcommand\listtablename{{\fontsize{12pt}{\baselineskip}\normalfont \bfseries LISTA DE TABELAS}}
\pdfbookmark[0]{\listtablename}{lot}
\listoftables*
\cleardoublepage
% ---



% ----------------------------------------
% inserir lista de abreviaturas e siglas
% ----------------------------------------
\renewcommand\listadesiglasname{{\fontsize{12pt}{\baselineskip}\normalfont \bfseries LISTA DE ABREVIATURAS E SIGLAS}}
\begin{siglas}
  \item[IMPA]  Insituto de Matemática Pura e Aplicada 
  \item[SBM]  Sociedade Brasileira de Matemática

  \end{siglas}
% ---


% ----------------------------------------
% inserir lista de símbolos
% ----------------------------------------
\renewcommand\listadesimbolosname{{\fontsize{12pt}{\baselineskip}\normalfont \bfseries LISTA DE SÍMBOLOS}}
\begin{simbolos}
  \item[$ \mathbb{N} $] Conjunto dos números naturais (sem o zero)
  \item[$ \mathbb{Z} $] Conjunto dos números inteiros 
  \item[$ \mathbb{Q} $] Conjunto dos números racionais 
  \item[$ \mathbb{R} $] Conjunto dos números reais
  \item[$\mathbb{A^*}$] Conjunto $A \setminus \{0\}$ (retira o zero), sendo A um conjunto qualquer
  \item[$\defeq$] O que está à esquerda do símbolo é, por definição, igual ao que está à direita
  \item[$\land$] Conjunção (E) das proposições, uma à esquerda e outra a direita. 
  \item[$\lor$] Disjunção (OU) das proposições, uma à esquerda e outra a direita. 
  \item[$\lor$] Disjunção (OU) das proposições, uma à esquerda e outra a direita. 
  \item[$\subset$] O conjunto à esquerda é subconjunto do conjunto da direita. 
  \item[$\exists!x$] Existe um único $x$ ...
  
 %\item[$ \mathbb{C} $] Conjunto dos números complexos
 %\item[$ C_c(R) $] Conjunto das funções $f: R \rightarrow \C$ contínuas com suporte compacto em um conjunto $R$.
\end{simbolos}
% ---


% ----------------------------------------
% inserir o sumario
% ----------------------------------------
\renewcommand\contentsname{{\fontsize{12pt}{\baselineskip}\normalfont \bfseries SUMÁRIO}}
\pdfbookmark[0]{\contentsname}{toc}
\tableofcontents*
\cleardoublepage
% ---

\end{document}