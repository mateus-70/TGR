\documentclass[../main.tex]{subfiles}
\begin{document}
\chapter{O CONJUNTO DOS NÚMEROS NATURAIS E OS AXIOMAS DE PEANO}
Neste capítulo apresentaremos a formalização aritmética do conjunto dos números naturais. Primeiro será apresentada uma breve parte histórica e sua importância, após isso, enunciaremos os axiomas de Peano. Depois definiremos uma adição, uma multiplicação e uma relação de ordem, bem como listaremos algumas propriedades básicas e faremos suas demonstrações. As referências básicas desse capítulo são \textcite{domingues-2009} e \textcite{ferreira}.

\section{Um pouco de história}

Uma possível narrativa para o início do trabalho com números seria a necessidade de pastores efetuarem a contagem de animais em seus rebanhos, embora essa narrativa não seja definitiva \Cite{roque}. Essa iniciação com números e o conhecimento obtido pode ter encontrado dois caminhos (ou fins) possíveis, foram mantidos ao longo do tempo ou não. Nesse sentido não é possível estabelecer uma matemática definitiva e uma única evolução \Cite[p. 35]{roque}. 

Ao longo da históra, foram desenvolvidos muitos conjuntos numéricos, para resolver problemas que os conjuntos anteriores não eram adequados. 
Com o avanço da matemática, acabou-se esbarrando em problemas que as crenças e técnicas da época não conseguiram resolver, assim uma formalização era necessária, mesmo que não fosse com objetivo de formalizar, mas resolver esses problemas \textcite[p. 407]{roque}

Quem fundamentou a aritmética como conhecemos hoje foi Giuseppe Peano, mas houve tentativas anteriores, podemos citar Frege. 
Peano conseguiu entre outros fatores por boas escolhas de símbolos, muitos dos quais usamos atualmente, e também por causa da explicitação das regras através de símbolos, e a ausência de hipóteses ocultas \textcite[p. 415]{boyer}

O trabalho de Peano ficou conhecido como Axiomas de Peano. Ele fundamentou a sua aritmética com 9 axiomas e com 3 conceitos primitivos, como está apresentado na \Cref{fig:axiomas-peano}.
\todo{ Referenciar ABNT a figura}
\begin{figure}
    \begin{center}
        \includegraphics{../include/peano_axioms_original}
        \caption{A versão original dos axiomas de Peano}\label{fig:axiomas-peano}
    \end{center}  
\end{figure}

Antes de continuarmos, podemos recolocar uma analogia da matemática com um jogo:

\begin{displayquote} Ao cirar um jogo, é importante que suas regras sejam suficientes e consistentes. Por \emph{suficiente} queremos dizer que as regras devem estabelecer o que é permitido fazer em qualquer situação que possa vir a ocorrer no desenrolar de uma partida do jogo. Por \emph{consistente} queremos dizer que as regras não devem contradizer-se, ou sua aplicação levar a situações contraditórias. \Parencite[p. 13-14]{barbosa}
\end{displayquote}

Para nós, as peças do jogo serão o conceito de um, o conceito de número natural, e o conceito da relação sucessor. As regras do jogo, naturalmente serão os axiomas que relacionarão esses 3 conceitos primitivos. Nessa analogia, as peças não são muito importantes.

\section{Axiomas}
O nosso estudo dos números naturais será iniciado pela sua apresentação em 3 conceitos primitivos, como os de Peano, e 5 axiomas (ao invés de 9), que são:

\begin{axi}\label{axi-existe-n-s}
    Existe um conjunto de exatamente todos os números naturais, que será denotado por $\mathbb{N}$, e existe uma função $s: \mathbb{N} \rightarrow \mathbb{N}$, que é a relação "sucessor". 
\end{axi}
\begin{axi}\label{axi-um-natural}
    Um é um número natural, isto é, $1 \in \mathbb{N}$.
\end{axi}
\begin{axi}\label{axi-um-nao-sucessor}
    Um não é sucessor de nenhum número, isto é, $1 \not \in Im(s)$ ou ainda, $\not \exists a \in \mathbb{N} : s(a) = 1$.
\end{axi}
\begin{axi}\label{axi-s-injetora}
    $s$ é injetora, isto é, $s(a) = s(b) \implies a = b$ \footnote{Vale notar a contra-positiva que estabelece, nesse caso: $a \neq b \implies s(a) \neq s(b)$}.
\end{axi}
\begin{axi}\label{axi-ind-finita}
    Se $\S$ é um subconjunto de $\mathbb{N}$, caso $1 \in \S$ e se para todo $k$ em $\S\ s(k)$ também esteja em $\S$, então $\S = \mathbb{N}$, isso é o mesmo que colocar: \\
     \[ \S \subseteq \mathbb{N} \land 1 \in S \land ( k \in \S \implies s(k) \in \S) \implies \S = \mathbb{N} .\]
\end{axi}
Este último axioma é chamado de axioma da indução finita.

Conforme os axiomas aprsentados, deve ser notado que o conjunto $\mathbb{N}$ (na nossa axiomatização) não tem o $0$ (zero) que é, usualmente, o neutro da soma em $\mathbb{N}$. O intuito de construir a partir do $1$ é pela questão que às vezes surge em diversas situações: "$0$ é um número natural?". Em especial, o próprio \textcite{lima-site} diz que o zero pode ser ou pode não ser. É dito que fica a critério de conveniência, embora para nós, seja 'inconveniente' perder o neutro da soma (que aparecerá sua primeira vez em $\mathbb{Z}$). 

Justificamos essa escolha pois essa dificuldade na ausência do zero terá algumas consequências de reajustes, que poderão ser observadas na nossa construção. Além disso, a bibliografia principal trata o zero como um número natural, então a construção será necessariamente com essas adaptações, o que na visão do autor, é algo positivo para o desenvolvimento do trabalho, mas o resultado final em certo ponto de vista fica prejudicado, pois fica mais "remendado".

Observemos também que o próprio Peano começou originalmente pelo $1$ e somente em trabalho posterior colocou o $0$ como o primeiro número natural.

Para comerçarmos nosso desenvolvimento, podemos notar que o \Cref{axi-um-natural} garante $\mathbb{N} \neq \emptyset $. Além dele, o \Cref{axi-existe-n-s} garante que $s(1) \in \mathbb{N}$, também $s(s(1)) \in \mathbb{N}$, $s(s(1)) \in \mathbb{N}$ e assim por diante.

Em seguida apresentamos um lema que será necessário para ampliação de nosso ferramental inicial.
\begin{lema}\label{n-dif-sucessor}
    Nenhum número natural é seu próprio sucessor, ou seja, $a \in \mathbb{N} \implies a \neq s(a) $.
\end{lema}
\begin{dem}
    Faremos por indução.\\
    Seja $\S = \{x \in \mathbb{N} : x \neq s(x) \}$
    Os axiomas \ref{axi-um-natural} e \ref{axi-um-nao-sucessor} garantem que $1 \in \S$.\\
    Supondo que existe $ k \in \S$ onde $ k \neq s(k)$, queremos provar que o sucessor de $k$ é diferente do sucessor do sucessor de $k$, isto é, $s(k) \neq s(s(k))$. Como $k \neq s(k)$ e $s$ é injetora pelo \Cref{axi-s-injetora}, concluímos que $s(k) \neq s(s(k))$.
\end{dem}
\begin{teo}\label{nat-suc-unico}
    Todo número natural, exceto o $1$ é sucessor de algum outro número natural, que é único.
    % , isto é, $\forall x (x \in \mathbb{N} \land x \neq 1 \implies \exists! y : s(y) = x)$.
\end{teo}
\begin{dem}
    A prova é feita por indução. \\ 
    Seja $\S = \{x \in \mathbb{N} : x \neq 1 \implies (\exists y \in \mathbb{N} \land s(y) = x) \}$.\\
    Sabemos que $1 \in \S$ porque $x = 1$ torna o antecedente falso, assim a implicação é verdadeira. Suponhamos que existe um $k$ em \S, queremos mostrar que $s(k) \in \S$. Devemos observar que $s(k)$ é o sucessor de $k$, assim, colocando na forma que usamos na definição do conjunto,
    $y = k$, $s(y) = s(k)$, dessa forma $s(k) \in \mathbb{N}$. Pelo \Cref{axi-ind-finita} temos que \S = $\mathbb{N}$.

    A unicidade do $y$ que usamos na construção de \S advém do \Cref{axi-s-injetora}, $s(z) = s(y) = x \implies z = y$ pela injetividade de $s$.
\end{dem}
\begin{obs}
    Diremos que o número $y$ do teorema anterior é antecessor de $x = s(y)$. Além disso vale notar que $s$ conforme definida no \Cref{axi-existe-n-s} não é sobrejetora somente pelo fato de o contradomínio ser $\mathbb{N}$. Mas se considerarmos o caso restrito para $\mathbb{N} \setminus \{1\}$ 
    teremos que $s$ é uma \emph{bijeção}.
\end{obs}

Faremos a notação usual para os números naturais, isto é, $\mathbb{N} = \{ 1, 2, 3, 4, ...\}$, onde $2 \defeq s(1), 3 \defeq s(2), 4 \defeq s(3)$ e assim por diante.

%%% ADICAO %%%


\section{A adição}
A adição em $\mathbb{N}$ é a que mais intuitivamente está relacionada com a contagem e união de coisas discretas.
\begin{defi}\label{def-adicao-N}
Sejam $a, b \in \mathbb{N}$. A adição entre $a$ e $b$, denotada por $a + b$ é definida com as seguintes condições: 
    \begin{enumerate}[label=(\roman*)]
        \item $a + 1 = s(a)$;
        \item $a + s(b) = s(a+b)$.
    \end{enumerate}
\end{defi}
Às vezes poderemos fazer referência à operação de adição com o nome de soma.

\begin{teo}
    A adição acima é uma função, que leva uma dupla de números naturais em um único número natural.
\end{teo}
\begin{dem}
    A prova de que a soma está bem definida para quaisquer $a,b \in \mathbb{N}$ é dada a seguir:
    Dado $a \in \mathbb{N}$ fixo, seja $\S = \{ x \in \mathbb{N} : a + x \ \text{está definido} \}$. Temos que $a + 1 = s(a)$ está definido, portanto $1$ está em \S. 
    Agora consideremos que vale para um dado $k \in \mathbb{N}$, que $a+k$ esteja bem definida. Temos então que $a + s(k) = s(a+k)$, como $a+k$ está definido, o sucessor também está definido, pelo axioma da indução, temos que \S = $\mathbb{N}$.
\end{dem}
\begin{lema}\label{soma-n-um-comut}
    O $1$ comuta com qualquer número na soma, isto é,$ \forall x \in \mathbb{N}, x + 1 = 1 + x$
\end{lema}
%       Template
%        Seja $\S = \{ x \in \mathbb{N} : \}$
\begin{dem}
    Seja $\S = \{ x \in \mathbb{N} : x + 1 = 1 + x\}$. O $1$ está em \S pois $1 + 1 = 1 + 1$.
    sabendo que existe um $k$ em \S, queremos provar que $s(k)$ também está em \S. Como
    $s(k) + 1 = s(s(k)) = s(k+1) = s(1+k) = 1 + s(k)$. Portanto $s(k)$ também está em \S sempre que $k$ também está, o que pelo princípio de indução \S = $\mathbb{N}$.
\end{dem} \\

\begin{prop}{A adição no conjunto dos números naturais tem as seguintes propriedades:}\label{nat-soma-props}
    \begin{enumerate}[label=(\roman*)]
        \item Fechamento;
    	\item Associativa;
    	\item Comutativa;
        \item Lei do cancelamento;
        \item Se $a,b \in \mathbb{N}$, temos $a + b \neq a$;\label{nat-soma-props-distinto}
    	\item Inexistência de neutro: $\not\exists \mathfrak{e} \in \mathbb{N} : \forall a, a + \mathfrak{e} = \mathfrak{e} + a = a$.
    \end{enumerate}
\end{prop}

\begin{dem}
    Antes de demonstrarmos propriamente, façamos a suposição que $a,b,c \in \mathbb{N}$ são números fixos.
    \begin{enumerate}[label=(\roman*)]    
        \item Fechamento: \\
            Seja $\S = \{ x \in \mathbb{N} : a + x \in \mathbb{N} \}$. Obviamente o $1$ está em \S. Suponhamos então que existe $k \in \S$, queremos saber se isso garante que $s(k) \in \S$. Temos então que $a + k \in \mathbb{N}$, já para $a + s(k) = s(a + k) \in \S$, pois pelo \Cref{axi-existe-n-s} a função tem domínio e contradomínio $\mathbb{N}$.
        \item Associativa: \\
             Seja $\S = \{ x \in \mathbb{N} : (a + b) + x = a + (b + x) \}$. Para o $1$ temos então
             $(a + b) + 1 = s(a + b) = a + s(b) = a + (b + 1)$, o que mostra que o $1$ está em \S. 
             Mostraremos que $k \in \S \implies s(k) \in \S$, pois:
             $(a + b) + s(k) = s( (a + b) + k) = s(a + (b + k)) = a + s(b + k) = a + (b + s(k))$, o que, pelo princípio de indução \S = $\mathbb{N}$.
        \item Comutativa: \\
            Consideremos o conjunto $\S = \{ x \in \mathbb{N} : a + x = x + a\}$. O $1 \in \S$ pois 
            $a + 1 = 1 + a$ conforme o \Cref{soma-n-um-comut}.
            Provemos então que se $k \in \S$ então $s(k) \in \S$, temos 
            $a + s(k) = s(a + k) = s(k + a) = k + s(a) = k + a + 1 = k + 1 + a = s(k) + a$.
        \item Lei do cancelamento: \\
            Seja $\S = \{ x \in \mathbb{N} : x + b = x + c \implies b = c \}$, vamos provar que \S = $\mathbb{N}$. Obviamente o $1$ está em \S pois $1 + b = 1 + c \iff b + 1 = c + 1$ e pelo \Cref{axi-s-injetora}, se dois elementos tem sucessores iguais, eles próprios são iguais. Agora consideremos se vale que $(k + b = k + c \implies b = c) \implies (s(k) + b = s(k) + c \implies b = c)$. \\
            $s(k) + b = s(k) + c \iff (k + b) + 1 = (k + c) + 1$, o que pelo mesmo motivo anterior, concluímos que $k + b = k + c$, e pela nossa hipótese concluímos que $b = c$.
        \item $a + b \neq a$;\\
            Consideremos quatro casos:
            \begin{enumerate}[label=(\arabic*)]
            \item $a=1 \land b=1$ \\
                Temos $a+b=a \iff 1+1=1 \absurd$. 
            \item $a=s(x) \land b=1$ \\
                Temos $a+b=a \iff s(x) + 1 = s(x) \iff s(s(x)) = s(x) \absurd$. 
            \item $a=1 \land b=s(y)$ \\
                Temos $a+b=a \iff 1+ s(y) = 1 \iff s(1+y) = 1 \absurd$. 
            \item $a=s(x) \land b=s(y)$ \\
                Temos $a+b=a \iff s(x) + s(y) = s(x) \iff x + 1 + y + 1 = x + 1 \iff s(1 + y) = 1 \absurd$.
            \end{enumerate}
    
        \item Inexistência de neutro: \\
            Como consequência da demonstração anterior, temos que não existe neutro na adição, isto é, $a + \mathfrak{e} = \mathfrak{e} + a = a$.
    \end{enumerate}
\end{dem}
 
Podemos observar que como não existe neutro, não poderemos aplicar descuidadamente as outras propriedades, por exemplo: não poderíamos ter provado que $a \neq a + 1 $ usando a lei do corte e supondo que fossem iguais, assim: $a = a+ 1 \implies 0 = 1$, porque isso não faz sentido no nosso desenvolvimento. Também podemos observar que a lei do cancelamento é independente da \Cref{agb-prop-leiCancelamento}, uma vez que não temos neutro.

%%% MULTIPLICACAO %%%
A multiplicação toma seu lugar como uma operação de redução da soma. Essa ideia de notação resumida pode ser explorada quando se faz o uso do sistema posicional, mas não faremos o uso do sistema posicional. 

\section{A multiplicação}
\begin{defi}\label{nat-def-multiplicacao}
    Sejam $a, b \in \mathbb{N}$. A multiplicação entre $a$ e $b$, denotada por $a \cdot b$ é definida com as seguintes condições: 
	\begin{enumerate}[label=(\roman*)]
		\item $a \cdot 1 = a$;
		\item $a \cdot s(b) = a \cdot b + a$.
	\end{enumerate}
\end{defi}

\begin{prop}{Para a multiplicação de números naturais são válidas as seguintes propriedades:}
    \begin{enumerate}[label=(\roman*)]
        \item Fechamento;
        \item Elemento neutro; \footnote{Se for observada a definição de multiplicação, esse número pode ser o $1$}.
        \item Distributiva;
        \item Comutativa;
        \item Associativa;
        \item Lei do cancelamento;
        \item Para $a,b \in N$ ocorre que, $a \cdot b = 1 \implies a = b = 1$.
    \end{enumerate}
\end{prop}
% template
% Seja $\S = \{ x \in \mathbb{N} : \}$
\begin{dem}
    Antes de demonstrarmos propriamente, façamos a suposição que $a,b,c \in \mathbb{N}$ são números fixos. Provaremos por indução, exceto no último item.
    \begin{enumerate}[label=(\roman*)]
        \item Fechamento: \\
            Seja $\S = \{x \in \mathbb{N} : ax \in \mathbb{N} \}$.
            O $1$ está em \S pois $1 \cdot 1 = 1$. Supondo que é válido $ak \in \mathbb{N}$ para algum $k \in \mathbb{N}$, temos $a \cdot s(k) = ak + a$, como tanto $ak$ quanto $a$ são números naturais, e como a soma é fechada em $\mathbb{N}$, temos $ak + a \in \mathbb{N}$. Pelo \Cref{axi-ind-finita}, \S = $\mathbb{N}$. 
        \item Elemento neutro: \\
            Seja $\S = \{ x \in \mathbb{N} : 1 \cdot x = x \cdot 1  = x \}$. O $1$ está em \S pois $1 \cdot 1 = 1 \cdot 1 = 1$. Supondo que $k \in \S$ podemos concluir que $s(k) \in \S$?
            Temos $1 \cdot s(k) = 1 \cdot k + 1 = k \cdot 1 + 1 = k + 1 = s(k) = s(k) \cdot 1$. Pelo \Cref{axi-ind-finita}, \S = $\mathbb{N}$.
            
            % \\ Além disso, sejam $e_1, e_2$ dois elementos neutros. Então $e_1 = e_1 \cdot e_2 = e_2 \implies e_1 = e_2$, portanto o neutro é único. Além disso a definição de multiplicação nos diz que é o $1$ quem é o neutro. 
        \item Distributiva à direita: 
            Seja $\S = \{ x \in \mathbb{N} : ( a + b ) x = ax + bx \}$. Temos que o $1 \in \S$ pois $( a + b ) 1 = a + b = a \cdot 1 + b \cdot 1$. Provemos que se $k \in \S$ temos que $s(k) \in \S$. Temos então: \\
            $( a + b ) \cdot s(k)= ( a + b ) k + ( a + b ) = ak + bk + a + b = ak + a + bk + b 
            = a \cdot s(k) + b \cdot s(k)$. Pelo \Cref{axi-ind-finita} tem-se que \S = $\mathbb{N}$.
            \\ \\
            Distributiva à esquerda: \\
            A prova da distributiva à esquerda é facilmente obtida quando já dispusermos da propriedade comutativa. Por sua vez a prova da comutatividade que faremos precisará apenas da distributiva à direita.
            % Seja $\S = \{ x \in \mathbb{N} : x (a+b) = xa + xb\} $. O $1 \in \S$ pois ele é o neutro, que comuta com qualquer número, assim temos 
            % $1 \cdot (a+b) = (a+b) \cdot 1 = a + b = a \cdot 1 + b \cdot 1 = 1 \cdot a + 1 \cdot b$. Supondo que vale para algum $k \in \mathbb{N}$, que $k(a+b) = ka + kb$, teremos então que $s(k) \cdot (a+b) = (k+1) \cdot (a+b) = k(a+b) + 1(a+b) = (ka + kb) + (a+b) = ka + a + kb + b = 
            % a \cdot s(k) + b \cdot s(k)$, e pelo \Cref{axi-ind-finita} concluímos que \S = \mathbb{N}.
        \item Comutativa: \\
            Seja $\S = \{ x \in \mathbb{N} : ax = xa \}$. Com certeza o $1$ está em \S, porque $1$ é o neutro da multiplicação. Agora suponhamos que $k \in \S$, vejamos se $s(k) \in \S$. Temos então: $a \cdot s(k) = ak + a = 1 \cdot a + ka = (1+k)a = s(k) \cdot a$. Pelo \Cref{axi-ind-finita} tem-se que \S = $\mathbb{N}$.

        \item Associativa:  \\
            Seja $\S = \{ x \in \mathbb{N} : a(bx) = (ab)x \}$. Sabemos que o $1$ está em \S pois $a (b \cdot 1) = a(b) = ab = (ab)\cdot 1$.
            Agora suponhamos que exista um $k \in \S$, assim, podemos omitir parênteses em $abx$ , consideremos $s(k)$, para ver se ele está ou não em \S. Temos
            $a(b \cdot s(k)) = a(bk + b) = abk + ab = s(k) \cdot  (ab)$, o que pelo \Cref{axi-ind-finita} tem-se que \S = $\mathbb{N}$. 
            
            
            
        \item Lei do cancelamento: \\
            Seja $\S = \{ x \in \mathbb{N} : xb = xc \implies b = c\}$. O $1$ está em \S pois $1b = 1c \implies b = c$. Provemos que $k\in \S \implies s(k) \in \S$. Temos que $s(k) \cdot b = s(k) \cdot c \iff b + bk = c + ck \iff b = c$. Pelo \Cref{axi-ind-finita} tem-se que \S = $\mathbb{N}$.
        \item $a \cdot b = 1 \implies a = b = 1$.
            Consideremos que o $a$ ou o $b$ podem ser igual (ou iguais) a $1$.
            Sem perda de generalidade, seja $a = 1$. Temos que $1 \cdot b = 1 \implies b = 1$ pois $1$ é o elemento neutro. Portanto se $a$ ou se $b$ forem $1$, obrigatoriamente o outro deverá ser, para que a igualdade ocorra.
            Consideremos um último caso, se $a \neq 1$ e $ b \neq 1$. Então existem $c$ e $d$ naturais tais que $a = s(c)$ e $b = s(d)$.
            Assim, $ab=1 \iff (c+1) (d+1) = 1 \implies cd + c + d + 1 = 1 \implies s(cd + c + d) = 1$, o que obviamente não pode ocorrer de acordo com o  \Cref{axi-um-nao-sucessor}.
    \end{enumerate}
\end{dem}
% Seja $\S = \{ x \in \mathbb{N} : \}$

Nesse desenvolvimento, as propriedades da multiplicação são semelhantes às que teríamos se tivéssemos tomado o $0$ como número natural. Uma propriedade que essa multiplicação não tem agora é o anulamento, que carece de significado. As propriedades associativa e comutativa já eram presentes na soma. Agora no produto, temos um elemento neutro que não há para a soma, além da distributiva que pode ser aplicada com soma e produto.
Na nossa lei do cancelamento, não precisamos especificar que o termo a ser cancelado é diferente de zero.

%%% A RELACAO DE ORDEM %%%

Até agora dispomos de uma soma e um produto em $\mathbb{N}$, mas isso ainda não nos possibilita comparar dois elementos de $\mathbb{N}$. Agora, com intuito de responder à pergunta, qual número vem "antes" ou qual número é "menor", devemos estabelecer uma relação de ordem. 

Reforçamos que existem diferentes relações de ordem num mesmo conjunto, conforme o exemplo que segue à \Cref{agb-def-relOrd}.

\section{A relação de ordem}
\begin{defi}\label{def-relOrdem-N}
Sejam $a, b \in \mathbb{N}$. Definiremos a relação $\leq$ entre $a$ e $b$, denotado por $a \leq b$, e diremos que $a$ se relaciona com $b$ através de '$\leq$' quando uma das seguintes situações ocorre:
    \begin{itemize}
        \item $a = b$;
        \item $a + n = b, n \in \mathbb{N}$.
    \end{itemize} 
\end{defi}
\begin{obs}
    Deve ser notado que quando $a \leq b$ uma e apenas uma das seguintes situações pode ocorrer $a = b$ ou $a + n = b, n \in \mathbb{N}$. Isso é devido à \Cref{nat-soma-props} item \ref{nat-soma-props-distinto}, que mostra que ambas não podem ocorrer simultaneamente. Já considerando que pelo menos uma situação deve ocorrer, é a totalidade, que será provada no final deste capítulo. 
\end{obs}
\begin{prop}{A relação de ordem em $\mathbb{N}$ tem as seguintes propriedades:}
    \begin{enumerate}[label=(\roman*)]
        \item Reflexiva;
        \item Antissimétrica;
        \item Transitiva;
        \item Tricotomia;
        \item Compatível com adição;
        \item Compatível com multiplicação.
    \end{enumerate}
\end{prop}
\begin{dem}
    Primeiro, suponhamos que $a,b,c,m,n$ são números naturais.
    \begin{enumerate}[label=(\roman*)]
        \item Reflexiva: \\
            É imediato que $a = a \implies a \leq a$.
        \item Antissimétrica, $a \leq b \land b \leq a \implies a=b$: \\
            Se $a=b$ não há nada a provar. Consideremos que sejam diferentes.
            Então $a \leq b \iff b = a + n$ e também, como $b \leq a \iff a = b + m$, substituindo, temos $b = (b+m) + n \implies b = b+r$ para algum $r$ natural, o que não pode ocorrer. % como demonstrado no item XXX etc.
        \item Transitiva $a \leq b \land b \leq c \implies a \leq c$: 
            Vamos considerar 4 casos:
            \begin{enumerate}[label=(\arabic*)]
                \item $a = b = c$ \\
                    Temos $a = c \implies a \leq c$\\
                \item $a = b < c$ \\
                    Temos $ b + n = c \implies a + n = c\implies a \leq c$\\
                \item $a < b = c$ \\
                    Temos $a + n = b = c \implies a \leq c$\\
                \item $a < b < c$ \\
                    Temos $a + m = b \land b + n = c \implies ( a + m ) + n = c \implies a \leq c$
            \end{enumerate}
        \item Tricotomia:
        Vamos provar por indução. \\
        Seja $\S = \{ x \in \mathbb{N} : x = a \lor x < a \lor x > a \}$. Sabemos que o $1 \in \S$ porque, ou ocorre que $a = 1$, ou $a = s(m) = m + 1$, para algum $m$ natural, assim, $1 < a$. \\
        Supondo que vale para algum $k$ que $k = a \lor k < a \lor k > a$, consideremos 3 casos.  \\
        
        No primeiro caso, $k = a$, assim $s(k) > a$. \\
        No segundo caso, $k < a$, assim $k + m = a$, para algum $m$ natural. Se $m = 1$, $k+1 = s(k) = a$. Se por outro lado, $m \neq 1$, então $m = s(n)$ para algum $n$ natural, assim, $a = k + m = k + n + 1 = s(k) + n$, dessa forma $a > s(k)$. \\
        No terceiro caso, $k > a$, assim $k = a + m$ para algum $m$ natural, e $s(k) = a + m + 1 > a$.

        A unicidade, embora não esteja explícita na criação do conjunto, pode ser vista no desenvolvimento de cada caso.
        Portanto, em todos os casos, $s(k) \in \S$, e pelo \Cref{axi-ind-finita}, \S = $\mathbb{N}$.
        
        \item Compatível com adição: \\
        Seja $a \leq b$. Se $a = b$ teremos que $a+c = b+c$ porque a adição é uma função! 
        Consideremos agora $a < b$. Então $b = a + m \implies b+c = a+m+c \implies a+c < b+c$.
        \item Compatível com multiplicação: \\
        É análogo ao caso da compatibilidade com a adição.
     \end{enumerate}
\end{dem}

\begin{corol}\label{nat-coro-tric}
    A relação de ordem $\leq$ definida anteriormente é total. \footnote{Basta observar a tricotomia}
\end{corol}

\begin{prop}
    Se $a,b,c \in \mathbb{N}$ e $a + c \leq b + c$ então $a \leq b$.
\end{prop}
\begin{dem}
    Se $a + c = b + c$, pela lei do cancelamento para a adição temos $a=b \therefore a \leq b$.
    Se $a + c + m = b + c$, para algum $m$ natural, novamente pelo cancelamento da adição temos $a+m = b$, assim $a < b \therefore a \leq b$.
\end{dem}
\begin{prop}
    Se $a,b,c \in \mathbb{N}$ e $a \cdot c \leq b \cdot c$ então $a \leq b$.
\end{prop}
\begin{dem}
    Se $ac = bc$, pela lei do cancelamento do produto, $a=b \therefore a \leq b$.
    Se por outro lado, $ac < bc$, então $bc = ac + m$, para algum $m$ natural. Mostraremos por contradição, que não pode ocorrer $a > b$.
    Suponhamos que isso ocorra $a > b$, então $ac > bc$, e assim $ac = bc + n$ para algum $n$ natural. Então temos que 
    $bc = ac + m \iff bc = (bc + n) + m$, o que não pode ocorrer pela \Cref{nat-soma-props}, $a+b \neq a$. Uma outra demonstração dessa segunda parte seria pela tricotomia, que não permite que ocorra $ac < bc $ e $ac > bc$.
\end{dem}


\end{document}