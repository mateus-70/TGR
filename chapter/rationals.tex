\documentclass[../main.tex]{subfiles}
\begin{document}
\chapter{O CONJUNTO DOS NÚMEROS RACIONAIS}\label{cap-racionais}
\section{Ideias iniciais e objetivos}

No capítulo anterior, conseguimos criar um conjunto $\mathbb{Z}$ cuja soma tivesse elemento neutro e que todo elemento tivesse um elemento simétrico nessa mesma operaçao. Neste capítulo vamos tentar construir um conjunto tal que para a multiplicação, dado qualquer elemento (exceto o neutro da soma), seja também possível encontrar um simétrico. Para isso a bibliografia utilizada será \textcite{domingues-2009} e \textcite{ferreira}.

O conjunto que criaremos será chamado de conjunto dos números racionais e será denotado por $\mathbb{Q}$.

Para criar os racionais, vamos utilizar o mesmo artifício das classes de equivalência num subconjunto do produto cartesiano. O produto cartesiano que precisaremos excluirá o neutro da soma em $\mathbb{Z}$ da segunda cordenada, que se justificará no próximo teorema.

Nesse sentido, consideramos 
 \[ \mathbb{Z} \times \mathbb{Z}^* = \{\, \left( a,b \right) : a \in \mathbb{Z} \land b \in \mathbb{Z}^* \,\}. \]

Sobre $\mathbb{Z} \times \mathbb{Z}^*$ vamos considerar a relação definida por $\left( a,b \right) \sim \left( c,d \right) \iff ad = bc$.

\begin{teo}
    A relação $\sim$ é de equivalência.
\end{teo}
\begin{dem}
    Sejam $a,c,e \in \mathbb{Z}$ e $b,d,f \in \mathbb{Z}^*$. São válidas as propriedades:
    \begin{enumerate}[label=(\roman*)]
        \item Reflexiva: $\left( a,b \right) \sim \left( a,b \right)$, pois $a  \cdot b=b  \cdot a$.
        \item Simétrica: $ \left( a ,b \right) \sim \left( c,d \right) \implies \left( c,d \right) \sim \left( a,b \right) $. \\
        Temos $  (a,b) \sim (c,d) \implies a  \cdot d = b  \cdot c \implies c  \cdot b = d  \cdot a \implies \left( c,d \right) \sim \left( a,b \right) $.
        % \item Transitiva: $ \left( a,b \right) \sim \left( c,d \right) \land \left( c,d \right) \sim \left( e,f \right) \implies \left( a,b \right) \sim \left( e,f \right)$. \\
        % Temos $\left( a,b \right) \sim \left( c,d \right) \iff a  \cdot d=b  \cdot c$ \\
        % e também que $\left( c,d \right) \sim \left( e,f \right) \iff c  \cdot f=d  \cdot e$. \\
        % Se $a=0$, temos que $ad = 0$, assim, ocorre que $b=0 \lor c=0$. Como $b \neq 0$, então $c=0$. Reciprocamente, se $c=0$ então $bc = 0 = ad$, como $d \neq 0$, então $a=0$. Desse modo concluímos também que, $a=0$ se, e somente se $c=0$ e $e=0$. Temos daí que $ad = bc = cf = de = 0$. \\
        % Agora supondo que $a \neq 0$, então $c \neq 0$ e $e \neq 0$.
        % Podemos multiplicar os termos das igualdades iniciais, igualando os resultados e aplicando o cancelamento do produto, ficamos com: \\ 
        % $\left( a  \cdot d \right)  \cdot \left( c  \cdot f \right)= \left(b  \cdot c\right)  \cdot \left(d  \cdot e\right) \iff a  \cdot f = b  \cdot e \iff \left( a,b \right) \sim \left( e,f \right)$. \footnote{Caso admitíssemos o $0$ na segunda coordenada, teríamos $\frac{0}{0} \sim \frac{a}{b}$ para qualquer $a,b$, mas a transitiva iria falhar, por exemplo, $\frac{1}{2} \not\sim \frac{3}{2}$.}\label{rac-just-zeroDenominador}

        \item Transitiva: $ \left( a,b \right) \sim \left( c,d \right) \land \left( c,d \right) \sim \left( e,f \right) \implies \left( a,b \right) \sim \left( e,f \right)$. \\
        Supondo $(a,b) \sim (c,d)$ e $(c,d) \sim (e,f)$ temos que $ad=bc$ e $cf = de$. Logo $adf = bcf$ e $bcf = bde$. Assim $adf=bde$, ou seja $afd=bed$. Como $d \neq 0$ segue que $af=be$, e então $(a,b) \sim (e,f)$.
        
    \end{enumerate}
\end{dem}

\begin{defi}\label{rac-def-conjuntoQ}
   O conjunto dos números racionais é o conjunto $\mathbb{Z} \times \mathbb{Z}^* / \sim$, que será denotado por $\mathbb{Q}$.
\end{defi}
Claro que com a \Cref{rac-def-conjuntoQ}, os números de $\mathbb{Q}$ são classes de equivalência em $\mathbb{Z} \times \mathbb{Z}^*$.
Dessa forma, se $x \in \mathbb{Q}$, então $x = \overline{(a,b)}$. Ao invés de utilizar a notação ostensiva de pares ordenados como utilizamos no \Cref{cap-inteiros}, passaremos a utilizar a notação de fração, que consiste em separar os elementos do par, na ordem, por uma barra horizontal, denotada $\frac{a}{b}$ ou por uma barra inclinada denotada $a/b$. Dessa forma, $x = \overline{(a,b)}$ passará a ser denotado $x = \frac{a}{b}$ ou $x = a/b$. Nesse sentido, também indicamos que $a \in \mathbb{Z}$ e $b \in \mathbb{Z}^*$. Além disso, $a$ é chamado numerador, e $b$ é chamado denominador. Ainda, indicaremos que $\frac{a}{b} = \frac{c}{d}$ se, e somente se, $ad = bc$, isto é, $\overline{(a,b)} \sim \overline{(c,d)}$.

\begin{ex}
    Representam o mesmo número as notações a seguir: $1/2 =  2/4 = 3/6$. O que também é fácil de provar, vejamos que $1/2 =  2/4$. De fato, em $\mathbb{Z}$ vale $1 \cdot 4 = 2 \cdot 2$. Do mesmo modo para mostrar que $1/2 = 3/6$, pois $1 \cdot 6 = 2 \cdot 3$.
\end{ex}

\begin{prop}\label{rac-prop-cancelarFatorComumNumeradorDenominador}
    Seja $\frac{a}{b}$ um número racional. Vale que $ \frac{a}{b} = \frac{ac}{bc}$, para qualquer $c \in \mathbb{Z}^*$.
\end{prop}
\begin{dem}
    Basta observar que $a \cdot bc = b \cdot ac$, ou que $(a,b) \sim (ac,bc)$.
\end{dem}

%%%% ADICAO %%%%%

\section{Adição em $\mathbb{Q}$}
\begin{defi}\label{rac-def-soma}
    Sejam $\frac{a}{b}$ e $\frac{c}{d}$ números racionais. A adição entre $\frac{a}{b}$ e $\frac{c}{d}$, denotada por $\frac{a}{b} + \frac{c}{d}$, é definida como $\frac{ad+bc}{bd}$.
\end{defi}
\begin{teo}\label{rac-teo-somaBemDefinida}
    A adição está bem definida, isto é, se $\frac{a}{b} = \frac{a'}{b'}$ e $\frac{c}{d} = \frac{c'}{d'}$, então 
    $\frac{a}{b} + \frac{c}{d} = \frac{a'}{b'} + \frac{c'}{d'}.$
\end{teo}
\begin{dem}
    Sejam $a/b$ e $a'/b'$ duas representações de um mesmo número racional, tal que $\frac{a}{b} = \frac{a'}{b'}$, isto é, $ab'=ba'$. Do mesmo modo, sejam $c/d$ e $c'/d'$ duas representações de um número racional qualquer, tal que $\frac{c}{d} = \frac{c'}{d'}$, isto é, $cd' = dc'$.
    
    Vamos mostrar que $\frac{a}{b} + \frac{c}{d} = \frac{a'}{b'} + \frac{c'}{d'}$.
    Pela definição de adição, no primeiro caso temos $\frac{ad+bc}{bd}$, e no segundo caso temos $\frac{a'd'+b'c'}{b'd'}$.
    Devemos mostrar que $\frac{ad+bc}{bd} = \frac{a'd'+b'c'}{b'd'}$, ou seja, que 
    \begin{align*}
        (ad+bc)(b'd') &= (bd)(a'd'+b'c') \\ \shortintertext{que equivale a}
        adb'd'+bcb'd' &= bda'd' + bdb'c' \\ \shortintertext{isto é}
        ab'dd' + cd'bb' &= a'bdd' + c'dbb'.
    \end{align*}
       
    Como essa última igualdade é uma tautologia, pois $ab' = a'b$ e $cd' = c'd$, a igualdade desejada é verdadeira.
\end{dem}
% \begin{prop} Alteração de formato do texto/apresentação da proposicao
%     A adição em \mathbb{Q} é associativa, comutativa, tem elemento neutro, cada elemento tem um único simétrico, atende a lei do cancelamento e é fechada em \mathbb{Q}.
% \end{prop}
\begin{teo}\label{rac-teo-somaPropriedades}
    Para a adição em $\mathbb{Q}$ valem as seguintes propriedades:
    \begin{enumerate}[label=(\roman*)]
        \item Fechamento;
        \item Associativa;
        \item Comutativa;
        \item Da existência do elemento neutro; 
        \item Da existência do elemento simétrico;
        \item Lei do cancelamento;
    \end{enumerate}
\end{teo}
\begin{dem}
    Sejam $\frac{a}{b}, \frac{c}{d}, \frac{e}{f}$ números racionais quaisquer, temos:
    \begin{enumerate}[label=(\roman*)]
        \item Fechamento: \\
        \[ \frac{a}{b}+ \frac{c}{d} = \frac{ad+bc}{bd} \in \mathbb{Q} \text{, pois }ad+bc \in \mathbb{Z}\text{ e }bd \in \mathbb{Z}^*. \]
        \item Associativa: \\
        \begin{align*}
            \left( \frac{a}{b} + \frac{c}{d} \right) + \frac{e}{f} 
            &= \frac{ad+bc}{bd} + \frac{e}{f} \\
            &= \frac{(ad+bc)f + bde}{bdf} \\
            &= \frac{adf + bcf + bde}{bdf} \\
            &= \frac{adf+b(cf+de)}{bdf} \\
            &= \frac{a}{b} + \frac{cf+de}{df} \\
            &= \frac{a}{b} + \left( \frac{c}{d} + \frac{e}{f} \right).
        \end{align*}
        
        \item Comutativa:
        \[ \frac{a}{b} + \frac{c}{d} = \frac{ad+bc}{bd} = \frac{cb + da}{db} = \frac{c}{d} + \frac{a}{b}. \]
        \item Da existência do elemento neutro: \\ 
        Consideremos o número $\frac{0}{a}$, com $a \neq 0$. Temos:
        \[ \frac{0}{a} + \frac{c}{d} = \frac{0d+ac}{ad} = \frac{ac}{ad} = \frac{c}{d} \text{ pois } ac \cdot d = ad \cdot c.\]
        Portanto $\frac{0}{a}$ é o neutro aditivo em $\mathbb{Q}$, que será denotado por $0 = \frac{0}{a}$.
        \item Da existência do elemento simétrico: \\
        Dado $\frac{a}{b} \in \mathbb{Q}$, têm-se que $\frac{-a}{b} \in \mathbb{Q}$ é tal que
        \[ \frac{a}{b} + \frac{-a}{b} = \frac{ab + (-ab)}{bb} = \frac{0}{bb} = 0.\]
        \item A  lei do cancelamento decorre do \Cref{agb-teo-leiCancelamento}.
    \end{enumerate}
\end{dem}


%%% MULTIPLICACAO %%%
\section{A multiplicação em $\mathbb{Q}$}

\begin{defi}\label{rac-def-produto}
    Sejam $\frac{a}{b}$ e $\frac{c}{d}$ números racionais quaisquer. A multiplicação de $\frac{a}{b}$ por $\frac{c}{d}$ será denotada por $\frac{a}{b} \cdot \frac{c}{d}$ e é definida por $\frac{ac}{bd}$.
\end{defi}
\begin{teo}\label{rac-teo-produtoBemDefinido}
    A multiplicação em $\mathbb{Q}$ está bem definida.
\end{teo}
\begin{dem}
    Sejam $\frac{a}{b} = \frac{a'}{b'}$ e $\frac{c}{d} = \frac{c'}{d'}$ números racionais quaisquer. Então temos $ab' = ba'$ e $cd' = dc'$. Segue que \[ ab'cd' = ba'dc' \iff acb'd' = bda'c', \] 
    com isso $\frac{ac}{bd} = \frac{a'c'}{b'd'}$, ou seja, $\frac{a}{b} \cdot \frac{c}{d} = \frac{a'}{b'} \cdot \frac{c'}{d'}$.
\end{dem}
\begin{teo}\label{rac-teo-produtoPropriedades}
    Para a multiplicação em $\mathbb{Q}$ valem as seguintes propriedades:
    \begin{enumerate}[label=(\roman*)]
        \item Fechamento;
        \item Associativa;
        \item Comutativa;
        \item Da existência do elemento neutro; 
        \item Da existência do elemento simétrico;
        \item Lei do cancelamento.
    \end{enumerate}
\end{teo}
\begin{dem}
    Sejam $\frac{a}{b}, \frac{c}{d}, \frac{e}{f} \in \mathbb{Q}$.
    \begin{enumerate}[label=(\roman*)]
        \item Fechamento: \\
        Temos que $\frac{a}{b} \cdot \frac{c}{d} = \frac{ac}{bc} \in \mathbb{Q}$, pois  
        $ac \in \mathbb{Z}$ e $bd \in \mathbb{Z}$. Além disso, em $\mathbb{Z}$ vale que 
        \[ bd = 0 \iff b = 0 \lor d = 0, \] e como originalmente tínhamos $b,d$ ambos não nulos, temos que $bd \neq 0$.
        
        \item Associativa: \\
            \[
            \left( \frac{a}{b} \cdot \frac{c}{d}\right) \cdot \frac{e}{f} = 
            \left(\frac{ac}{bd}\right) \frac{e}{f} = 
            \frac{ace}{bdf} = \frac{a}{b} \cdot \frac{ce}{df} = 
            \frac{a}{b} \cdot \left(\frac{c}{d} \cdot \frac{e}{f}\right)
            .
            \]
        
        \item Comutativa: \\
        \[ \frac{a}{b} \cdot \frac{c}{d} = 
            \frac{ac}{bd} = 
            \frac{ca}{db} = 
            \frac{c}{d} \cdot \frac{a}{b} . \]
            
        
        \item Da existência do elemento neutro: \\
        Afirmamos que o $\frac{1}{1}$ é o neutro do produto, pois: $\left(1c/1d \right) = c/d$.
        Assim, provamos que é neutro pela esquerda. Pela direita é evidente pela comutatividade do produto em $\mathbb{Z}$. 
        \begin{obs}
            Para cada $a \in \mathbb{Z}^*$ tem-se que $\frac{a}{a}=\frac{1}{1}$, pois $a \cdot 1 = 1 \cdot a.$
        \end{obs}
        
        \item Da existência do elemento simétrico: \\
        Vamos obter o simétrico de $\frac{a}{b}$. Por hipótese, temos $b \neq 0$. Suponhamos que $a=0$. Vejamos se algum $\frac{c}{d}$
        pode ser simétrico de $\frac{a}{b}$. Temos $\frac{a}{b} \frac{c}{d} = \frac{ac}{bd} = \frac{0}{bd} \neq \frac{1}{1}$. Assim não podemos ter zero no numerador. \\
        Suponhamos por outro lado, $a \neq 0$. Temos $\frac{ac}{bd} = \frac{1}{1} \implies ac = bd$, que é o mesmo que dizer que $\frac{a}{b} = \frac{d}{c}$. Para que a igualdade ocorra, basta tomar $c=b$ e $d=a$, assim, o simétrico do produto de $\frac{a}{b}$, para $a \neq 0$, é $\frac{c}{d} = \frac{b}{a}$.
        
        \item A lei do cancelamento é consequência do \Cref{agb-teo-leiCancelamento}.
    \end{enumerate}    
\end{dem}


\section{A relação de ordem em $\mathbb{Q}$}
Para definirmos a relação de ordem em $\mathbb{Q}$ vamos proceder de maneira semelhante à de $\mathbb{Z}$, mas com uma diferença crucial, que exploraremos logo depois da \Cref{rac-def-relacaoOrdem}.
\begin{teo}\label{rac-teo-trocaSinaisNumeradorDenominador}
    Sejam $a,b$ números inteiros, sendo $b$ não nulo. As igualdades (em $\mathbb{Q}$) a seguir são válidas:
    \[ \frac{-a}{b} = \frac{a}{-b} = -\frac{a}{b}. \]
\end{teo}
\begin{dem}
    Para a primeira igualdade, temos 
    \[ \frac{-a}{b} = \frac{a}{-b} \iff (-a)(-b) = ba, \]
    que vale conforme \Cref{int-corol-produtoRegraSinal2}.
    Para mostrar a segunda igualdade, consideremos o número racional $\frac{a}{b}$. Ele é simétrico aditivo de $-\frac{a}{b}$, pois a soma é comutativa. Vejamos que ele também é simétrico aditivo de $\frac{-a}{b}$, pois vale que 
    \[ \frac{-a}{b} + \frac{a}{b} = \frac{(-a)b + ba}{bb} = \frac{ab - ab}{bb} = \frac{0}{bb} = \frac{0}{1}, \] 
    pelo \Cref{int-teo-produtoRegraSinal1}.
\end{dem}
\begin{corol}\label{rac-corol-escolhaDenominador}
    Qualquer que seja o número racional $\frac{a}{b}$, é sempre possível escolher uma representação $\frac{c}{d}$, de tal modo que $\frac{a}{b} = \frac{c}{d}$, com $d > 0$.
\end{corol}
\begin{dem}
    A primeira igualdade do \Cref{rac-teo-trocaSinaisNumeradorDenominador} diz que $\frac{-a}{b} = \frac{a}{-b}$. Os denominadores são $b$ e $-b$, ambos não nulos, logo pelo \Cref{int-corol-numeroOuSimetricoPositivo}, ao menos um deles é positivo.
\end{dem}

\begin{defi}\label{rac-def-relacaoOrdem}
    Sejam $\frac{a}{b}$ e $\frac{c}{d}$ números racionais quaisquer, sendo $b$ e $d$ inteiros positivos. A relação de ordem $\leq$ entre $\frac{a}{b}$ e $\frac{c}{d}$ será denotada por $\frac{a}{b} \leq \frac{c}{d}$ para indicar que $ad \leq bc$ e diremos que $\frac{a}{b}$ é menor do que ou igual a $\frac{c}{d}$. 
\end{defi}
\begin{obs}
    Deve ser notado que a última desigualdade da \Cref{rac-def-relacaoOrdem}, bem como seu produto, são consideradas em $\mathbb{Z}$.
\end{obs}

Vamos fazer agora a análise que comentamos no início da seção. Quando definimos a relação de ordem em $\mathbb{Z}$, podíamos utilizar qualquer representante da classe. Nos racionais, por outro lado, devemos restringir que os denominadores sejam positivos.

\begin{afi}
    A \Cref{rac-def-relacaoOrdem} não pode ser enfraquecida apenas retirando a obrigatoriedade dos denominadores serem positivos.
\end{afi}

\begin{dem}
    Observemos que o racional nulo pode ser representado das seguintes maneiras: $\frac{0}{1} = \frac{0}{-1}$ uma vez que $0 \cdot (-1) = 1 \cdot 0$.
    Seja $\frac{0}{1} \leq \frac{a}{b}$, temos $0b \leq 1a$. Por outro lado, de $\frac{0}{-1} \leq \frac{a}{b}$ temos $0b \leq -1a$,
    daí pelo \Cref{int-corol-numeroOuSimetricoPositivo}, temos $a \leq 0$. Note que não fizemos nenhuma suposição sobre $a \in \mathbb{Z}$. Assim, o mesmo método que diz $a \leq 0$ também diz que $0 \leq a$, o que é uma contradição.
\end{dem}

Com isso, a relação de ordem fica mais limitada no que diz respeito a escolhermos elementos para trabalhar com ela, pois agora não podemos utilizar denominadores negativos. Por outro lado, como o denominador ser negativo ou positivo é uma questão apenas de representação do número, e não do número em si, isso não danifica o caráter da relação de ordem, que é a de ordenar números e não suas representações. 

Devemos notar também que as operações de adição e multiplicação são independentes de representação, ou seja, podemos calcular uma soma ou produto sem receio de usar denominadores negativos, embora o denominador zero não possa ser utilizado, não tanto por questão de representação, mas porque começa a se configurar o conceito de número, e todo número racional precisa ter um denominador não nulo.

\begin{prop}\label{rac-prop-relOrdemBemDef}
    A relação de ordem está bem definida, isto é, se $\frac{a}{b} = \frac{a'}{b'}$ e $\frac{c}{d} = \frac{c'}{d'}$ com 
    $\frac{a}{b} \leq \frac{c}{d}$ e com $b,b',d,d' > 0$, então $\frac{a'}{b'} \leq \frac{c'}{d'}$.
\end{prop}
\begin{dem}
    Das hipóteses, temos: \\
    \[ ab' = ba',\ cd' = dc', \] 
    \[ ad \leq bc \iff 0 \leq bc-ad. \]
    Consideremos a expressão $(b'c'-a'd')bd$, temos: 
    \begin{align*}
        (b'c'-a'd')bd 
        &= (c'b'-a'd')bd  \\
        &= c'b'bd - a'd'bd \\
        &= dc'bb' - ba'dd' \\
        &= cd'b'b - ab'd'd \\
        &= (bc-ad) b'd'.     
    \end{align*}


    Então concluímos que, 
    \[ (b'c'-a'd')bd = (bc-ad) b'd'. \]

    Devemos observar que $0 < bd$, pois $b$ e $d$ são positivos por hipótese. O lado direito da igualdade é não negativo, pois 
    tanto $bc-ad$ quando $b'd'$ não são negativos. Assim concluímos que o lado esquerdo também não é negativo, logo $b'c'-a'd'$ deve ser não negativo, o que nos dá:
    \[ 0 \leq b'c'-a'd' \iff a'd' \leq b'c' \iff \frac{a'}{b'} \leq \frac{c'}{d'}. \]
    % \[ (b'c'-a'd')(bd)(b'd')(bd) = (bc-ad)(b'd')(b'd')(bd) \]

    % Agora vamos fazer considerações à respeito da relação ordem e dos sinais. \\
    % Temos $0 \leq (bc-ad)$ consequência direta da hipótese. Utilizando a compatibilidade do produto com a relação de ordem em $\mathbb{Z}$, e
    % considerando o membro direito da última equação e observando a expressão $(bc-ad)bd$, ela é positiva pois $b$ e $d$ tem o mesmo sinal, ambos positivos, por hipótese. Também será positivo $(b'd') \cdot (b'd')$, conforme \Cref{int-prop-quadradoPositivo}. \\
    % Consideremos o primeiro caso, suponhamos que $b,d$ tenham o mesmo sinal, desse modo
    % \[ (b'c'-a'd')(bd)(b'd')(bd) = (bc-ad)bd(b'd')(b'd') > 0 \]
    % e 
    % \[ (b'c'-a'd')b'd' > 0 \].
    
    
    % Se $(bc-ad)bd \geq 0$ então \\
    % $(c'b'-a'd')b'd'(bd)(bd) = (bc-ad)db(d'b')(d'b') > 0$ com isso \\
    % $(b'c'-a'd')b'd' > 0$.
    
\end{dem}

% \begin{prop}
%     Dado um número racional $\frac{a}{b}$, é sempre possível encontrar esse número com denominador positivo.
% \end{prop}
% \begin{dem}
%     Seja $b \leq 0$, tem-se que $\frac{a}{b} = \frac{-a}{-b}$ pois $a \cdot (-b) = b \cdot (-a)$, assim, $-b \geq 0$.
% \end{dem}

% \begin{prop}
%     Se um número racional $\frac{a}{b} > \frac{0}{1}$, e $b > 0$, então $a > 0$.
% \end{prop}
% \begin{dem}
%     Temos $0b < a$.
% \end{dem}

\begin{prop}\label{rac-prop-relOrdem}
    A relação de ordem no conjunto dos números racionais tem as seguintes propriedades:
    \begin{enumerate}[label=(\roman*)]
        \item Reflexiva;
        \item Antissimétrica;
        \item Transitiva;
        \item Totalidade;
        \item Compatibilidade com a adição;
        \item Compatibilidade com a multiplicação.
    \end{enumerate}
\end{prop}

\begin{dem}
    Sejam $\frac{a}{b}, \frac{c}{d}, \frac{e}{f}$ números racionais quaisquer tais que $b,d,f > 0$.
    \begin{enumerate}[label=(\roman*)]
        \item Reflexiva: \\
        \[ \frac{a}{b} \leq \frac{a}{b} \iff ab \leq ba. \]
        
        \item Antissimétrica: \\
        Supondo $\frac{a}{b} \leq \frac{c}{d}$ temos $ad \leq bc$. Por outro lado, $\frac{c}{d} \leq \frac{a}{b} $ implica que  $ cb \leq da$. Pela antissimetria de $\leq$ em $\mathbb{Z}$, concluímos que $ad = bc$, ou seja, $\frac{a}{b} = \frac{c}{d}$.
        
        \item Transitiva: \\
        Supondo $\frac{a}{b} \leq \frac{c}{d}$ e $\frac{c}{d} \leq \frac{e}{f}$, temos $ad \leq bc$ e $cf \leq de$.
        Pela compatibilidade do produto em $\mathbb{Z}$, temos: 
        \[adf \leq bcf \text{ e } bcf \leq bde, \]
        assim  $adf \leq bde$, e como $d \neq 0$, $af \leq be$. Portanto $\frac{a}{b} \leq \frac{e}{f}$.

        % Pela compatibilidade do produto em $\mathbb{Z}$, temos: \\
        % $adf \leq bcf$ e $bcf \leq bde$, assim $adf \leq bde$.
        % Logo $af \leq be \therefore \frac{a}{b} \leq \frac{e}{f}$.
        
        \item Totalidade: \\
        Vamos mostrar que $\frac{a}{b} \leq \frac{c}{d}$ ou $\frac{c}{d} \leq \frac{a}{b}$.
        Em $\mathbb{Z}$ vale a totalidade da relação de ordem, e assim $ad \leq bc$ ou $bc \leq ad$. Caso seja $ad \leq  bc$ temos $\frac{a}{b} \leq \frac{c}{d}$. Por outro lado, se for $bc \leq ad$, temos que $cb \leq da$, e portanto $\frac{c}{d} \leq \frac{a}{b}$. 
        
        \item Compatibilidade com a adição: \\
        Suponha $\frac{a}{b} \leq \frac{c}{d}$. Logo $ad \leq bc$. Como $f \neq 0$, as linhas abaixo são equivalentes:
        % Temos $\frac{a}{b} \leq \frac{c}{d} \iff ad \leq bc$. As linhas abaixo são equivalentes: \\
        \[ afd \leq bcf \]
        \[affd \leq bcff\]
        \[ affd + bedf \leq bcff + bdef \]   
        \[ (af+be)df \leq bf(cf+de) \]
        \[ \frac{af+be}{bf} \leq \frac{cf+de}{df} \]
        \[ \frac{a}{b} + \frac{e}{f} \leq \frac{c}{d} + \frac{e}{f}. \]
        
        \item Compatibilidade com a multiplicação: \\
        Se $\frac{a}{b} \leq \frac{c}{d}$ e $\frac{0}{1} \leq \frac{e}{f}$, temos $ad \leq bc$ e $e \geq 0$. Com isso 
        \begin{align*}
            ad \leq bc 
            &\iff adef \leq bcef \\
            &\iff  aedf \leq bfce \\
            &\iff \frac{ae}{bf} \leq \frac{ce}{df} \\
            &\iff \frac{a}{b} \frac{e}{f} \leq \frac{c}{d} \frac{e}{f}.   
        \end{align*}        
            % \[ ad \leq bc \]
            % \[adef \leq bcef \] 
            % \[ aedf \leq bfce \]
            % \[\frac{ae}{bf} \leq \frac{ce}{df} \] 
            % \[\frac{a}{b} \frac{e}{f} \leq \frac{c}{d} \frac{e}{f}. \]                   
    \end{enumerate}
\end{dem}

\begin{prop}\label{rac-prop-numeradorPositivo}
    Seja $r = \frac{a}{b}$ um racional, em que $b > 0$. Temos que $r \in \mathbb{Q}_{+}^*$ se, e somente se, $a > 0$.
\end{prop}
\begin{dem}
    Seja $0 = \frac{0}{1}$. Temos $0 < r = \frac{a}{b} \iff 0b \leq 1a \iff 0 \leq a$. \\
    Notemos que ambos os denominadores são positivos, o que nos permite efetivamente fazer a comparação por meio da relação de ordem.
\end{dem}

\begin{prop}\label{rac-prop-produtoMaioresMaior}
    Sejam $r,s,r',s' \in \mathbb{Q}_{+}^*$. Se $r \leq r'$ e $s \leq s'$, então $rs \leq r's'$.
\end{prop}
\begin{dem}
    Sejam $r = \frac{a}{b},\ s = \frac{c}{d},\ r' = \frac{a'}{b'},\ s' = \frac{c'}{d'}$, com todos os denominadores positivos. Pela \Cref{rac-prop-numeradorPositivo}, concluímos que cada numerador também é positivo. \\
    Temos por definição
    \[ r \leq r' \iff ab' \leq ba', \]
    \[ s \leq s' \iff cd' \leq dc'. \]

    Multiplicando os termos à esquerda entre si, e os termos à direita entre si, obtemos:
     \begin{align*}
         ab' \cdot cd' \leq ba' \cdot dc' 
         &\iff acb'd' \leq bda'c' \\
         &\iff \frac{ac}{bd} \leq \frac{a'c'}{b'd'} \\
         &\iff \frac{a}{b} \cdot \frac{c}{d} \leq \frac{a'}{b'} \cdot \frac{c'}{d'} \\
         &\iff rs \leq r's'.
     \end{align*}
     
    
    % \[ ab' \cdot cd' \leq ba' \cdot dc' \]
    % \[ acb'd' \leq bda'c' \]
    % \[ \frac{ac}{bd} \leq \frac{a'c'}{b'd'} \]
    % \[ \frac{a}{b} \cdot \frac{c}{d} \leq \frac{a'}{b'} \cdot \frac{c'}{d'} \]
    % \[ rs \leq r's' \]
    
    % \begin{equation*}
    %     % \frac{a}{b} \frac{c}{d} = \\
    %     \frac{ac}{bd} \leq \frac{a'c'}{b'd'} \iff \\ 
    %     acb'd' \leq bda'c' \iff ab' \cdot cd' \leq ba' \cdot dc' \\ 
    %     \iff ac \cdot b'd' \leq bd \cdot a'c' \iff \\
    %     \frac{ac}{bd} \leq \frac{a'c'}{b'd'}.
    % \end{equation*}
        
\end{dem}

\begin{prop}\label{rac-prop-solucaoAigualXB}
    Sejam $r,s \in \mathbb{Q}$ números fixos quaisquer, com $s \neq 0$. A equação $r = s \cdot t$ sempre admite solução para $t \in \mathbb{Q}$. 
\end{prop}
\begin{dem}
    De $r = st$, com $s \neq 0$, temos 
    \[ \frac{1}{s}\frac{r}{1} = \frac{1}{s}\frac{st}{1} \iff \frac{r}{s} = \frac{st}{s} = \frac{t}{1} = t. \]
    Logo, $t = \frac{r}{s} \in \mathbb{Q}$ é solução de $r=st$.
\end{dem}

\section{Imersão de $\mathbb{Z}$ em $\mathbb{Q}$}
\begin{teo}\label{rac-teo-imersao}
    Considere a função definida abaixo:
    \begin{align*}
        f \colon &\mathbb{Z} \rightarrow \mathbb{Q} \\
            & x \mapsto \frac{x}{1}.
    \end{align*}
    Essa função tem as propriedades a seguir:
    \begin{enumerate}[label=(\roman*)]
        \item $f(a + b) = f(a) + f(b)$;
        \item $f(a \cdot b) = f(a) \cdot f(b)$;
        \item $a \leq b \implies f(a) \leq f(b)$.
    \end{enumerate}
\end{teo}
\begin{dem}
    Temos que 
    \begin{enumerate}[label=(\roman*)]
        \item $f(a + b) = \frac{a+b}{1} = \frac{a}{1} + \frac{b}{1} = f(a) + f(b)$;
        \item $f(a \cdot b) = \frac{ab}{1} = \frac{a}{1} \frac{b}{1} = f(a) \cdot f(b)$;
        \item Supondo $a \leq b$, tem-se que $\frac{a}{1} \leq \frac{b}{1} $. Logo $ a1 \leq 1b$.
    \end{enumerate}
\end{dem}

Na demonstração do resultado a seguir vamos utilizar as imersões de $\mathbb{N}$ em $\mathbb{Z}$ e de $\mathbb{Z}$ em $\mathbb{Q}$ para confirmar uma visão prática. Um exemplo do que quero dizer é que $2$ é um número natural com a notação de número natural. Já $\frac{x}{2}$ com $x \in \mathbb{Z}$ é um número racional, mas o $2$ que figura no denominador é um número inteiro, com notação de número natural. Como já provamos a imersão de $\mathbb{N}$ em $\mathbb{Z}$, vamos utilizar a notação dos naturais com o sinal de $-$ quando necessário.

\begin{prop}\label{rac-prop-diferencaPositiva}
    Sejam $r,s$ números racionais positivos e distintos, tal que $s < r$. Então existe um único racional positivo $t$ tal que  $r = s+t$.
\end{prop}
\begin{dem}
    Temos que 
    \[ s<r \implies s-s < r-s \iff 0 < r-s. \]
    Assim o elemento $r-s$ é positivo, e também $s+(r-s) = r$, dessa forma $r-s$ é um $t$ do enunciado. Para mostrar que nenhum outro $t' \neq t$ satisfaz o enunciado, podemos supor que seja verdade que $r=s+t'$. Daí vem $r-s=t'$, e substituindo $r-s$ por $t$, obtemos $t = t'$. 
\end{dem}

\begin{prop}\label{rac-prop-somaEntreDobro}
    Sejam $s, r \in \mathbb{Q}$ com $s < r$. Vale que $2s < s+r < 2r$.
\end{prop}
\begin{dem}
    Como $s < r$ temos que $r = s + t$, para algum $t$ racional positivo (\Cref{rac-prop-diferencaPositiva}).
    % O número $t$ é positivo pois $s < r \implies 0 = s-s < r-s$. Aí temos $0 < t \implies 2s < 2s+t = s+r$. 
    Da compatibilidade da adição $s< r \implies s+s < s+r \implies 2s < s+r$.
    E para mostrar que $s+r < 2r$, temos
    \[ 0 < t \implies t < t+t = 2t, \] 
    assim 
    \[ s+r =  t+2s < 2t+2s = 2(t+s) = 2r. \] 
\end{dem}

\begin{corol}\label{rac-corol-terceiroEntreDois}
    Entre quaisquer dois números racionais distintos há sempre algum racional entre eles. É o mesmo que dizer
    que se $s,r \in \mathbb{Q}$ com $s<r$ então existe $t \in \mathbb{Q}$ tal que  $s < t < r$.
\end{corol}
\begin{dem}
    Supondo $s < r$, temos que $2s < s+r < 2r$, conforme \Cref{rac-prop-somaEntreDobro}, aí multiplicando por $\frac{1}{2} > 0$, obtemos 
    \[ s < \frac{s+r}{2} < r. \]
    Logo, existe $t = \frac{s+r}{2}$ tal que $s < t < r$.
\end{dem}

\begin{teo}\label{rac-teo-ilimitadoSuperiormente}
    O conjunto $\mathbb{Z}$ não é limitado superiormente em $\mathbb{Q}$.
\end{teo}
\begin{dem}
    Vamos argumentar por contradição, para provar que nenhum racional é cota superior de $\mathbb{Z}$.
    Suponha que $\frac{r}{s}$ é um racional, tal que $s>0$, em que $\frac{r}{s}$ é cota superior de $\mathbb{Z}$. Tomemos um número inteiro em $\mathbb{Q}$, que é da forma $\frac{a}{1}$, com $a \in \mathbb{Z}$. Devemos ter $\frac{a}{1} < \frac{r}{s}$, pois $\frac{r}{s}$ é uma cota superior de $\mathbb{Z}$. Logo, $as < r$. Mas essa desigualdade, com $r$ e $s$ fixos e $a$ qualquer, é o mesmo que dizer que $r$ é maior do que qualquer produto $as$, para quaisquer $a,s \in \mathbb{Z}_{+}$. Mas os inteiros positivos são identificados com os naturais, que por sua vez, são ilimitados em $\mathbb{Z}$. Desse modo, $r$ é uma cota superior de um conjunto ilimitado, o que é uma contradição.
\end{dem}

\end{document}