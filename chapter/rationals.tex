\documentclass[../main.tex]{subfiles}
\begin{document}
\chapter{O CONJUNTO DOS NÚMEROS RACIONAIS}
\section{Ideias iniciais e objetivos}

No capítulo anterior, conseguimos criar um conjunto \Z cuja soma tivesse elemento neutro e que todo elemento tivesse um elemento simétrico nessa mesma operaçao. Neste capítulo vamos tentar construir um conjunto tal que para a multiplicação, dado qualquer elemento (exceto o neutro da soma), seja também possível encontrar um simétrico. Para isso a bibliografia utilizada será \parencite{domingues-2009} e \parencite{ferreira}.

O conjunto que criaremos será chamado de conjunto dos números racionais e será denotado por \Q.

Para criar os racionais, vamos utilizar o mesmo artifício das classes de equivalência num subconjunto do produto cartesiano. O produto cartesiano que precisaremos excluirá o neutro da soma em \Z da segunda cordenada, que se justificará no próximo teorema.

$$\Z \times \Z^* = \{ \left( a,b \right) : a \in \Z \land b \in \Z^* \}$$

Sobre $\Z \times \Z^*$ vamos considerar a relação definida por $\left( a,b \right) \sim \left( c,d \right) \iff ad = bc$. Muito, muito analogamente à criação de \Z.

\begin{teo}
    A relação $\sim$ é de equivalência.
\end{teo}
\begin{dem}
    \begin{enumerate}[label=(\roman*)]
        \item Reflexiva: $\left( a,b \right) \sim \left( a,b \right) \iff a  \cdot b=b  \cdot a$.
        \item Simetria: $ \left( a ,b \right) \sim \left( c,d \right) \iff \left( c,d \right) \sim \left( a,b \right) $ \\
        Temos $ a  \cdot d = b  \cdot c = c  \cdot b = d  \cdot a \iff \left( c,d \right) \sim \left( a,b \right) $.
        \item Transitiva: $ \left( a,b \right) \sim \left( c,d \right) \land \left( c,d \right) \sim \left( e,f \right) \implies \left( a,b \right) \sim \left( e,f \right)$. \\
        Temos $\left( a,b \right) \sim \left( c,d \right) \iff a  \cdot d=b  \cdot c$ \\
        e também que $\left( c,d \right) \sim \left( e,f \right) \iff c  \cdot f=d  \cdot e$. \\
        Se $a=0$, temos que $ad = 0$, assim, ocorre que $b=0 \lor c=0$. Como $b \neq 0$, então $c=0$. Reciprocamente, se $c=0$ então $bc = 0 = ad$, como $d \neq 0$, então $a=0$. Desse modo concluímos também que, $a=0$ se, e somente se $c=0$ e $e=0$. Temos daí que $ad = bc = cf = de = 0$. \\
        Agora supondo que $a \neq 0$, então $c \neq 0$ e $e \neq 0$.
        Podemos multiplicar os termos das igualdades iniciais, igualando os resultados e aplicando o cancelamento do produto, ficamos com: \\ 
        $\left( a  \cdot d \right)  \cdot \left( c  \cdot f \right)= \left(b  \cdot c\right)  \cdot \left(d  \cdot e\right) \iff a  \cdot f = b  \cdot e \iff \left( a,b \right) \sim \left( e,f \right)$. \footnote{Caso admitíssemos o $0$ na segunda coordenada, teríamos $\frac{0}{0} \sim \frac{a}{b}$ para qualquer $a,b$, mas a transitiva iria falhar, por exemplo, $\frac{1}{2} \not\sim \frac{3}{2}$.}
    \end{enumerate}
\end{dem}

\begin{defi}
   Os números racionais são o conjunto $\Z \times \Z^* / \sim$, que será denotado por $\Q$.
\end{defi}
Claro que com essa definição, os números de \Q são classes de equivalência em $\Z \times \Z^*$, então quando escrevermos que $\frac{a}{b}$ ou $a/b$ fica entendido que eles são números racionais, estaremos também afirmando que $a \in \Z$ e que $b \in \Z^*$. Nesses exemplos, o $a$ é chamado numerador, e o $b$ é chamado denominador.

Como exemplo, representam o mesmo número as notações a seguir: $1/2, 2/4, 3/6$. O que também é fácil de provar, vejamos as duas primeiras representações: em \Z vale $1 \cdot 4 = 2 \cdot 2$.

\begin{obs}
    Ao invés de utilizar a notação ostensiva de pares ordenados como utilizamos nos inteiros, passaremos a utilizar a notação de fração, que consiste em separar os elementos do par, na ordem, por uma barra horizontal, denotada $\frac{a}{b}$ ou por uma barra inclinada denotada assim: $a/b$, em ambos os casos representam $\overline{\left( a,b \right)} \in \Q$. 
\end{obs}

\begin{prop}
    Seja $\frac{a}{b}$ um número racional, vale que $ \frac{a}{b} = \frac{ac}{bc}$, para qualquer $c \in \Z^*$.
\end{prop}
\begin{dem}
    Basta observar que $a \cdot bc = b \cdot ac$.
\end{dem}
%%%% ADICAO %%%%%

\section{Adição em \Q}
\begin{defi}
    Sejam $\frac{a}{b}$ e $\frac{c}{d}$ números racionais. Chamamos de adição entre $\frac{a}{b}$ e $\frac{c}{d}$, denotada por $\frac{a}{b} + \frac{c}{d}$ que é definido como $\frac{ad+bc}{bd}$.
\end{defi}
\begin{prop}
    A adição está bem definida.
\end{prop}
\begin{dem}
    Sejam $a/b$ e $a'/b'$ duas representações de um mesmo número racional, tal que não necessariamente $a = a'$ e $b=b'$. Do mesmo modo, sejam $c/d$ e $c'/d'$ duas representações de um número racional qualquer, tal que não necessariamente $c=c'$ e $d=d'$.
    
    Vamos mostrar que $\frac{a}{b} + \frac{c}{d} = \frac{a'}{b'} + \frac{c'}{d'}$.
    Pela definição de adição, no primeiro caso temos: $\frac{ad+bc}{bd}$, e no segundo caso temos $\frac{a'd'+b'c'}{b'd'}$.
    Vamos mostrar que o representante do primeiro caso se relaciona com o segundo, isto é, 
    \begin{center}
        $(ad+bc)(b'd') = (bd)(a'd'+b'c')$ \\
        $adb'd'+bcb'd' = bda'd' + bdb'c'$ \\
        $ab'dd' + cd'bb' = a'bdd' + c'dbb'$
    \end{center}
    Essa última igualdade é uma tautologia, pois $ab' = a'b$ e $cd' = c'd$.
\end{dem}
% \begin{prop} Alteração de formato do texto/apresentação da proposicao
%     A adição em \Q é associativa, comutativa, tem elemento neutro, cada elemento tem um único simétrico, atende a lei do cancelamento e é fechada em \Q.
% \end{prop}
\begin{prop}{Para a adição valem as seguintes propriedades:}
    \begin{enumerate}[label=(\roman*)]
        \item fechamento;
        \item associativa;
        \item comutativa;
        \item do elemento neutro; 
        \item do elemento simétrico;
        \item lei do cancelamento;
    \end{enumerate}
\end{prop}

\begin{dem}
    Sejam $\frac{a}{b}, \frac{c}{d}, \frac{e}{f}$ números racionais quaisquer, temos:
    \begin{enumerate}[label=(\roman*)]
        \item fechamento: pois $ad+bc \in \Z$ e $bd \in \Z^*$.
        \item associativa: \\
        \[ \left( \frac{a}{b} + \frac{c}{d} \right) + \frac{e}{f} = \frac{ad+bc}{bd} + \frac{e}{f} = \frac{(ad+bc)f + bde}{bdf}
        = \frac{adf + bcf + bde}{bdf} = \]
        \[ \frac{adf+b(cf+de)}{bdf} = \frac{a}{b} + \frac{cf+de}{df} = \frac{a}{b} + \left( \frac{c}{d} + \frac{e}{f} \right). \]
        
        \item comutativa:
        \[ \frac{a}{b} + \frac{c}{d} = \frac{ad+bc}{bd} = \frac{cb + da}{db} = \frac{c}{d} + \frac{a}{b}. \]
        \item do elemento neutro: Consideremos o número $\frac{0}{a}$, assim, $a \neq 0$. Temos:
        \[ \frac{0}{a} + \frac{c}{d} = \frac{0d+ac}{ad} = \frac{ac}{ad} = \frac{c}{d} \text{ pois } ac \cdot d = ad \cdot c. \footnote{Como também não depende de $a$, usualmente escreve-se somente $0$.}\]
        \item do elemento simétrico: basta considerar $\frac{a}{b}$ e $\frac{-a}{b}$, temos:
        \[ \frac{a}{b} + \frac{-a}{b} = \frac{ab + (-ab)}{bb} = \frac{0}{bb} = 0.\]
        \item lei do cancelamento, conforme \cref{agb-prop-leiCancelamento}.
    \end{enumerate}
\end{dem}


%%% MULTIPLICACAO %%%
\section{Multiplicação}

\begin{defi}
    Sejam $\frac{a}{b}$ e $\frac{c}{d}$ números racionais quaisquer. A multiplicação de $\frac{a}{b}$ por $\frac{c}{d}$ será denotada por $\frac{a}{b} \cdot \frac{c}{d}$ e é definida por $\frac{ac}{bd}$.
\end{defi}
\begin{prop}
    A multiplicação em \Q está bem definida.
\end{prop}
\begin{dem}
    Sejam $\frac{a}{b} = \frac{a'}{b'}$ e $\frac{c}{d} = \frac{c'}{d'}$ números racionais quaisquer. Então temos $ab' = ba'$ e $cd' = dc'$. Segue que $ab'cd' = ba'dc' \iff acb'd' = bda'c'$ então a classe $\frac{ac}{bd}$ é a mesma que classe $\frac{a'c'}{b'd'}$.
\end{dem}
\begin{prop}{Para a multiplicação valem as seguintes propriedades:}
    \begin{enumerate}[label=(\roman*)]
        \item fechamento;
        \item associativa;
        \item comutativa;
        \item do elemento neutro; 
        \item do elemento simétrico;
        \item lei do cancelamento;
    \end{enumerate}
\end{prop}
\begin{dem}
    \begin{enumerate}[label=(\roman*)]
        \item Fechamento: \\
        Temos que $ac \in \Z$ e $bd \in \Z$. Além disso, porque em \Z vale que $bd = 0 \iff b = 0 \lor d = 0$, considerando originalmente como tínhamos $b,d$ ambos não nulos, temos $bd \neq 0$.
        
        \item Associativa: \\
            \[
            \left( \frac{a}{b} \cdot \frac{c}{d}\right) \cdot \frac{e}{f} = 
            \left(\frac{a}{b} \cdot \frac{c}{d}\right) \frac{e}{f} = 
            \frac{ace}{bdf} = \frac{a}{b} \cdot \frac{ce}{df} = 
            \frac{a}{b} \cdot \left(\frac{c}{d} \cdot \frac{e}{f}\right)
            .
            \]
            
            
        
        \item Comutativa: \\
        \[ \frac{a}{b} \cdot \frac{c}{d} = 
            \frac{ac}{bd} = 
            \frac{ca}{db} = 
            \frac{c}{d} \cdot \frac{a}{b} . \]
            
        
        \item Elemento neutro: Afirmamos que o $\frac{1}{1}$ é o neutro do produto, pois: $\left(1c/1d \right) = c/d$.
        Assim, provamos que é neutro pela esquerda. Pela direita é evidente pela comutatividade do produto em \Z. A unicidade é consequência do teorema \cref{agb-neutro-unico}.
        
        \item Simétrico: Vamos obter o simétrico de $\frac{a}{b}$. Por hipótese temos $b \neq 0$. Suponhamos que $a=0$. Vejamos se algum $\frac{c}{d}$
        pode ser simétrico de $\frac{a}{b}$. Temos $\frac{a}{b} \frac{c}{d} = \frac{ac}{bd} = \frac{0}{bd} \neq \frac{1}{1}$. Assim não podemos ter zero no numerador. \\
        Suponhamos por outro lado, $a \neq 0$. Temos $\frac{ac}{bd} = \frac{1}{1} \implies ac = bd$, que é o mesmo que dizer que $\frac{a}{b} = \frac{d}{c}$. Para que a igualdade ocorra, basta tomar $c=b$ e $d=a$, assim, o simétrico do produto de $\frac{a}{b}$ é $\frac{c}{d} = \frac{b}{a}$. O simétrico é único, conforme \cref{agb-teo-simetricoUnico}.
        
        \item a lei do cancelamento é provada pela \cref{agb-prop-leiCancelamento}.

    \end{enumerate}    
\end{dem}


\section{Relação de ordem}
\begin{defi}
    Sejam $\frac{a}{b}$ e $\frac{c}{d}$ números racionais quaisquer. A relação de ordem $\leq$ entre $\frac{a}{b}$ e $\frac{c}{d}$ será denotada por $\frac{a}{b} \leq \frac{c}{d}$ e representa que $ad \leq bc$. 
\end{defi}
\begin{obs}
    Deve ser notado que a última desigualdade da definição, bem como seu produto são em \Z.
\end{obs}
\begin{prop}
    A relação de ordem está bem definida. \todo{Isso está incompleto} \\
    vi a demosntracao nesse site \url{https://math.stackexchange.com/questions/4626648/proving-order-is-well-defined-on-mathbbq}\\
    O problema é que eu não achei legal em vários lugares já supor que os denominadores são positivos.
    Isto é, sejam $\frac{a}{b} = \frac{a'}{b'}$ e $\frac{c}{d} = \frac{c'}{d'}$. Ainda, suponha que $\frac{a}{b} \leq \frac{c}{d}$. Então 
    $\frac{a'}{b'} = \frac{c'}{d'}$.
\end{prop}
\begin{dem}
    Das hipóteses temos: \\
    $ab' = ba'$, $cd' = dc'$ e $ad \leq bc \iff 0 \leq bc-ad$ 
    Temos que: 
    \[ (c'b'-a'd')bd = \]
    \[ c'b'bd - a'd'bd = \]
    \[ cd'b'b - ab'd'd = \]
    \[ (bc-ad) b'd' \]

    Assim, 
    \[ (b'c'-a'd')(bd)(b'd')(bd) =\]
    \[ (bc-ad)(b'd')(b'd')(bd) \]

    Agora vamos fazer considerações à respeito da relação ordem e dos sinais. \\
    Temos $0 \leq (bc-ad)$ por hipótese.
    Observando a expressão $(bc-ad)bd$, ela será positiva se $b$ e $d$ tiverem o mesmo sinal, e será negativa caso contrário. Por outro lado, um número multiplicado por ele mesmo sempre será positivo. \\
    Consideremos o primeiro caso, suponhamos que $b,d$ tenham o mesmo sinal, desse modo
    \[ (b'c'-a'd')(bd)(b'd')(bd) = (bc-ad)bd(b'd')(b'd') > 0 \]
    e 
    \[ (b'c'-a'd')b'd' > 0 \].
    
    
    % Se $(bc-ad)bd \geq 0$ então \\
    % $(c'b'-a'd')b'd'(bd)(bd) = (bc-ad)db(d'b')(d'b') > 0$ com isso \\
    % $(b'c'-a'd')b'd' > 0$.
    
\end{dem}

\begin{prop}
    Dado um número racional $\frac{a}{b}$, é sempre possível encontrar esse número com denominador positivo.
\end{prop}
\begin{dem}
    Seja então $b \leq 0$, tem-se que $\frac{a}{b} = \frac{-a}{-b}$ pois $a \cdot (-b) = b \cdot (-a)$, assim, $-b \geq 0$.
\end{dem}

\begin{prop}
    Se um número racional $\frac{a}{b} > \frac{0}{1}$, e $b > 0$, então $a > 0$.
\end{prop}
\begin{dem}
    Temos $0b < a$.
\end{dem}

\begin{prop}{Relação de ordem}
    \begin{enumerate}[label=(\roman*)]
        \item Reflexiva:
        \item Antissimétrica:
        \item Transitiva:
        \item Totalidade:
        \item Compatível com a adição:
        \item Compatível com a multiplicação:
    \end{enumerate}
\end{prop}

\begin{dem}
    Sejam $\frac{a}{b}, \frac{c}{d}, \frac{e}{f}$ números racionais quaisquer tais que $b,d,f > 0$.
    \begin{enumerate}[label=(\roman*)]
        \item Reflexiva, pois $\frac{a}{b} \leq \frac{a}{b} \iff ab \leq ba$.
        \item Antissimétrica, pois se $\frac{a}{b} \leq \frac{c}{d} \iff ad \leq bc$. Por outro lado, $\frac{c}{d} \leq \frac{a}{b} \iff cb \leq da$. Pela antissimetria de $\leq$ em \Z, concluímos que $\frac{a}{b} = \frac{c}{d}$.
        \item Transitiva: \\
        $\frac{a}{b} \leq \frac{c}{d} \iff ad \leq bc$. \\
        $\frac{c}{d} \leq \frac{e}{f} \iff cf \leq de$. \\
        Pela compatibilidade do produto em \Z, temos: \\
        $adf \leq bcf$ e $bcf \leq bde$, assim $adf \leq bde$.
        Logo $af \leq be \therefore \frac{a}{b} \leq \frac{e}{f}$.
        
        \item Totalidade:
        Temos que $\frac{a}{b} \leq \frac{c}{d} \iff ad \leq bc$. \\
        Também, $\frac{c}{d} \leq \frac{a}{b} \iff cb \leq da$. \\
        Como a relação de ordem em \Z é total, aqui também é, como consequência.
        
        \item Compatível com a adição: \\
        Temos $\frac{a}{b} \leq \frac{c}{d} \iff ad \leq bc$. As linhas abaixo são equivalentes: \\
        \[ afd \leq bcf \]
        \[ afd + bed \leq bcf + bde \]   
        \[ (af+be)df \leq bf(cf+de) \]
        \[ \frac{af+be}{bf} \leq \frac{cf+de}{df} \]
        \[ \frac{a}{b} + \frac{e}{f} \leq \frac{c}{d} + \frac{e}{f}. \]
        
        \item Compatível com a multiplicação:
        Se $\frac{a}{b} \leq \frac{c}{d}$ e $\frac{0}{1} \leq \frac{e}{f}$, temos: \\
        
            \[ ad \leq bc \]
            \[adef \leq bcef \] 
            \[ aedf \leq bfce \]
            \[\frac{ae}{bf} \leq \frac{ce}{df} \] 
            \[\frac{a}{b} \frac{e}{f} \leq \frac{c}{d} \frac{e}{f}. \]                   
    \end{enumerate}
\end{dem}


\section{Imersão de \Z em \Q}
\begin{teo}\label{rac-teo-imersao}
    Seja a função $f: \Z \rightarrow \Q, f(x) \mapsto \frac{x}{1}$. Essa função tem as propriedades a seguir:
    \begin{enumerate}[label=(\roman*)]
        \item $f(a + b) = f(a) + f(b)$;
        \item $f(a \cdot b) = f(a) \cdot f(b)$;
        \item $a \leq b \implies f(a) \leq f(b)$.
    \end{enumerate}
\end{teo}
\begin{dem}
    \begin{enumerate}[label=(\roman*)]
        \item $f(a + b) = \frac{a+b}{1} = \frac{a}{1} + \frac{b}{1} = f(a) + f(b)$;
        \item $f(a \cdot b) = \frac{ab}{1} = \frac{a}{1} \frac{b}{1} = f(a) \cdot f(b)$;
        \item Temos que $a \leq b$, assim, $\frac{a}{1} \leq \frac{b}{1} \iff a1 \leq 1b$ o que sempre ocorre, observando a hipótese.
    \end{enumerate}
\end{dem}

\end{document}