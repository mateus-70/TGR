\documentclass[../main.tex]{subfiles}
\begin{document}

\chapter*[Apêndice]{APÊNDICE A - ÁLGEBRA BÁSICA}
\addcontentsline{toc}{chapter}{APÊNDICE A - ÁLGEBRA BÁSICA}

\begin{defi}
    Seja $A$ um conjunto arbitrário. Uma operação $*$ sobre $A$ é uma função que a cada $x,y \in A$ associa um elemento $x * y \in A$, ou seja, associa a cada 2 elementos em $A$ a sua imagem $x * y$ que também é um elemento de $A$.
\end{defi}
Por essa definição, a adição em \N, \Z, \Q e \R são operações, sobre cada um desses conjuntos, cada um com sua adição. Por outro lado, a subtração em \N não é uma operação em \N pois $1-2 \neq \in \N$. Também a divisão não é uma operação em \N e em \Z, pelo mesmo motivo.

\begin{defi}
    Seja $A$ um conjunto e seja $*$ uma operação em $A$. Se existir um $e \in A$ tal que para qualquer $x \in A$ valer $e * x = x$ dizemos que $e$ é elemento neutro à esquerda.
\end{defi}
Por analogia definimos neutro à direita.

\begin{defi}
    Seja $A$ um conjunto, , e seja $*$ uma operação sobre $A$. Dizemos que a operação $*$ tem a propriedade:
    \begin{enumerate}[label=(\roman*)]
        \item associativa, quando para quaisquer $x,y,z \in A$ é válido que $ x * ( y * z ) = ( x * y ) * z$.
        \item comutativa, quando para quaisquer $x,y \in A$ é válido que $x * y = y * x$.
        \item do elemento neutro, quando existe $e \in A$, onde $e$ é neutro à esquerda e à direita.
        \item do elemento simétrico, quando para qualquer $x \in A$ existe algum $w \in A$, onde é válido que $x * w$ é o elemento neutro de $*$.
        \item do fechamento, quando para quaisquer $x, y \in A$ ocorre que $x * y \in A$.
    \end{enumerate}
\end{defi}
Em geral, nesse trabalho, o elemento neutro para a operação de adição é o $0$ e o elemento neutro para a multiplicação é o $1$. 
\begin{obs}
    A propriedade do fechamento na verdade é imediata da definição de operação, colocamos para realçar o fato de que a operação tem imagem em $A$ e não qualquer elemento que não esteja em $A$.
\end{obs}

\begin{defi}
    Sejam $A$ um conjunto e sejam $+$ e $\cdot$ operações sobre $A$. Dizemos que a multiplicação é distributiva em relação a soma quando para quaisquer $x, y, z \in A$ é valido que $x \cdot (y + z) = x\cdot y + x\cdot z$.
\end{defi}
Frequentemente omitimos o símbolo da multiplicação.

\begin{teo}
    Seja $A$ um conjunto e $*$ uma operação em $A$. Se existir elemento neutro para $*$, então ele é único.
\end{teo}
\begin{dem}
    Sejam $e_1 e e_2$ dois elementos neutros para $*$. Temos que $e_1 = e_1 * e_2 = e_2$, portanto $e_1 = e_2$. A primeira igualdade é porque $e_2$ é neutro à direita, a segunda igualdade é porque $e_1$ é neutro à esquerda.
\end{dem}
Em particular deve ser observada a comutatividade do elemento neutro com qualquer elemento, para uma dada operação.
\begin{teo}
    Seja $A$ um conjunto e $*$ uma operação em $A$. Se $*$ é associativa e tem a propriedade do elemento simétrico, então o simétrico de cada elemento é único.
\end{teo}
\begin{dem}
    Seja $x \in A$, e seja $y, z$ dois elementos simétricos de $x$, assim temos que $x*y = e = x*z$. 
    Temos $y = y * e = y * (x * z) = (y * x) * z = ( x * y ) * z = e * z = z$.
\end{dem}

\begin{defi}
    Seja $A$ um conjunto, e sejam $+, \cdot$ duas operações sobre $A$, chamadas de adição e multiplicação. Se são válidas as propriedades:
    \begin{enumerate}[label=(\roman*)]
        \item associativa da adição
        \item comutativa da adição;
        \item do elemento neutro para adição;
        \item do elemento simétrico para a adição;
        \item associativa da multiplicação;
        \item comutativa da multiplicação;
        \item do elemento neutro da multiplicação;
        \item da distributiva do produto em relação à soma;
    \end{enumerate}
    dizemos que $(A, +, \cdot)$ é um anel. Quando não houver ambiguidade chama-se apenas de anel $A$.

\end{defi}

\begin{defi}
    Um corpo é um anel $A$ em que cada elemento diferente do zero aditivo tem um simétrico multiplicativo.
\end{defi}

\begin{defi}{Par ordenado}\label{def-par-ord}
     Dados um conjunto não vazio $A$ e $a,b \in A$, definimos o \emph{par ordenado $(a,b)$} como sendo o conjunto $\{\{a\}, \{a,b\}\}$.
\end{defi}
\begin{obs}
    Essa definição de par ordenado visa formar conjuntos onde a "ordem" importa, isto é, $(a , b) = (c , d) \iff a=c \land b=d$.
\end{obs}
\begin{defi}{Produto Cartesiano}\label{def-prod-cart}
     Dado um conjunto $A$, o produto cartesiano de A por A, denotado por $A \times A$, é o conjunto de todos os pares ordenados compostos por elementos de $A$, isto é, $A \times A = \{ (x,y) : x,y \in A \}$.
\end{defi}
\begin{defi}{Relação binária}
    Uma \emph{relação binária R} num conjunto $A$ é qualquer subconjunto do produto cartesiano $A \times A$, isto é, $R \subset A \times A$.
\end{defi}
\begin{defi}
    Uma relação $R$ é chamada de \emph{Relação de equivalência} quando possuir as seguintes propriedades:
    \begin{enumerate}[label=(\roman*)]
        \item reflexiva $aRa, \forall a \in A$;
        \item simétrica $aRb \implies bRa, \forall a,b \in A$;
        \item transitiva $aRb \land bRc \implies aRc, \forall a,b,c \in A$.
    \end{enumerate}
\end{defi}
\begin{defi}
    Seja $R$ uma relação de equivalência num conjunto $A$ e seja $a \in A$ um elemento fixado arbitrariamente. O conjunto \\
    $$\Bar{a} = \{x \in A : xRa\}$$
    chama-se \emph{classe de equivalência de a pela relação R}.
\end{defi}
\begin{teo}
    Sejam $R$ uma relação de equivalência em um conjunto $A$ e sejam $a,b$ elementos quaisquer de $A$, então:
    \begin{enumerate}
        \item $a \in \Bar{a}$;
        \item $\Bar{a} = \Bar{b} \iff aRb$;
        \item $\Bar{a} \neq \Bar{b} \implies \Bar{a} \cap \Bar{b} = \emptyset $.
    \end{enumerate}
\end{teo}
\begin{defi}
    Seja $R$ uma relação de equivalência num conjunto $A$. O conjunto constituído das classes de equivalência em $A$ pela relação $R$ é denotado por $A / R$ e denominado \emph{conjunto quociente} de $A$ por $R$. \\
    Assim, $A / R = \{\Bar{a} : a \in A\}$.
\end{defi}
\end{document}
