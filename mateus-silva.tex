%-------------------------------------------------------------------------
%-------------------------------------------------------------------------
% Modelo de TGR para o curso de Licenciatura em Matemática - UDESC/CCT
%
% Este modelo usa o pacote ABNTeX2, para instalar siga as seguintes instruções:
% Para poder usar o ABNTeX2 é necessário ter o LaTeX instalado. Para Windows, o MikTeX é o mais usado. 
% Vá até a página de download do abntex2 (http://code.google.com/p/abntex2/downloads/list), baixe uma das versões (eu usei abntex2.tds-1.7.1.zip) e descompacte em uma pasta temporária.
% Encontre a pasta na qual o MikTeX instalou seus arquivos (no meu computador foi C:\Arquivos de Programas(x86)\MikTex 2.9).
% Vá na pasta C:\temp\abntex-<versão>\texmf, selecione o conteúdo e copie tudo.
% Agora vá na pasta de instalação do MikTex e cole tudo (talvez vá sobrescrever alguma coisa). São os diretórios bibtex, doc e tex.
% Feito isso, será necessário atualizar o MikTeX destas mudanças. Para isso, vá em  Iniciar/Programas/MikTeX/Maintenance(Admin)/Settings e clique no botão Refresh Now.
% No Linux basta instalar o pacote texlive-full.
%
%
% ------------------------------------------------------------------------
% ------------------------------------------------------------------------

\documentclass[
	% -- opções da classe memoir --
	12pt,				% tamanho da fonte
	oneside,			% para impressão em verso e anverso. Oposto a oneside
	a4paper,			% tamanho do papel. 
	% -- opções do pacote babel --
	english,			% idioma adicional para hifenização
	french,				% idioma adicional para hifenização
	spanish,			% idioma adicional para hifenização
	brazil,				% o último idioma é o principal do documento
	]{abntex2}


% ---
% PACOTES
% ---

% ---
% Pacotes fundamentais 
% ---
\usepackage{cmap}				% Mapear caracteres especiais no PDF
\usepackage{
lmodern}			% Usa a fonte Latin Modern			
\usepackage[T1]{fontenc}		% Selecao de codigos de fonte.
\usepackage[utf8]{inputenc}		% Codificacao do documento (conversão automática dos acentos)
\usepackage{lastpage}			% Usado pela Ficha catalográfica
\usepackage{indentfirst}		% Indenta o primeiro parágrafo de cada seção.
\usepackage{color}				% Controle das cores
%\usepackage[dvips]{graphicx}    % Inclusão de gráficos (.eps)
\usepackage{graphicx} 
\usepackage{epstopdf}			%conversão direta para pdf permitindo inserir figuras jpg
\usepackage{amsmath}
\usepackage{amsfonts}
\usepackage{amssymb}
\usepackage{amsthm}
\usepackage[left=3cm,right=2cm,top=3cm,bottom=2cm]{geometry}


% ---
\usepackage{titlesec} %Adequação dos títulos as normas da UDESC
% ---
\titleformat{\chapter}
{\normalfont\normalfont\bfseries}
{\thechapter.}{0.5em}{}



\titleformat{\section}
{\normalfont}
{\thesection.}{0.5em}{}

\titleformat{\subsection}
{\normalfont}
{\thesection.}{0.5em}{}

% ---
\usepackage{etoc} %Adequação do sumário as normas da UDESC
% ---

\etocsetstyle{section}
{\normalfont}
{\addvspace{.5ex}\noindent\setlength{\leftskip}{1.5cm}\noindent}
{\llap{\makebox[1.5cm][l]{{\etocnumber}}}\etocname
  \hspace{10pt}\nobreak\dotfill\hspace{10pt}{\etocpage}\par}
{}
\etocsetstyle{chapter} 
{\bfseries}
{\addvspace{.5ex}\setlength{\leftskip}{1.5cm}\noindent}
{\llap{\makebox[1.5cm][l]{\bfseries\etocnumber}}\etocname
  \hspace{10pt}\nobreak\dotfill\hspace{10pt}\etocpage\par}
{}

\etocsetstyle{chapter} 
{\bfseries}
{\addvspace{.5ex}\setlength{\leftskip}{1.5cm}\noindent}
{\llap{\makebox[1.5cm][l]{\bfseries\etocnumber}}\etocname
  \hspace{10pt}\nobreak\dotfill\hspace{10pt}\etocpage\par}
{}

\etocsetstyle{subsection}
{\bfseries}
{\addvspace{.5ex}\noindent\setlength{\leftskip}{1.5cm}\noindent}
{\llap{\makebox[1.5cm][l]{{\bfseries\etocnumber}}}\etocname
  \hspace{10pt}\nobreak\dotfill\hspace{10pt}{\etocpage}\par}
{}
\etocsetstyle{chapter} 
{\bfseries}
{\addvspace{.5ex}\setlength{\leftskip}{1.5cm}\noindent}
{\llap{\makebox[1.5cm][l]{\bfseries\etocnumber}}\etocname
  \hspace{10pt}\nobreak\dotfill\hspace{10pt}\etocpage\par}
{}

\etocsettocstyle{\chapter*{\bfseries SUMÁRIO}}
{\bfseries}
\setcounter{tocdepth}{1}
\setcounter{secnumdepth}{2}



% ---
% Novos Comandos 
% ---
\newcommand{\R}{\ensuremath {\mathbb{R}} }
\newcommand{\Z}{\ensuremath {\mathbb{Z}} }
\newcommand{\Q}{\ensuremath {\mathbb{Q}} }
\newcommand{\N}{\ensuremath {\mathbb{N}} }
\newcommand{\C}{\ensuremath {\mathbb{C}} }
\newcommand{\K}{\ensuremath {\mathbb{K}} }
\renewcommand{\S}{\ensuremath {\mathbb{S}} }



% ---
% Estilo teoremas  e atalhos para comandos 
% ---
\theoremstyle{plain}
\newtheorem{teo}{Teorema}[chapter]
\newtheorem{prop}{Proposição}[chapter]
\newtheorem{lema}{Lema}[chapter]
\newtheorem{corol}{Corolário}[chapter]
\newtheorem{defi}{Definição}[chapter]
\newtheorem{axi}{Axioma}
\newtheorem{peano}{Peano}

\theoremstyle{definition}
\newtheorem{obs}{Observação}[chapter]
\newtheorem{ex}{Exemplo}[chapter]

\newenvironment{dem}[1][\textbf{Demonstração:} \ ]{\textbf{#1}}{\hfill\rule{2mm}{2mm}}
\newcommand{\afi}{\textbf{Afirmação:} }
% ---

%\usepackage{lipsum}				% para geração de dummy text

% ---
% Pacotes de citações
% ---
\usepackage[brazilian,hyperpageref]{backref}	 % Paginas com as citações na bibl
\usepackage[alf]{abntex2cite}	% Citações padrão ABNT





% --- 
% CONFIGURAÇÕES DE PACOTES
% --- 

% ---
% Configurações do pacote backref
% Usado sem a opção hyperpageref de backref
\renewcommand{\backrefpagesname}{Citado na(s) página(s):~}
% Texto padrão antes do número das páginas
\renewcommand{\backref}{}
% Define os textos da citação
\renewcommand*{\backrefalt}[4]{
	\ifcase #1 %
		Nenhuma citação no texto.%
	\or
		Citado na página #2.%
	\else
		Citado #1 vezes nas páginas #2.%
	\fi}%
% ---

% ---
%%Cabeçalho
% ---
\makepagestyle{meuestilo}
  %%cabeçalhos
 % \makeevenhead{meuestilo} %%pagina par
   %  {topo par à esquerda}
   %  {centro \thepage}
    % {direita}
  \makeoddhead{meuestilo} %%pagina ímpar ou com oneside
     {} %esquerda
     {} %centro
     {\thepage}%direita
  
 

%
%%Mateus Schroeder da Silva
%

\newcommand{\authorName}{MATEUS SCHROEDER DA SILVA}
\newcommand{\orientationBy}{Doutor Marnei Luis Mandler}
\newcommand{\titleTGR}{A CONSTRUÇÃO DOS NÚMEROS}
\newcommand{\subtitleTGR}{DOS NATURAIS AOS REAIS}
\newcommand{\titleComplete}{\textbf{\titleTGR:} \subtitleTGR}

% ----
% Início
% ----

\titulo{\titleComplete}
\autor{\authorName}
\local{JOINVILLE, SC}
\date{2023}
\tipotrabalho{TRABALHO DE CONCLUSÃO DE CURSO}
\preambulo{ Trabalho de conclusão de curso apresentado como requisito parcial para obtenção do título de licenciado em Matemática pelo curso de Licenciatura em Matemática do Centro de Ciências Tecnológicas - CCT, da Universidade do Estado de Santa Catarina - UDESC. \\
 Orientador: \orientationBy}
% ---


\makeatother
% --- 

% --- 
% Espaçamentos entre linhas e parágrafos 
% --- 

% O tamanho do parágrafo é dado por:
\setlength{\parindent}{1.25cm}

% Controle do espaçamento entre um parágrafo e outro:
\setlength{\parskip}{0.2cm}  % tente também \onelineskip

% ---
% compila o indice
% ---
\makeindex
% ---

% ----
% Início do documento
% ----
\begin{document}

% Retira espaço extra obsoleto entre as frases.
\frenchspacing 

% ----------------------------------------------------------
% ELEMENTOS PRÉ-TEXTUAIS
% ----------------------------------------------------------

\pretextual

% ---------------------------------------------
% capa
% ---------------------------------------------
\begin{center}
\thispagestyle{empty}%página não enumerada
 \textbf{ UNIVERSIDADE DO ESTADO DE SANTA CATARINA - UDESC\\
CENTRO DE CIÊNCIAS TECNOLÓGICAS - CCT\\
CURSO DE LICENCIATURA EM MATEMÁTICA\\}
\vspace{4 cm}\textbf{\authorName}

\vspace{4 cm}\titleComplete

\vspace{\stretch{1}}
\textbf{JOINVILLE - SC}

\textbf{2023}
 \pagebreak
\end{center}
% ---


% ------------------------------------------------
% folha de rosto
% ------------------------------------------------

\begin{folhaderosto}

  \begin{center}
    \textbf{\authorName}

    \vspace*{\fill}\vspace*{\fill}
    \titleComplete
    \vspace*{\fill}
  \end{center}
  
  \begin{flushright}
  \begin{minipage}[t]{8 cm}
  { Trabalho de conclusão de curso apresentado como requisito parcial para obtenção do título de licenciado em Matemática pelo curso de Licenciatura em Matemática do Centro de Ciências Tecnológicas - CCT, da Universidade do Estado de Santa Catarina - UDESC. 

 Orientador: \orientationBy}
  \end{minipage}
  \end{flushright}
  
  \vspace{\stretch{1}}
  
  \begin{center}
 \textbf{JOINVILLE - SC}
 
 \textbf{2023} 
  \end{center}
\end{folhaderosto}
% ---



% ----------------------------------------------------
% Inserir folha de aprovação
% ----------------------------------------------------

% Isto é um exemplo de Folha de aprovação, elemento obrigatório da NBR
% 14724/2011 (seção 4.2.1.3). Você pode utilizar este modelo até a aprovação
% do trabalho. Após isso, substitua todo o conteúdo deste arquivo por uma
% imagem da página assinada pela banca com o comando abaixo:
%
% \includepdf{folhadeaprovacao_final.pdf}
% ---


% ---------------------------------------------------
% folha de aprovação
% ---------------------------------------------------
\begin{folhadeaprovacao}

  \begin{center}
    \textbf{\authorName}
\vspace {1 cm}

   \titleComplete 
   \vspace {1 cm}
  \end{center}
    
\begin{flushright}
  \begin{minipage}[t]{8 cm}
  { Trabalho de conclusão de curso apresentado como requisito parcial para obtenção do título de licenciado em Matemática pelo curso de Licenciatura em Matemática do Centro de Ciências Tecnológicas - CCT, da Universidade do Estado de Santa Catarina - UDESC. 

 Orientador: \orientationBy}
 
  \end{minipage}
  \end{flushright}
     
     \begin{minipage}[c]{3cm} 
	Membros:
\end{minipage}
\begin{minipage}[c]{8 cm}
	\begin{center}
	\textbf{BANCA EXAMINADORA}
	\vspace {2 cm}
	
    \orientationBy
	
    Nome da Instituição
    
    \vspace {1.5 cm}
    
    Nome do Membro da banca e Titulação
	
    Nome da Instituição
    
    \vspace {1.5 cm}
    
    Nome do Membro da banca e Titulação
	
    Nome da Instituição
    \vspace {1.5 cm}
    
    
\end{center}

\end{minipage}

\vspace*{\fill}
     \begin{center}
	     Joinville, \today.
\end{center}

    
\end{folhadeaprovacao}
% ---



% ---------------------------------------
% Dedicatória
% ---------------------------------------
\vspace*{\fill}

\begin{dedicatoria}
   \vspace*{10 cm}
   \begin{flushright}
   \begin{minipage}[t]{7cm}
     Elemento opcional utilizado pelo autor para registrar homenagens ou dedicatórias à determinada(s) pessoa(s).
   \end{minipage}
   \end{flushright}
\end{dedicatoria}
% ---


% ---------------------------------------
% Agradecimentos
% ---------------------------------------
\begin{center}
	\textbf{AGRADECIMENTOS}
\end{center}

Elemento opcional utilizado pelo autor para registrar agradecimento às pessoas que contribuíram para a elaboração do trabalho.


\newpage
% ---


% -----------------------------------------
% Epígrafe
% -----------------------------------------
\vspace*{\fill}

\begin{epigrafe}
    \vspace*{10cm}
	\begin{flushright}
	  \begin{minipage}[t]{7cm}
		\textit{Elemento opcional utilizado pelo autor para apresentar uma citação relacionada com a matéria tratada no corpo do trabalho.}
		
	 \end{minipage}
	\end{flushright}
\end{epigrafe}
% ---


% -----------------------------------------
% RESUMOS
% -----------------------------------------
% resumo em português
\begin{center}
	\textbf{RESUMO}
\end{center}



  \vspace{\onelineskip}

\noindent Elemento obrigatório que contém a apresentação concisa dos pontos relevantes do trabalho, fornecendo uma visão rápida e clara do conteúdo e das conclusões do mesmo. A apresentação e a redação do resumo devem seguir os requisitos estipulados pela NBR 6028 (ABNT, 2003). Deve descrever de forma clara e sintética a natureza do trabalho, o objetivo, o método, os resultados e as conclusões, visando fornecer elementos para o leitor decidir sobre a consulta do trabalho no todo.

 \vspace{\onelineskip}
    
 \noindent\textbf{Palavras-chave:} Palavra 1. Palavra 2. Palavra 3. Palavra 4. Palavra 5.

\newpage


% resumo em inglês
 \begin{otherlanguage*}{english}

\begin{center}
	\textbf{ABSTRACT}
\end{center}



 \vspace{\onelineskip}

\noindent Elemento obrigatório para todos os trabalhos de conclusão de curso. Opcional para os demais trabalhos acadêmicos, inclusive para artigo científico. Constitui a versão do resumo em português para um idioma de divulgação internacional. Deve aparecer em página distinta e seguindo a mesma formatação do resumo em português.
 \vspace{\onelineskip}

\noindent\textbf{Keywords:} Keyword 1. Keyword 2. Keyword 3. Keyword 4. Keyword 5.

 \end{otherlanguage*}

\newpage


% ---


% -----------------------------------------
% inserir lista de ilustrações
% -----------------------------------------
\renewcommand\listfigurename{{\fontsize{12pt}{\baselineskip}\normalfont \bfseries LISTA DE ILUSTRAÇÕES}}
\pdfbookmark[0]{\listfigurename}{lof}
\listoffigures*
\cleardoublepage
% ---


% -----------------------------------------
% inserir lista de tabelas
% -----------------------------------------
\renewcommand\listtablename{{\fontsize{12pt}{\baselineskip}\normalfont \bfseries LISTA DE TABELAS}}
\pdfbookmark[0]{\listtablename}{lot}
\listoftables*
\cleardoublepage
% ---



% ----------------------------------------
% inserir lista de abreviaturas e siglas
% ----------------------------------------
\renewcommand\listadesiglasname{{\fontsize{12pt}{\baselineskip}\normalfont \bfseries LISTA DE ABREVIATURAS E SIGLAS}}
\begin{siglas}
  \item[ABNT]  Associação Brasileira de Normas Técnicas 
  \item[ENEM]  Exame nacional do Ensino Médio
  \item[Fig.]  Area of the $i^{th}$ component
  \item[FLT]   Fermat's Last Theorem  
  \item[MEC]   Ministério da Educação
  \end{siglas}
% ---


% ----------------------------------------
% inserir lista de símbolos
% ----------------------------------------
\renewcommand\listadesimbolosname{{\fontsize{12pt}{\baselineskip}\normalfont \bfseries LISTA DE SÍMBOLOS}}
\begin{simbolos}
  \item[$ \mathbb{N} $] Conjunto dos números naturais 
  \item[$ \mathbb{Z} $] Conjunto dos números inteiros 
  \item[$ \mathbb{Q} $] Conjunto dos números racionais 
  \item[$ \mathbb{R} $] Conjunto dos números reais 
 %\item[$ \mathbb{C} $] Conjunto dos números complexos
 %\item[$ C_c(R) $] Conjunto das funções $f: R \rightarrow \C$ contínuas com suporte compacto em um conjunto $R$.
\end{simbolos}
% ---


% ----------------------------------------
% inserir o sumario
% ----------------------------------------
\renewcommand\contentsname{{\fontsize{12pt}{\baselineskip}\normalfont \bfseries SUMÁRIO}}
\pdfbookmark[0]{\contentsname}{toc}
\tableofcontents*
\cleardoublepage
% ---



% ----------------------------------------------------------
% ELEMENTOS TEXTUAIS
% ----------------------------------------------------------

\textual

% ----------------------------------------------------------
% Introdução
% ----------------------------------------------------------
  \pagestyle{meuestilo}

\chapter{INTRODUÇÃO} % LETRAS MAIÚSCULAS


A introdução apresenta os objetivos do trabalho, bem como as razões de sua elaboração. Tem caráter didático de apresentação.
Deve abordar:

    a) o problema de pesquisa, proposto de forma clara e objetiva;
    
    b) os objetivos, delimitando o que se pretende fazer;
    
    c) a justificativa, destacando a importância do estudo;
    
    d) apresentar as definições e conceitos necessários para a compreensão do estudo;
    
    e) apresentar a forma como está estruturado o trabalho e o que contém cada uma de suas partes.
    
O desenvolvimento é a demonstração lógica de todo o trabalho, detalha a pesquisa ou o estudo realizado. Explica, discute e demonstra a pertinência das teorias utilizadas na exposição e resolução do problema. 

O desenvolvimento pode ser subdivido em seções e subseções com nomenclaturas definidas pelo autor conforme conteúdo apresentado. 
% ---

% INICIO CAPITULOS
\chapter{O CONJUNTO DOS NÚMEROS NATURAIS E OS AXIOMAS DE PEANO}
Os números naturais foram usados por muito tempo como algo por si só e sem fundamentação por outras coisas que os justificassem. Não houve preocupação na sua formalização até recentemente, quando Hermann Grassmann mostrou na década de 1860 que vários fatos de aritmética básica podiam ser obtidas através de alguns "fatos básicos" (ou axiomas) da aritmética.
Houveram algumas tentativas de formalização do conjunto, e no século XIX o italiano Giuseppe Peano formulou seus axiomas utilizando o trabalho de Grassmann e Richard Dedekind como base.

Peano em seu 'Arithmetices Principia, Nova Methodo Exposita' fundamentou a sua aritmética com $9$ axiomas, que em notação moderna são:
\begin{peano}
    $1 \in \N$
\end{peano}
\begin{peano}
    $a \in \N \rightarrow a = a$
\end{peano}
\begin{peano}
    $a,b,c \in \N \rightarrow a = b \iff b = a$
\end{peano}
\begin{peano}
    $a,b,c \in \N \rightarrow (a = b \land b = c \rightarrow a = c) $
\end{peano}
\begin{peano}
    $a = b \land b \in \N \rightarrow a \in \N$
\end{peano}
\begin{peano}
    $a \in \N \rightarrow sa \in \N$
\end{peano}
\begin{peano}
    $a,b \in \N \rightarrow (a = b \iff sa = sb)$
\end{peano}
\begin{peano}
    $a \in \N \rightarrow sa \neq 1$
\end{peano}
\begin{peano}
    $A \in \textbf{SET} \land 1 \in A \land (x \in \N \land x \in A \rightarrow sx \in A)$
\end{peano}

Esses axiomas relacionam conceitos primitivos. Conceitos primitivos podem ser entendidos como base para fazer proposições e outros conceitos. Peano escolheu 3 conceitos primitivos (nós faremos igualmente), que listaremos e forneceremos uma 'interpretação' inicial:
\begin{itemize}
    \item $1$: Algum elemento.
    \item Número natural: Algo que é número e que podemos chamar de 'natural'.
    \item sucessor: uma relação entre exatamente 2 elementos.
\end{itemize}

*citação do jogo de xadrez das peças de geometria*


Para este trabalho, apresentaremos 4 axiomas, mas para isso precisaremos de algumas definições sobre \emph{funções}:

\begin{defi}{Produto cartesiano}
    O produto cartesiano de X por Y é o conjunto $\{(x,y) : x \in X \land y \in Y\}$. Denotamos por $X \times Y$.
\end{defi}
\begin{obs}
    O elemento $(x,y)$ é chamado de par ordenado, e ocorre que $(a,b) = (c,d) \iff a = c \land b = d$. Fica subentendido na definição anterior que é o conjunto de todos os pares ordenados com $x \in X$ e $y \in Y$.
\end{obs}

\begin{defi}{Relação}
    Uma relação é qualquer subconjunto do produto cartesiano.
\end{defi}




\begin{axi}\label{axi-existe-n-s}
    Existe um conjunto de exatamente todos os números naturais, que será denotado por $\N$, e existe uma função $s: \N \rightarrow \N$, que é a relação "sucessor". \footnote{O leitor poderia perguntar, como podemos fazer um produto cartesiano de conjuntos que não conhecemos?}
\end{axi} % Não teria que especificar que é o mesmo 's' em todos os axiomas?
\begin{axi}\label{axi-um-natural}
    Um é um número natural, isto é, $1 \in \N$
\end{axi}
\begin{axi}\label{axi-um-nao-sucessor}
    Um não é sucessor de nenhum número, isto é, $1 \not \in Im(s)$ ou ainda, $\not \exists k \in \N : sk = 1$
\end{axi}
\begin{axi}\label{axi-s-injetora}
    $s$ é injetora, isto é, $sa = sb \implies a = b$
\end{axi}
\begin{obs}
    Vale notar a contra positiva que estabelece, nesse caso: $a \neq b \implies sa \neq sb$
\end{obs}
\begin{axi}\label{axi-ind-finita}
    Se $\S$ é um subconjunto de $\N$, caso $1 \in \S$ e se para todo $k$ em $\S\ sk$ também esteja em $\S$, então $\S = \N$, isso é o mesmo que colocar:
     $\S \subseteq \N \land 1 \in S \land ( k \in \S \implies sk \in \S) \implies \S = \N$
\end{axi}
\begin{obs}
    Denotaremos o sucessor de $k$ por $sk$ ao invés de $s(k)$ por questão de brevidade. Além disso, ao longo do trabalho, colocaremos poucos parênteses a fim de evitar sobrecarregar a notação desnecessariamente.
\end{obs}
Este último axioma é chamado de axioma da indução finita.

Devemos notar que o axioma \ref{axi-um-natural} garante $\N \neq \emptyset $. Além dele, como $1 \in \N$ e pelo axioma \ref{axi-existe-n-s} temos que $s1 \in \N$. Analogamente, $ss1 \in \N$.

\begin{lema}{Nenhum número natural é seu próprio sucessor}
    $k \in \N \implies k \neq sk $
\end{lema}
\begin{dem}
    Faremos por indução. Seja $\S \subseteq \N$ tal que $k \in \S \implies k \neq sk$.\\
    O axioma \ref{axi-um-natural} e \ref{axi-um-nao-sucessor} garantem que $1 \in \S$.\\
    Supondo que $\exists k \in \S : k \neq sk$, queremos provar que $s(k) \neq s(sk)$, que vem do axioma \ref{axi-s-injetora}.
\end{dem}
\begin{teo}
    Todo número natural, exceto o $1$ é sucessor de algum outro número natural, que é único, isto é, $\forall k (k \in \N \land k \neq 1 \implies \exists! r : sr = k)$
\end{teo}
\begin{dem}
    A prova é feita por indução. Seja $\S = \{x \in \N : x \neq 1 \implies (\exists!r \in \N : sr = k)\}$.\\
    Claro que $1 \in \S$. Com um $k$ em  \S, queremos mostrar que $sk \in \S$. \\
    É imediato que $sk \in \S$ pois, se $k \in \S$, para o sucessor $sk$ existe um número $k$ tal que $\underbrace{sk}_{\text{Da regra de } \S} = \underbrace{sk}_{\text{Axioma }\ref{axi-ind-finita}}$.\\
    A unicidade de $r$ também é imediata, pois $sa = sb = k \implies a = b$ pelo axioma \ref{axi-s-injetora}.
\end{dem}
\begin{obs}
    Diremos que o número que o número $a$ é antecessor de $k = sa$.
\end{obs}
\begin{defi}{Adição}\label{adicao}
A adição (x,y) $\rightarrow x + y$ em \N é definida como: 
    \begin{itemize}
        \item $a + 1 = sa$
        \item $a + sb = s(a+b)$
    \end{itemize}
\end{defi}

Listaremos algumas propriedades da adição e depois provaremos essas afirmações. Sejam $a, b \in \N$
\begin{itemize}
    \item Fechamento: $a + b \in \N$
	\item Associativa: $(a + b) + c = a + (b + c)$
	\item Comutativa: $a + b =  b + a$
	\item Inexistência de neutro: $\not\exists e \in \N : \forall a, a + e = e + a = a$
\end{itemize}

\begin{dem}{Fechamento:}
    
\end{dem}


% FIM CAPITULOS


% ---------------------------------------------------------
%% primeiro capítulo
%% ---------------------------------------------------------
%\chapter{CAPÍTULO FORMATO E FONTES} % LETRAS MAIÚSCULAS
%
%Você deve sempre mudar de pagina ao iniciar um novo capitulo de seu trabalho.
%
%Para os títulos utilizar o mesmo tipo e tamanho de fonte do texto. 
%
%Usar fonte do tipo Arial ou Times New Roman com tamanho 12, espaçamento entre as linhas 1,5 cm, margem direta 2 cm, margem esquerda 3,0 cm, margem superior 3 cm e inferior com 2cm.
%
%Citações com mais de três linhas, legendas de ilustrações e tabelas, rodapés devem ser digitados em tamanho 10 pt. Citações com mais de 3 linhas devem ter recuo de 4 cm a partir da esquerda.
%
%Abaixo um exemplo de como citar a bibliografia no texto
%
%\begin{citacao}
% Tal disciplina é caracterizada pelo estudo de funções reais ligadas aos processos de limite e pelo fato de seu conteúdo apoiar-se sobre quase todo o saber escolar ensinado aos alunos até então (BARROSO et al, 2009, p.).
%\end{citacao}
%
%Para as divisões principais utilizar maiúsculo e negrito, divisão secundária em letras maiúsculas sem negrito, seção terciária fonte normal com as primeiras letras em maiúsculo e em negrito, seção quartenária utilizar fonte normal somente a primeira letra da primeira palavra em maiúsculo, conforme ex.
%
%%---
%
%% ---
%\section{EXEMPLOS} % LETRAS MAIÚSCULAS
%% ---
%
%
%Nesta seção veremos alguns exemplos de teorema, proposição, observação, lema, formas de numeração, equações, etc.  O texto a seguir não faz muito sentido, são pequenas partes tiradas da minha dissertação de mestrado.
%
%\begin{defi}
%Uma álgebra é um espaço vetorial $A$ sobre o corpo dos números complexo $\C$, munida com uma multiplicação que satisfaz:
%\begin{itemize}
%\item $a(bc)=(ab)c$
%\item $(a+b)c = ac+bc$, \,\,\, $a(b+c)=ab+ac$
%\item $\lambda(ab)=(\lambda a)b=a(\lambda b)$
%\end{itemize}
%para quaisquer $a, b, c \in A$ e $\lambda \in \C.$
%\end{defi}
%
%\begin{obs}
%Se $A$ é um espaço de Banach com relação a uma norma, $\Vert \cdot \Vert $, tal que $$\Vert ab \Vert \leq \Vert a \Vert \Vert b \Vert$$ para quaisquer $a, b \in A$, então $A$ é uma álgebra de Banach.
%\end{obs}
%
%
%\begin{defi} 
%Uma involução em uma álgebra $A$ é uma aplicação $\ast: A \rightarrow A$ que satisfaz:
%\begin{enumerate}
%\item $(a+b)^{\ast}= a^{\ast}+b^{\ast}$
%\item $(\lambda a )^{\ast} = \overline{\lambda} a^{\ast}$
%\item $(ab)^{\ast}=b^{\ast}a^{\ast}$
%\item ${a^{\ast}}^{\ast}=a$
%\end{enumerate}
%para quaisquer $a, b \in A$ e $\lambda \in \C.$ Uma álgebra $A$ munida de uma involução é chamada de $\ast$ - álgebra.
%\end{defi}
%% ---
%
%
%Abaixo um exemplo de como citar a bibliografia no texto.
%
%O exemplo do corpo quadrático $\Q[\sqrt{-5}]$ mostra que o anel dos inteiros algébrico de um corpo de número algébricos nem sempre  fatorial.  De acordo com \cite{endler2006}, para consertar esta falha, foi introduzido por Kummer a noção de "número ideal", que deu origem à noção de "ideal" devida a Dedekind.
%
%
%% ---
%\subsection{Mais exemplos} %Letras Maiúsculas e Minículas
%% ---
%
%
%\begin{defi}\
%\begin{enumerate}
%\renewcommand{\labelenumi}{(\alph{enumi})} % numeração (a), (b), (c) ...
%\item  seminorma sobre um espaço vetorial $X$ é uma função $p: X \rightarrow \R_{+}$ tal que
%\begin{description}
%\item[$\star$] $p(x+y)\leq p(x)+p(y)$
%\item[$\star$] $p(\lambda x)= |\lambda|p(x)$
%\end{description}
%para todo $x, y \in X$ e $\lambda \in \C$. \\
%\item  Uma C*-seminorma $p$ sobre uma $\ast$ -álgebra $A$ é uma seminorma que satisfaz, para todo $x, y \in A$,
%\begin{description}
%\item[$\diamond$] $p(xy)\leq p(x)p(y)$
%\item[$\diamond$] $p(x^*)=p(x)$
%\item[$\diamond$] $p(x^*x)=p(x)^2.$
%\end{description}
%\end{enumerate}
%\end{defi}
%
%
%\begin{lema} 
%Seja $G$ um grupóide.
%\renewcommand{\labelenumi}{(\roman{enumi})} % numeração (i), (ii), ...
%\begin{enumerate}
%  \item $G^0 = \{x \in G \,\, : \,\, x=x^{-1}\}$.
%  \item $G^2=\{(x,y) \in G \times G \,\, : \,\, s(x)=r(y)\}$.
%\end{enumerate}
%\end{lema}
%
%
%\noindent \begin{dem} 
%Se $x \in G^0$ então existe $y \in G$ tal que $x=yy^{-1}$, logo $x^{-1}=(yy^{-1})^{-1}= yy^{-1}=x$. 
%\par $(\supseteq)$ Seja $x \in G$ tal que $x=x^{-1}$. Suponha que não existe $y \in G$ tal que $x=yy^{-1}$, ou seja, $x \neq yy^{-1}$ para todo $y \in G$. Então $x^{-1} \neq (yy^{-1})^{-1}=yy^{-1}=x$, implica que $x^{-1}\neq x$. Absurdo! Portanto existe $y \in G$ tal que $x=yy^{-1}$ e $x \in G^0$...
%\end{dem}
%
%
%Equações:
%
%\begin{equation}\label{produto}  
%f \ast g (x,z) = \sum_{y \in [x]} f(x,y)g(y,z),
%\end{equation}
%
%\begin{equation}\label{involucao} 
%f^*(x,y)= \overline{f(x,y)}.
%\end{equation}
%
%
%\begin{prop}\label{continuas}
%Seja $R$ uma relação étale então as aplicações range e source são contínuas.
%\end{prop}
%
%
%\begin{obs} 
%Pela Proposição \ref{continuas}, temos que $r$ é uma aplicação contínua, então $r|_{\bigtriangleup}$ também é uma aplicação contínua. Portanto $\bigtriangleup = \left(r|_{\bigtriangleup}\right) ^{-1}(X)$ é aberta em $(R, \tau)$.
%\end{obs}
%
%
%Da equação \ref{involucao}, segue que
%\begin{align*}  
%        X & = \sum_{z \in [y]}\overline{f(y,z)}\,\overline{g(z,x)}= \sum_{z \in [y]} f^*(z,y)g^*(x,z)\\
%          & = \sum_{z\in [y]}g^*(x,z)f^*(z,y) = g^* \ast f^* (x,y)
%\end{align*}
%
%
%\begin{teo}
%Para uma relação étale $R$ e  $x \in X$ a representação  $\lambda_x$ definida acima é uma representação não-degenerada e limitada de $C_c(R)$.
%\end{teo}
%
%
%
%
%% ----------------------------------------------------------
%% segundo capítulo
%% ----------------------------------------------------------
%\chapter{FIGURAS E TABELAS}
%
%Exemplos de inserção de figura e tabela.
%
%% ---
%\section{INSERIR FIGURAS}
%% ---
%
%As identificações de qualquer tipo de ilustração devem aparecer na parte superior, precedida da palavra designativa, seguida de seu número de ordem de ocorrência no texto, em algarismos arábicos, travessão e do respectivo título (texto com espaçamento entrelinhas simples e fonte 12). Na parte inferior deve-se indicar a fonte consultada (tamanho 10). Todas as figuras devem ser centralizadas em relação a margem. A figura deve ser colocada após sua citação no texto, deixando-se entrelinhas 1,5 entre o texto e a figura. Após a figura, o texto segue a um espaço de 1,5. Veja na Figura 1 um exemplo de figura:
%
%Seguem dois exemplos de figuras:
%
%Aqui a figura fica dentro de um retângulo (caixa).
%
%%\begin{figure}[!h]
%%	\centering  
%%	\caption{Diagrama de Bratteli}
%%	\label{diagrama}
%%	\fbox{\includegraphics[width=10cm,height=8cm]{diagrama.eps}}
%%	\fonte{{\small Produção do autor}}    
%%\end{figure}
%
%Fora da caixa.
%
%%\begin{figure*}[!h]
%%	\centering
%%	\caption{Título da Figura}
%%	\label{referencia}
%%	\includegraphics[width=7cm]{exemplo1.eps}
%%	\fonte{{\small Produção do autor}}
%%\end{figure*}
%
%
%
%
%%---------
%\section{CRIAR TABELA}
%
%Segundo a ABNT em tabelas não deve-se utilizar linhas verticais para separar as colunas, conforme exemplo abaixo na Tabela \ref{tabelaABNT}. 
%
%\begin{table}[!h]
%	\centering
%	\caption{Percentual de alunos aprovados e reprovados na disciplina de Cálculo I por curso}
%	\label{tabelaABNT}
%	\begin{tabular}{l c c} \hline
%		\textbf{Curso}   &   \textbf{Aprovados (\%)}   &   \textbf{Reprovados (\%)}  \\  \hline
%		Engenharia Civil   &  65  &  35   \\   
%		Engenharia Elétrica &  40  &  60   \\ 
%		Engenharia Mecânica   & 68 & 32 \\
%		Engenharia de Produção & 34 &  66\\ \hline
%	\end{tabular}
%	\fonte{{\small Acervo do autor}}
%	
%\end{table}
%
%
%
%
%

% ------------------------------------------------------------
% Conclusão
% ------------------------------------------------------------
\chapter*[Conclusão]{CONSIDERAÇÕES FINAIS}
\addcontentsline{toc}{chapter}{CONSIDERAÇÕES FINAIS}

É a parte final do texto. Deve retomar o problema inicial, revendo os objetivos
e comentando se foram atingidos ou não, enunciando as principais contribuições.
Sintetiza as principais idéias, bem como os resultados, avaliando pontos positivos e
negativos. Geralmente inclui recomendações e/ou sugestões. 

% ----------------------------------------------------------
% Referências bibliográficas
% ----------------------------------------------------------
	\chapter*[Referências]{REFERÊNCIAS}
\addcontentsline{toc}{chapter}{REFERÊNCIAS}




\bibliography{BARBOSA, João Lucas Marques. Geometria Euclidiana Plana. 11. ed. Rio de Janeiro: Sbm, 2012.}





% ----------------------------------------------------------
% Apêndices
% ----------------------------------------------------------


\chapter*[Apêndice]{APÊNDICE A - TÍTULO}
\addcontentsline{toc}{chapter}{APÊNDICE A - TÍTULO}
% ----------------------------------------------------------

Lalalalalala....

% ----------------------------------------------------------
%\chapter{Titulo do Segundo Apêndice}
% ----------------------------------------------------------

%Lalalalalala...


% ---


% ----------------------------------------------------------
% Anexos
% ----------------------------------------------------------



% ---
\chapter*[Anexos]{ANEXO A - TÍTULO}
\addcontentsline{toc}{chapter}{ANEXO A - TÍTULO}
% ---

Lalalalalala...

% ---
%\chapter{Nome do anexo 2}
% ---

%lalalalalalalala....



%---------------------------------------------------------------------
% INDICE REMISSIVO
%---------------------------------------------------------------------

\printindex

\end{document}
